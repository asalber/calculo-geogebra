% !TEX root = ../practicas_geogebra.tex
% Author: Alfredo Sánchez Alberca (asalber@ceu.es)
\chapter{Límites y Continuidad}

\section{Fundamentos teóricos}
En esta práctica se introducen los conceptos de límite y continuidad de una función real, ambos muy relacionados.

\subsection{Límite de una función en un punto}
El concepto de límite está muy relacionado con el de proximidad y tendencia de una serie de valores. De manera informal, diremos que $l\in \mathbb{R}$ es el \emph{límite} de una función $f(x)$ en un punto $a\in \mathbb{R}$, si $f(x)$ tiende o se aproxima cada vez más a $l$, a medida que $x$ se aproxima a $a$, y se escribe
\[ \lim_{x\rightarrow a} f(x)=l.\]

Si lo que nos interesa es la tendencia de $f(x)$ cuando nos aproximamos al punto $a$ sólo por un lado, hablamos de \emph{límites laterales}. Diremos que $l$ es el \emph{límite por la izquierda} de una función $f(x)$ en un punto $a$, si $f(x)$ tiende o se aproxima cada vez más a $l$, a medida que $x$ se aproxima a $a$ por la izquierda, es decir con valores $x<a$, y se denota por
\[ \lim_{x\rightarrow a^-} f(x)=l.\]
Del mismo modo, diremos que $l$ es el \emph{límite por la derecha} de una función $f(x)$ en un punto $a$, si $f(x)$ tiende o se aproxima cada vez más a $l$, a medida que $x$ se aproxima a $a$ por la derecha, es decir con valores $x>a$, y se denota por
\[ \lim_{x\rightarrow a^+} f(x)=l.\]

Por supuesto, para que exista el límite global de la función $f(x)$ en el punto $a$, debe existir tanto el límite por la izquierda, como el límite por la derecha, y ser iguales, es decir
\[
\left.
\begin{array}{l}
\displaystyle \lim_{x\rightarrow a^-} f(x)=l \\
\displaystyle \lim_{x\rightarrow a^+} f(x)=l
\end{array}
\right\}
\Longrightarrow
\lim_{x\rightarrow a} f(x)=l.
\]

\subsection{Álgebra de límites}
Para el cálculo práctico de límites, se utiliza el siguiente
teorema, conocido como Teorema de \emph{Álgebra de Límites}.
% !TEX root = ../practicas_geogebra.tex

% % !TEX root = ../practicas_geogebra.tex
% e $\lim_{x\rightarrow
% % !TEX root = ../practicas_geogebra.tex
% , entonces se cumple
% % !TEX root = ../practicas_geogebra.tex

% % !TEX root = ../practicas_geogebra.tex

% % !TEX root = ../practicas_geogebra.tex
% x)\pm g(x))=l_1\pm l_2$.
% % !TEX root = ../practicas_geogebra.tex
% x)\cdot g(x))=l_1\cdot l_2$.
% % !TEX root = ../practicas_geogebra.tex
% rac{f(x)}{g(x)}=\dfrac{l_1}{l_2}$ si $l_2\neq 0$.
% % !TEX root = ../practicas_geogebra.tex
% mentales}


% \subsection{Así% !TEX root = ../practicas_geogebra.tex
% Author: Alfredo Sánchez Alberca (asalber@ceu.es)
\chapter{Funciones Elementales}

% \section{Fundamentos teóricos}

% En esta práctica se introducen los conceptos básicos sobre funciones reales de variable real, esto es, funciones
% \[f:\mathbb{R}\rightarrow \mathbb{R}.\]

% \subsection{Dominio e imagen}

% El \emph{Dominio} de la función $f$ es el conjunto de los números reales $x$ para los que existe $f(x)$ y se designa mediante $\dom f$.

% La \emph{Imagen} de $f$ es el conjunto de los números reales $y$ para los que existe algún $x\in \mathbb{R}$ tal que $f(x)=y$, y se denota por $\im f$.


% \subsection{Signo y crecimiento}
% El \emph{signo} de la función es positivo $(+)$ en los valores de $x$ para los que $f(x)>0$ y negativo $(-)$ en los que $f(x)<0$.
% Los valores de $x$ en los que la función se anula se conocen como \emph{raíces} de la función.

% Una función $f(x)$ es \emph{creciente} en un intervalo $I$ si $\forall\, x_1, x_2 \in I$ tales que $x_1<x_2$ se verifica que $f(x_1)\leq f(x_2)$.

% Del mismo modo, se dice que una función $f(x)$ es \emph{decreciente} en un intervalo $I$ si $\forall\, x_1, x_2 \in I$ tales que $x_1<x_2$ se verifica que $f(x_1)\geq f(x_2)$. En la figura~\ref{g:crecimiento} se muestran estos conceptos.

% \begin{figure}[h!]
% 	\centering \subfigure[Función creciente.] {\label{g:funcion_creciente}
% 		\scalebox{1}{\input{img/funciones_elementales/funcion_creciente}}}\qquad
% 	\subfigure[Función decreciente.]{\label{g:funcion_decreciente}
% 		\scalebox{1}{\input{img/funciones_elementales/funcion_decreciente}}}
% 	\caption{Crecimiento de una función.}
% 	\label{g:crecimiento}
% \end{figure}


% \subsection{Extremos Relativos}
% Una función $f(x)$ tiene un \emph{máximo relativo} en $x_0$ si existe un entorno $A$ de $x_0$ tal que $\forall x \in A$
% se verifica que $f(x)\leq f(x_0)$.

% Una función $f(x)$ tiene un \emph{mínimo relativo} en $x_0$ si existe un entorno $A$ de $x_0$ tal que $\forall x\in A$
% se verifica que $f(x)\geq f(x_0)$.

% Diremos que la función $f(x)$ tiene un \emph{extremo relativo} en un punto si tiene un \emph{máximo o mínimo relativo}
% en dicho punto. Estos conceptos se muestran en la figura~\ref{g:extremos}.

% \begin{figure}[h!]
% 	\centering \subfigure[Máximo relativo.] {\label{g:maximo}
% 		\scalebox{1}{\input{img/funciones_elementales/maximo}}}\qquad
% 	\subfigure[Mínimo relativo.]{\label{g:minimo}
% 		\scalebox{1}{\input{img/funciones_elementales/minimo}}}
% 	\caption{Extremos relativos de una función.}
% 	\label{g:extremos}
% \end{figure}

% Una función $f(x)$ está \emph{acotada superiormente} si $\exists K\in\mathbb{R}$ tal que $f(x)\leq K$ $\forall x \in \dom f$. Análogamente, se dice que una función $f(x)$ está \emph{acotada inferiormente} si $\exists K\in\mathbb{R}$ tal que $f(x)\geq K$ $\forall x \in \dom f$.

% Una función $f(x)$ está \emph{acotada} si lo está superior e inferiormente, es decir si $\exists K\in\mathbb{R}$ tal que $|f(x)|\leq K$ $\forall x \in \dom f$.


% \subsection{Concavidad}

% De forma intuitiva se puede decir que una función $f(x)$ es \emph{cóncava} en un intervalo $I$ si $\forall\, x_1, x_2
% \in I$, el segmento de extremos $(x_1,f(x_1))$ y $(x_2,f(x_2))$ queda por encima de la gráfica de $f$.

% Análogamente se dirá que es \emph{convexa} si el segmento anterior queda por debajo de la gráfica de $f$.

% Diremos que la función $f(x)$ tiene un \emph{punto de inflexión} en $x_0$ si en ese punto la función pasa de cóncava a
% convexa o de convexa a cóncava. Estos conceptos se ilustran en la figura~\ref{g:concavidad}.

% \begin{figure}[h!]
% 	\centering \subfigure[Función cóncava.] {\label{g:funcion_convexa}
% 		\scalebox{1}{\input{img/funciones_elementales/funcion_convexa}}}\qquad
% 	\subfigure[Función convexa.]{\label{g:funcion_concava}
% 		\scalebox{1}{\input{img/funciones_elementales/funcion_concava}}}
% 	\caption{Concavidad de una función.}
% 	\label{g:concavidad}
% \end{figure}

% \subsection{Asíntotas}

% La recta $x=a$ es una \emph{asíntota vertical} de la función $f(x)$ si al menos uno de los límites laterales de $f(x)$ cuando $x$ tiende hacia $a$ es $+\infty$ o $-\infty$, es decir cuando se verifique alguna de las siguientes igualdades
% \[
% 	\ \lim_{x\rightarrow a^{+}}f(x)=\pm\infty   \quad \textrm{o} \quad
% 	\lim_{x\rightarrow a^{-}}f(x)=\pm\infty
% \]

% La recta $y=b$ es una \emph{asíntota horizontal} de la función $f(x)$ si alguno de los límites de $f(x)$ cuando $x$ tiende hacia $+\infty$ o $-\infty$ es igual a $b$, es decir cuando se verifique
% \[
% 	\ \lim_{x\rightarrow -\infty }f(x)=b    \quad \textrm{o} \quad
% 	\ \lim_{x\rightarrow +\infty }f(x)=b
% \]

% La recta $y=mx+n$ es una \emph{asíntota oblicua} de la función $f(x)$ si alguno de los límites de $f(x)-(mx+n)$ cuando $x$ tiende hacia $+\infty$ o $-\infty$ es igual a 0, es decir si

% \[
% 	\ \lim_{x\rightarrow -\infty }{(f(x)-mx)}=n    \quad \textrm{o} \quad
% 	\ \lim_{x\rightarrow +\infty }{(f(x)-mx)}=n
% \]

% En la figura~\ref{g:asintotas} se muestran los distintos tipos de asíntotas.

% \begin{figure}[h!]
% 	\centering \subfigure[Asíntota horizontal y vertical.] {\label{g:asintotahorizontalyvertical}
% 		\scalebox{1}{\input{img/funciones_elementales/asintota_vertical}}}\qquad\qquad
% 	\subfigure[Asíntota vertical y oblicua.]{\label{g:asintotaoblicua}
% 		\scalebox{1}{\input{img/funciones_elementales/asintota_oblicua}}}
% 	\caption{Tipos de asíntotas de una función.}
% 	\label{g:asintotas}
% \end{figure}


% \subsection{Periodicidad}
% Una función $f(x)$ es \emph{periódica} si existe $h\in\mathbb{R^{+}}$ tal que \[f(x+h)=f(x)\  \forall x\in \dom f\] siendo el período $T$ de la función, el menor valor $h$ que verifique la igualdad anterior.

% En una función periódica, por ejemplo $f(x)=A\sen(wt)$, se denomina \emph{amplitud} al valor de $A$, y es la mitad de la diferencia entre los valores máximos y mínimos de la función. En la figura~\ref{g:periodoyamplitud} se ilustran estos conceptos.

% \begin{figure}[h!]
% 	\centering
% 	\scalebox{0.8}{\input{img/funciones_elementales/funcion_periodica}}
% 	\caption{Periodo y amplitud de una función periódica.}
% 	\label{g:periodoyamplitud}
% \end{figure}

% \clearpage
% \newpage

\section{Ejercicios resueltos}

\begin{enumerate}[leftmargin=*]
\item Se considera la función
      \[
      f(t)=\frac{t^{4} +19\cdot t^{2} - 5}{t^{4} +9\cdot t^{2} - 10}.
      \]

      Representarla gráficamente y determinar a partir de dicha representación:

      \begin{enumerate}
      \item  Dominio.
            \begin{indication}
            \begin{enumerate}
            \item Para representarla gráficamente, introducir la función en la barra de \field{Entrada} de la \field{Vista CAS} y activar la \field{Vista Gráfica}.
            \item Para determinar el dominio tan sólo hay que determinar los valores de $x$ en los que existe la función.
            \item Recordar que, tanto para éste como para el resto de los apartados del ejercicio, pretendemos llegar a conclusiones aproximadas que tan sólo sacamos del análisis de la gráfica.
            \end{enumerate}
            \end{indication}

      \item  Imagen.
            \begin{indication}
            Fijarse en los valores de la variable $y$ hasta los que llega la función.
            \end{indication}

      \item  Asíntotas.
            \begin{indication}
            Son las líneas rectas, ya sea horizontales, verticales u oblicuas, hacia las que tiende la función.
            \end{indication}

      \item  Raíces.
            \begin{indication}
            Son los valores de la variable $x$, si los hay, en los que la función vale 0.
            \end{indication}

      \item Signo.
            \begin{indication}
            Hay que determinar, aproximadamente, por un lado los intervalos de variable $x$ en los que la función es positiva, y por el otro aquellos en
            los que es negativa.
            \end{indication}

      \item  Intervalos de crecimiento y decrecimiento.
            \begin{indication}
            De nuevo, por un lado hay que determinar los intervalos de variable $x$ en los que a medida que crece $x$ también lo hace $y$, que serían
            los intervalos de crecimiento, y también aquellos otros en los que a medida que crece $x$ decrece $y$, que serían los intervalos de
            decremimiento.
            \end{indication}

      \item Intervalos de concavidad y convexidad.
            \begin{indication}
            Para los intervalos de concavidad y convexidad, nos fijamos en el segmento de línea recta que une dos puntos cualquiera del intervalo. Si
            dicho segmento queda por encima de la gráfica, entonces la función es cóncava en el intervalo, mientras que si queda por debajo, entonces es
            convexa en el mismo.
            \end{indication}

      \item Extremos relativos.
            \begin{indication}
            Determinamos, aproximadamente, los puntos en los que se encuentran los máximos y mínimos relativos de la función.
            \end{indication}

      \item Puntos de inflexión.
            \begin{indication}
            Determinamos, aproximadamente, los puntos en los que la función cambia de curvatura, de cóncava a convexa o a la inversa.
            \end{indication}
      \end{enumerate}

\item Representar en una misma gráfica las funciones $2^{x}, e^{x}, 0.7^{x}, 0.5^{x}$. A la vista de las gráficas obtenidas, indicar cuáles
      de las funciones anteriores son crecientes y cuáles son decrecientes.
      \begin{indication}
      Introducir cada función en la barra de \field{Entrada} de la \field{Vista CAS}.
      \end{indication}

      ¿En general, para qué valores de $a$ será la función creciente? ¿Y para qué valores de $a$ será decreciente? Probar con
      distintos valores de $a$ representando gráficamente nuevas funciones si fuera necesario.


\item Representar en una misma gráfica las funciones siguientes, indicando su período y amplitud.
      \begin{enumerate}
      \item $\sen{x}$, $\sen{x}+2$, $\sen{(x+2)}$.
      \item $\sen{2x}$, $2\sen{x}$, $\sen\frac{x}{2}$.
            \begin{indication}
            Introducir cada función en la barra de \field{Entrada} de la \field{Vista CAS}.
            \end{indication}
      \end{enumerate}


\item Representar en una gráfica la función
      \[
      \ f(x)=\left\{
      \begin{array}{cl}
      -2x   & \hbox{si $x\leq0$;} \\
      x^{2} & \hbox{si $x>0$.}    \\
      \end{array}
      \right.
      \]

      \begin{indication}
      Para representar funciones a trozos, Geogebra utiliza el comando
      \begin{center}
            \command{Si(<Condición>, <Entonces>, <Si no>)} 
      \end{center}
      y se pueden anidar varios comandos unos dentro de otros. 
      Utilizando este comando para representar la función anterior, habría que introducir la expresión
      \begin{center}
        Si[x<=0, -2x, x\^2]    
      \end{center}
      \end{indication}
\end{enumerate}


\section{Ejercicios propuestos}
\begin{enumerate}[leftmargin=*]
\item Hallar el dominio de las siguientes funciones a partir de sus representaciones gráficas:

      \begin{enumerate}
      \item $f(x)=\dfrac{x^{2} + x + 1}{x^{3} - x}$
      \item $g(x)=\sqrt[2]{x^{4}-1}$.
      \item $h(x)=\cos{\dfrac{x + 3}{x^{2} + 1}}$.
      \item $l(x)=\arcsen{\dfrac{x}{1+x}}$.
      \end{enumerate}

\item Se considera la función
      \[
      \ f(x)=\frac{x^{3} + x +2}{5x^{3} - 9x^{2} - 4x + 4}.
      \]

      Representarla gráficamente y determinar a partir de dicha representación:

      \begin{enumerate}
      \item Dominio.
      \item Imagen.
      \item Asíntotas.
      \item Raíces.
      \item Signo.
      \item Intervalos de crecimiento y decrecimiento.
      \item Intervalos de concavidad y convexidad.
      \item Extremos relativos.
      \item Puntos de inflexión.
      \end{enumerate}

\item Representar en una misma gráfica las funciones $\log_{10}{x}$, $\log_{2}{x}$, $\log{x}$, $\log_{0.5}{x}$.
      \begin{enumerate}
      \item A la vista de las gráficas obtenidas, indicar cuáles de las funciones anteriores son crecientes y cuáles son decrecientes.
      \item Determinar, a partir de los resultados obtenidos, o representando nuevas funciones si fuera necesario, para qué valores de $a$ será
            creciente la función $\log_{a}{x}$.
      \item Determinar, a partir de los resultados obtenidos, o representando nuevas funciones si fuera necesario, para qué valores de $a$ será
            decreciente la función $\log_{a}{x}$.
      \end{enumerate}

\item Completar las siguientes frases con la palabra igual, o el número de veces que sea mayor o menor en cada caso:
      \begin{enumerate}
      \item La función $\cos{2x}$ tiene un período............ que la función $\cos{x}$.
      \item La función $\cos{2x}$ tiene una amplitud............ que la función $\cos{x}$.
      \item La función $\cos\dfrac{x}{2}$ tiene un período............ que la función $\cos{3x}$.
      \item La función $\cos\dfrac{x}{2}$ tiene una amplitud............ que la función $\cos{3x}$.
      \item La función $3\cos{2x}$ tiene un período............ que la función $\cos\dfrac{x}{2}$.
      \item La función $3\cos{2x}$ tiene una amplitud............ que la función $\cos\dfrac{x}{2}$.
      \end{enumerate}

\item Hallar a partir de la representación gráfica, las soluciones de $e^{-1/x}=\dfrac{1}{x}$.

\item Representar en una gráfica la función
      \[
      \ f(x)=\left\{
      \begin{array}{ll}
      x^{3}   & \hbox{si $x<0$}    \\
      e^{x}-1 & \hbox{si $x\geq0$} \\
      \end{array}
      \right.
      \]

\end{enumerate}


% Como interpreta% !TEX root = ../practicas_geogebra.tex
% Author: Alfredo Sánchez Alberca (asalber@ceu.es)
\chapter{Funciones Elementales}

% \section{Fundamentos teóricos}

% En esta práctica se introducen los conceptos básicos sobre funciones reales de variable real, esto es, funciones
% \[f:\mathbb{R}\rightarrow \mathbb{R}.\]

% \subsection{Dominio e imagen}

% El \emph{Dominio} de la función $f$ es el conjunto de los números reales $x$ para los que existe $f(x)$ y se designa mediante $\dom f$.

% La \emph{Imagen} de $f$ es el conjunto de los números reales $y$ para los que existe algún $x\in \mathbb{R}$ tal que $f(x)=y$, y se denota por $\im f$.


% \subsection{Signo y crecimiento}
% El \emph{signo} de la función es positivo $(+)$ en los valores de $x$ para los que $f(x)>0$ y negativo $(-)$ en los que $f(x)<0$.
% Los valores de $x$ en los que la función se anula se conocen como \emph{raíces} de la función.

% Una función $f(x)$ es \emph{creciente} en un intervalo $I$ si $\forall\, x_1, x_2 \in I$ tales que $x_1<x_2$ se verifica que $f(x_1)\leq f(x_2)$.

% Del mismo modo, se dice que una función $f(x)$ es \emph{decreciente} en un intervalo $I$ si $\forall\, x_1, x_2 \in I$ tales que $x_1<x_2$ se verifica que $f(x_1)\geq f(x_2)$. En la figura~\ref{g:crecimiento} se muestran estos conceptos.

% \begin{figure}[h!]
% 	\centering \subfigure[Función creciente.] {\label{g:funcion_creciente}
% 		\scalebox{1}{\input{img/funciones_elementales/funcion_creciente}}}\qquad
% 	\subfigure[Función decreciente.]{\label{g:funcion_decreciente}
% 		\scalebox{1}{\input{img/funciones_elementales/funcion_decreciente}}}
% 	\caption{Crecimiento de una función.}
% 	\label{g:crecimiento}
% \end{figure}


% \subsection{Extremos Relativos}
% Una función $f(x)$ tiene un \emph{máximo relativo} en $x_0$ si existe un entorno $A$ de $x_0$ tal que $\forall x \in A$
% se verifica que $f(x)\leq f(x_0)$.

% Una función $f(x)$ tiene un \emph{mínimo relativo} en $x_0$ si existe un entorno $A$ de $x_0$ tal que $\forall x\in A$
% se verifica que $f(x)\geq f(x_0)$.

% Diremos que la función $f(x)$ tiene un \emph{extremo relativo} en un punto si tiene un \emph{máximo o mínimo relativo}
% en dicho punto. Estos conceptos se muestran en la figura~\ref{g:extremos}.

% \begin{figure}[h!]
% 	\centering \subfigure[Máximo relativo.] {\label{g:maximo}
% 		\scalebox{1}{\input{img/funciones_elementales/maximo}}}\qquad
% 	\subfigure[Mínimo relativo.]{\label{g:minimo}
% 		\scalebox{1}{\input{img/funciones_elementales/minimo}}}
% 	\caption{Extremos relativos de una función.}
% 	\label{g:extremos}
% \end{figure}

% Una función $f(x)$ está \emph{acotada superiormente} si $\exists K\in\mathbb{R}$ tal que $f(x)\leq K$ $\forall x \in \dom f$. Análogamente, se dice que una función $f(x)$ está \emph{acotada inferiormente} si $\exists K\in\mathbb{R}$ tal que $f(x)\geq K$ $\forall x \in \dom f$.

% Una función $f(x)$ está \emph{acotada} si lo está superior e inferiormente, es decir si $\exists K\in\mathbb{R}$ tal que $|f(x)|\leq K$ $\forall x \in \dom f$.


% \subsection{Concavidad}

% De forma intuitiva se puede decir que una función $f(x)$ es \emph{cóncava} en un intervalo $I$ si $\forall\, x_1, x_2
% \in I$, el segmento de extremos $(x_1,f(x_1))$ y $(x_2,f(x_2))$ queda por encima de la gráfica de $f$.

% Análogamente se dirá que es \emph{convexa} si el segmento anterior queda por debajo de la gráfica de $f$.

% Diremos que la función $f(x)$ tiene un \emph{punto de inflexión} en $x_0$ si en ese punto la función pasa de cóncava a
% convexa o de convexa a cóncava. Estos conceptos se ilustran en la figura~\ref{g:concavidad}.

% \begin{figure}[h!]
% 	\centering \subfigure[Función cóncava.] {\label{g:funcion_convexa}
% 		\scalebox{1}{\input{img/funciones_elementales/funcion_convexa}}}\qquad
% 	\subfigure[Función convexa.]{\label{g:funcion_concava}
% 		\scalebox{1}{\input{img/funciones_elementales/funcion_concava}}}
% 	\caption{Concavidad de una función.}
% 	\label{g:concavidad}
% \end{figure}

% \subsection{Asíntotas}

% La recta $x=a$ es una \emph{asíntota vertical} de la función $f(x)$ si al menos uno de los límites laterales de $f(x)$ cuando $x$ tiende hacia $a$ es $+\infty$ o $-\infty$, es decir cuando se verifique alguna de las siguientes igualdades
% \[
% 	\ \lim_{x\rightarrow a^{+}}f(x)=\pm\infty   \quad \textrm{o} \quad
% 	\lim_{x\rightarrow a^{-}}f(x)=\pm\infty
% \]

% La recta $y=b$ es una \emph{asíntota horizontal} de la función $f(x)$ si alguno de los límites de $f(x)$ cuando $x$ tiende hacia $+\infty$ o $-\infty$ es igual a $b$, es decir cuando se verifique
% \[
% 	\ \lim_{x\rightarrow -\infty }f(x)=b    \quad \textrm{o} \quad
% 	\ \lim_{x\rightarrow +\infty }f(x)=b
% \]

% La recta $y=mx+n$ es una \emph{asíntota oblicua} de la función $f(x)$ si alguno de los límites de $f(x)-(mx+n)$ cuando $x$ tiende hacia $+\infty$ o $-\infty$ es igual a 0, es decir si

% \[
% 	\ \lim_{x\rightarrow -\infty }{(f(x)-mx)}=n    \quad \textrm{o} \quad
% 	\ \lim_{x\rightarrow +\infty }{(f(x)-mx)}=n
% \]

% En la figura~\ref{g:asintotas} se muestran los distintos tipos de asíntotas.

% \begin{figure}[h!]
% 	\centering \subfigure[Asíntota horizontal y vertical.] {\label{g:asintotahorizontalyvertical}
% 		\scalebox{1}{\input{img/funciones_elementales/asintota_vertical}}}\qquad\qquad
% 	\subfigure[Asíntota vertical y oblicua.]{\label{g:asintotaoblicua}
% 		\scalebox{1}{\input{img/funciones_elementales/asintota_oblicua}}}
% 	\caption{Tipos de asíntotas de una función.}
% 	\label{g:asintotas}
% \end{figure}


% \subsection{Periodicidad}
% Una función $f(x)$ es \emph{periódica} si existe $h\in\mathbb{R^{+}}$ tal que \[f(x+h)=f(x)\  \forall x\in \dom f\] siendo el período $T$ de la función, el menor valor $h$ que verifique la igualdad anterior.

% En una función periódica, por ejemplo $f(x)=A\sen(wt)$, se denomina \emph{amplitud} al valor de $A$, y es la mitad de la diferencia entre los valores máximos y mínimos de la función. En la figura~\ref{g:periodoyamplitud} se ilustran estos conceptos.

% \begin{figure}[h!]
% 	\centering
% 	\scalebox{0.8}{\input{img/funciones_elementales/funcion_periodica}}
% 	\caption{Periodo y amplitud de una función periódica.}
% 	\label{g:periodoyamplitud}
% \end{figure}

% \clearpage
% \newpage

\section{Ejercicios resueltos}

\begin{enumerate}[leftmargin=*]
\item Se considera la función
      \[
      f(t)=\frac{t^{4} +19\cdot t^{2} - 5}{t^{4} +9\cdot t^{2} - 10}.
      \]

      Representarla gráficamente y determinar a partir de dicha representación:

      \begin{enumerate}
      \item  Dominio.
            \begin{indication}
            \begin{enumerate}
            \item Para representarla gráficamente, introducir la función en la barra de \field{Entrada} de la \field{Vista CAS} y activar la \field{Vista Gráfica}.
            \item Para determinar el dominio tan sólo hay que determinar los valores de $x$ en los que existe la función.
            \item Recordar que, tanto para éste como para el resto de los apartados del ejercicio, pretendemos llegar a conclusiones aproximadas que tan sólo sacamos del análisis de la gráfica.
            \end{enumerate}
            \end{indication}

      \item  Imagen.
            \begin{indication}
            Fijarse en los valores de la variable $y$ hasta los que llega la función.
            \end{indication}

      \item  Asíntotas.
            \begin{indication}
            Son las líneas rectas, ya sea horizontales, verticales u oblicuas, hacia las que tiende la función.
            \end{indication}

      \item  Raíces.
            \begin{indication}
            Son los valores de la variable $x$, si los hay, en los que la función vale 0.
            \end{indication}

      \item Signo.
            \begin{indication}
            Hay que determinar, aproximadamente, por un lado los intervalos de variable $x$ en los que la función es positiva, y por el otro aquellos en
            los que es negativa.
            \end{indication}

      \item  Intervalos de crecimiento y decrecimiento.
            \begin{indication}
            De nuevo, por un lado hay que determinar los intervalos de variable $x$ en los que a medida que crece $x$ también lo hace $y$, que serían
            los intervalos de crecimiento, y también aquellos otros en los que a medida que crece $x$ decrece $y$, que serían los intervalos de
            decremimiento.
            \end{indication}

      \item Intervalos de concavidad y convexidad.
            \begin{indication}
            Para los intervalos de concavidad y convexidad, nos fijamos en el segmento de línea recta que une dos puntos cualquiera del intervalo. Si
            dicho segmento queda por encima de la gráfica, entonces la función es cóncava en el intervalo, mientras que si queda por debajo, entonces es
            convexa en el mismo.
            \end{indication}

      \item Extremos relativos.
            \begin{indication}
            Determinamos, aproximadamente, los puntos en los que se encuentran los máximos y mínimos relativos de la función.
            \end{indication}

      \item Puntos de inflexión.
            \begin{indication}
            Determinamos, aproximadamente, los puntos en los que la función cambia de curvatura, de cóncava a convexa o a la inversa.
            \end{indication}
      \end{enumerate}

\item Representar en una misma gráfica las funciones $2^{x}, e^{x}, 0.7^{x}, 0.5^{x}$. A la vista de las gráficas obtenidas, indicar cuáles
      de las funciones anteriores son crecientes y cuáles son decrecientes.
      \begin{indication}
      Introducir cada función en la barra de \field{Entrada} de la \field{Vista CAS}.
      \end{indication}

      ¿En general, para qué valores de $a$ será la función creciente? ¿Y para qué valores de $a$ será decreciente? Probar con
      distintos valores de $a$ representando gráficamente nuevas funciones si fuera necesario.


\item Representar en una misma gráfica las funciones siguientes, indicando su período y amplitud.
      \begin{enumerate}
      \item $\sen{x}$, $\sen{x}+2$, $\sen{(x+2)}$.
      \item $\sen{2x}$, $2\sen{x}$, $\sen\frac{x}{2}$.
            \begin{indication}
            Introducir cada función en la barra de \field{Entrada} de la \field{Vista CAS}.
            \end{indication}
      \end{enumerate}


\item Representar en una gráfica la función
      \[
      \ f(x)=\left\{
      \begin{array}{cl}
      -2x   & \hbox{si $x\leq0$;} \\
      x^{2} & \hbox{si $x>0$.}    \\
      \end{array}
      \right.
      \]

      \begin{indication}
      Para representar funciones a trozos, Geogebra utiliza el comando
      \begin{center}
            \command{Si(<Condición>, <Entonces>, <Si no>)} 
      \end{center}
      y se pueden anidar varios comandos unos dentro de otros. 
      Utilizando este comando para representar la función anterior, habría que introducir la expresión
      \begin{center}
        Si[x<=0, -2x, x\^2]    
      \end{center}
      \end{indication}
\end{enumerate}


\section{Ejercicios propuestos}
\begin{enumerate}[leftmargin=*]
\item Hallar el dominio de las siguientes funciones a partir de sus representaciones gráficas:

      \begin{enumerate}
      \item $f(x)=\dfrac{x^{2} + x + 1}{x^{3} - x}$
      \item $g(x)=\sqrt[2]{x^{4}-1}$.
      \item $h(x)=\cos{\dfrac{x + 3}{x^{2} + 1}}$.
      \item $l(x)=\arcsen{\dfrac{x}{1+x}}$.
      \end{enumerate}

\item Se considera la función
      \[
      \ f(x)=\frac{x^{3} + x +2}{5x^{3} - 9x^{2} - 4x + 4}.
      \]

      Representarla gráficamente y determinar a partir de dicha representación:

      \begin{enumerate}
      \item Dominio.
      \item Imagen.
      \item Asíntotas.
      \item Raíces.
      \item Signo.
      \item Intervalos de crecimiento y decrecimiento.
      \item Intervalos de concavidad y convexidad.
      \item Extremos relativos.
      \item Puntos de inflexión.
      \end{enumerate}

\item Representar en una misma gráfica las funciones $\log_{10}{x}$, $\log_{2}{x}$, $\log{x}$, $\log_{0.5}{x}$.
      \begin{enumerate}
      \item A la vista de las gráficas obtenidas, indicar cuáles de las funciones anteriores son crecientes y cuáles son decrecientes.
      \item Determinar, a partir de los resultados obtenidos, o representando nuevas funciones si fuera necesario, para qué valores de $a$ será
            creciente la función $\log_{a}{x}$.
      \item Determinar, a partir de los resultados obtenidos, o representando nuevas funciones si fuera necesario, para qué valores de $a$ será
            decreciente la función $\log_{a}{x}$.
      \end{enumerate}

\item Completar las siguientes frases con la palabra igual, o el número de veces que sea mayor o menor en cada caso:
      \begin{enumerate}
      \item La función $\cos{2x}$ tiene un período............ que la función $\cos{x}$.
      \item La función $\cos{2x}$ tiene una amplitud............ que la función $\cos{x}$.
      \item La función $\cos\dfrac{x}{2}$ tiene un período............ que la función $\cos{3x}$.
      \item La función $\cos\dfrac{x}{2}$ tiene una amplitud............ que la función $\cos{3x}$.
      \item La función $3\cos{2x}$ tiene un período............ que la función $\cos\dfrac{x}{2}$.
      \item La función $3\cos{2x}$ tiene una amplitud............ que la función $\cos\dfrac{x}{2}$.
      \end{enumerate}

\item Hallar a partir de la representación gráfica, las soluciones de $e^{-1/x}=\dfrac{1}{x}$.

\item Representar en una gráfica la función
      \[
      \ f(x)=\left\{
      \begin{array}{ll}
      x^{3}   & \hbox{si $x<0$}    \\
      e^{x}-1 & \hbox{si $x\geq0$} \\
      \end{array}
      \right.
      \]

\end{enumerate}

% rica de los límites, definiremos rectas
% particulares a % !TEX root = ../practicas_geogebra.tex
% Author: Alfredo Sánchez Alberca (asalber@ceu.es)
\chapter{Funciones Elementales}

% \section{Fundamentos teóricos}

% En esta práctica se introducen los conceptos básicos sobre funciones reales de variable real, esto es, funciones
% \[f:\mathbb{R}\rightarrow \mathbb{R}.\]

% \subsection{Dominio e imagen}

% El \emph{Dominio} de la función $f$ es el conjunto de los números reales $x$ para los que existe $f(x)$ y se designa mediante $\dom f$.

% La \emph{Imagen} de $f$ es el conjunto de los números reales $y$ para los que existe algún $x\in \mathbb{R}$ tal que $f(x)=y$, y se denota por $\im f$.


% \subsection{Signo y crecimiento}
% El \emph{signo} de la función es positivo $(+)$ en los valores de $x$ para los que $f(x)>0$ y negativo $(-)$ en los que $f(x)<0$.
% Los valores de $x$ en los que la función se anula se conocen como \emph{raíces} de la función.

% Una función $f(x)$ es \emph{creciente} en un intervalo $I$ si $\forall\, x_1, x_2 \in I$ tales que $x_1<x_2$ se verifica que $f(x_1)\leq f(x_2)$.

% Del mismo modo, se dice que una función $f(x)$ es \emph{decreciente} en un intervalo $I$ si $\forall\, x_1, x_2 \in I$ tales que $x_1<x_2$ se verifica que $f(x_1)\geq f(x_2)$. En la figura~\ref{g:crecimiento} se muestran estos conceptos.

% \begin{figure}[h!]
% 	\centering \subfigure[Función creciente.] {\label{g:funcion_creciente}
% 		\scalebox{1}{\input{img/funciones_elementales/funcion_creciente}}}\qquad
% 	\subfigure[Función decreciente.]{\label{g:funcion_decreciente}
% 		\scalebox{1}{\input{img/funciones_elementales/funcion_decreciente}}}
% 	\caption{Crecimiento de una función.}
% 	\label{g:crecimiento}
% \end{figure}


% \subsection{Extremos Relativos}
% Una función $f(x)$ tiene un \emph{máximo relativo} en $x_0$ si existe un entorno $A$ de $x_0$ tal que $\forall x \in A$
% se verifica que $f(x)\leq f(x_0)$.

% Una función $f(x)$ tiene un \emph{mínimo relativo} en $x_0$ si existe un entorno $A$ de $x_0$ tal que $\forall x\in A$
% se verifica que $f(x)\geq f(x_0)$.

% Diremos que la función $f(x)$ tiene un \emph{extremo relativo} en un punto si tiene un \emph{máximo o mínimo relativo}
% en dicho punto. Estos conceptos se muestran en la figura~\ref{g:extremos}.

% \begin{figure}[h!]
% 	\centering \subfigure[Máximo relativo.] {\label{g:maximo}
% 		\scalebox{1}{\input{img/funciones_elementales/maximo}}}\qquad
% 	\subfigure[Mínimo relativo.]{\label{g:minimo}
% 		\scalebox{1}{\input{img/funciones_elementales/minimo}}}
% 	\caption{Extremos relativos de una función.}
% 	\label{g:extremos}
% \end{figure}

% Una función $f(x)$ está \emph{acotada superiormente} si $\exists K\in\mathbb{R}$ tal que $f(x)\leq K$ $\forall x \in \dom f$. Análogamente, se dice que una función $f(x)$ está \emph{acotada inferiormente} si $\exists K\in\mathbb{R}$ tal que $f(x)\geq K$ $\forall x \in \dom f$.

% Una función $f(x)$ está \emph{acotada} si lo está superior e inferiormente, es decir si $\exists K\in\mathbb{R}$ tal que $|f(x)|\leq K$ $\forall x \in \dom f$.


% \subsection{Concavidad}

% De forma intuitiva se puede decir que una función $f(x)$ es \emph{cóncava} en un intervalo $I$ si $\forall\, x_1, x_2
% \in I$, el segmento de extremos $(x_1,f(x_1))$ y $(x_2,f(x_2))$ queda por encima de la gráfica de $f$.

% Análogamente se dirá que es \emph{convexa} si el segmento anterior queda por debajo de la gráfica de $f$.

% Diremos que la función $f(x)$ tiene un \emph{punto de inflexión} en $x_0$ si en ese punto la función pasa de cóncava a
% convexa o de convexa a cóncava. Estos conceptos se ilustran en la figura~\ref{g:concavidad}.

% \begin{figure}[h!]
% 	\centering \subfigure[Función cóncava.] {\label{g:funcion_convexa}
% 		\scalebox{1}{\input{img/funciones_elementales/funcion_convexa}}}\qquad
% 	\subfigure[Función convexa.]{\label{g:funcion_concava}
% 		\scalebox{1}{\input{img/funciones_elementales/funcion_concava}}}
% 	\caption{Concavidad de una función.}
% 	\label{g:concavidad}
% \end{figure}

% \subsection{Asíntotas}

% La recta $x=a$ es una \emph{asíntota vertical} de la función $f(x)$ si al menos uno de los límites laterales de $f(x)$ cuando $x$ tiende hacia $a$ es $+\infty$ o $-\infty$, es decir cuando se verifique alguna de las siguientes igualdades
% \[
% 	\ \lim_{x\rightarrow a^{+}}f(x)=\pm\infty   \quad \textrm{o} \quad
% 	\lim_{x\rightarrow a^{-}}f(x)=\pm\infty
% \]

% La recta $y=b$ es una \emph{asíntota horizontal} de la función $f(x)$ si alguno de los límites de $f(x)$ cuando $x$ tiende hacia $+\infty$ o $-\infty$ es igual a $b$, es decir cuando se verifique
% \[
% 	\ \lim_{x\rightarrow -\infty }f(x)=b    \quad \textrm{o} \quad
% 	\ \lim_{x\rightarrow +\infty }f(x)=b
% \]

% La recta $y=mx+n$ es una \emph{asíntota oblicua} de la función $f(x)$ si alguno de los límites de $f(x)-(mx+n)$ cuando $x$ tiende hacia $+\infty$ o $-\infty$ es igual a 0, es decir si

% \[
% 	\ \lim_{x\rightarrow -\infty }{(f(x)-mx)}=n    \quad \textrm{o} \quad
% 	\ \lim_{x\rightarrow +\infty }{(f(x)-mx)}=n
% \]

% En la figura~\ref{g:asintotas} se muestran los distintos tipos de asíntotas.

% \begin{figure}[h!]
% 	\centering \subfigure[Asíntota horizontal y vertical.] {\label{g:asintotahorizontalyvertical}
% 		\scalebox{1}{\input{img/funciones_elementales/asintota_vertical}}}\qquad\qquad
% 	\subfigure[Asíntota vertical y oblicua.]{\label{g:asintotaoblicua}
% 		\scalebox{1}{\input{img/funciones_elementales/asintota_oblicua}}}
% 	\caption{Tipos de asíntotas de una función.}
% 	\label{g:asintotas}
% \end{figure}


% \subsection{Periodicidad}
% Una función $f(x)$ es \emph{periódica} si existe $h\in\mathbb{R^{+}}$ tal que \[f(x+h)=f(x)\  \forall x\in \dom f\] siendo el período $T$ de la función, el menor valor $h$ que verifique la igualdad anterior.

% En una función periódica, por ejemplo $f(x)=A\sen(wt)$, se denomina \emph{amplitud} al valor de $A$, y es la mitad de la diferencia entre los valores máximos y mínimos de la función. En la figura~\ref{g:periodoyamplitud} se ilustran estos conceptos.

% \begin{figure}[h!]
% 	\centering
% 	\scalebox{0.8}{\input{img/funciones_elementales/funcion_periodica}}
% 	\caption{Periodo y amplitud de una función periódica.}
% 	\label{g:periodoyamplitud}
% \end{figure}

% \clearpage
% \newpage

\section{Ejercicios resueltos}

\begin{enumerate}[leftmargin=*]
\item Se considera la función
      \[
      f(t)=\frac{t^{4} +19\cdot t^{2} - 5}{t^{4} +9\cdot t^{2} - 10}.
      \]

      Representarla gráficamente y determinar a partir de dicha representación:

      \begin{enumerate}
      \item  Dominio.
            \begin{indication}
            \begin{enumerate}
            \item Para representarla gráficamente, introducir la función en la barra de \field{Entrada} de la \field{Vista CAS} y activar la \field{Vista Gráfica}.
            \item Para determinar el dominio tan sólo hay que determinar los valores de $x$ en los que existe la función.
            \item Recordar que, tanto para éste como para el resto de los apartados del ejercicio, pretendemos llegar a conclusiones aproximadas que tan sólo sacamos del análisis de la gráfica.
            \end{enumerate}
            \end{indication}

      \item  Imagen.
            \begin{indication}
            Fijarse en los valores de la variable $y$ hasta los que llega la función.
            \end{indication}

      \item  Asíntotas.
            \begin{indication}
            Son las líneas rectas, ya sea horizontales, verticales u oblicuas, hacia las que tiende la función.
            \end{indication}

      \item  Raíces.
            \begin{indication}
            Son los valores de la variable $x$, si los hay, en los que la función vale 0.
            \end{indication}

      \item Signo.
            \begin{indication}
            Hay que determinar, aproximadamente, por un lado los intervalos de variable $x$ en los que la función es positiva, y por el otro aquellos en
            los que es negativa.
            \end{indication}

      \item  Intervalos de crecimiento y decrecimiento.
            \begin{indication}
            De nuevo, por un lado hay que determinar los intervalos de variable $x$ en los que a medida que crece $x$ también lo hace $y$, que serían
            los intervalos de crecimiento, y también aquellos otros en los que a medida que crece $x$ decrece $y$, que serían los intervalos de
            decremimiento.
            \end{indication}

      \item Intervalos de concavidad y convexidad.
            \begin{indication}
            Para los intervalos de concavidad y convexidad, nos fijamos en el segmento de línea recta que une dos puntos cualquiera del intervalo. Si
            dicho segmento queda por encima de la gráfica, entonces la función es cóncava en el intervalo, mientras que si queda por debajo, entonces es
            convexa en el mismo.
            \end{indication}

      \item Extremos relativos.
            \begin{indication}
            Determinamos, aproximadamente, los puntos en los que se encuentran los máximos y mínimos relativos de la función.
            \end{indication}

      \item Puntos de inflexión.
            \begin{indication}
            Determinamos, aproximadamente, los puntos en los que la función cambia de curvatura, de cóncava a convexa o a la inversa.
            \end{indication}
      \end{enumerate}

\item Representar en una misma gráfica las funciones $2^{x}, e^{x}, 0.7^{x}, 0.5^{x}$. A la vista de las gráficas obtenidas, indicar cuáles
      de las funciones anteriores son crecientes y cuáles son decrecientes.
      \begin{indication}
      Introducir cada función en la barra de \field{Entrada} de la \field{Vista CAS}.
      \end{indication}

      ¿En general, para qué valores de $a$ será la función creciente? ¿Y para qué valores de $a$ será decreciente? Probar con
      distintos valores de $a$ representando gráficamente nuevas funciones si fuera necesario.


\item Representar en una misma gráfica las funciones siguientes, indicando su período y amplitud.
      \begin{enumerate}
      \item $\sen{x}$, $\sen{x}+2$, $\sen{(x+2)}$.
      \item $\sen{2x}$, $2\sen{x}$, $\sen\frac{x}{2}$.
            \begin{indication}
            Introducir cada función en la barra de \field{Entrada} de la \field{Vista CAS}.
            \end{indication}
      \end{enumerate}


\item Representar en una gráfica la función
      \[
      \ f(x)=\left\{
      \begin{array}{cl}
      -2x   & \hbox{si $x\leq0$;} \\
      x^{2} & \hbox{si $x>0$.}    \\
      \end{array}
      \right.
      \]

      \begin{indication}
      Para representar funciones a trozos, Geogebra utiliza el comando
      \begin{center}
            \command{Si(<Condición>, <Entonces>, <Si no>)} 
      \end{center}
      y se pueden anidar varios comandos unos dentro de otros. 
      Utilizando este comando para representar la función anterior, habría que introducir la expresión
      \begin{center}
        Si[x<=0, -2x, x\^2]    
      \end{center}
      \end{indication}
\end{enumerate}


\section{Ejercicios propuestos}
\begin{enumerate}[leftmargin=*]
\item Hallar el dominio de las siguientes funciones a partir de sus representaciones gráficas:

      \begin{enumerate}
      \item $f(x)=\dfrac{x^{2} + x + 1}{x^{3} - x}$
      \item $g(x)=\sqrt[2]{x^{4}-1}$.
      \item $h(x)=\cos{\dfrac{x + 3}{x^{2} + 1}}$.
      \item $l(x)=\arcsen{\dfrac{x}{1+x}}$.
      \end{enumerate}

\item Se considera la función
      \[
      \ f(x)=\frac{x^{3} + x +2}{5x^{3} - 9x^{2} - 4x + 4}.
      \]

      Representarla gráficamente y determinar a partir de dicha representación:

      \begin{enumerate}
      \item Dominio.
      \item Imagen.
      \item Asíntotas.
      \item Raíces.
      \item Signo.
      \item Intervalos de crecimiento y decrecimiento.
      \item Intervalos de concavidad y convexidad.
      \item Extremos relativos.
      \item Puntos de inflexión.
      \end{enumerate}

\item Representar en una misma gráfica las funciones $\log_{10}{x}$, $\log_{2}{x}$, $\log{x}$, $\log_{0.5}{x}$.
      \begin{enumerate}
      \item A la vista de las gráficas obtenidas, indicar cuáles de las funciones anteriores son crecientes y cuáles son decrecientes.
      \item Determinar, a partir de los resultados obtenidos, o representando nuevas funciones si fuera necesario, para qué valores de $a$ será
            creciente la función $\log_{a}{x}$.
      \item Determinar, a partir de los resultados obtenidos, o representando nuevas funciones si fuera necesario, para qué valores de $a$ será
            decreciente la función $\log_{a}{x}$.
      \end{enumerate}

\item Completar las siguientes frases con la palabra igual, o el número de veces que sea mayor o menor en cada caso:
      \begin{enumerate}
      \item La función $\cos{2x}$ tiene un período............ que la función $\cos{x}$.
      \item La función $\cos{2x}$ tiene una amplitud............ que la función $\cos{x}$.
      \item La función $\cos\dfrac{x}{2}$ tiene un período............ que la función $\cos{3x}$.
      \item La función $\cos\dfrac{x}{2}$ tiene una amplitud............ que la función $\cos{3x}$.
      \item La función $3\cos{2x}$ tiene un período............ que la función $\cos\dfrac{x}{2}$.
      \item La función $3\cos{2x}$ tiene una amplitud............ que la función $\cos\dfrac{x}{2}$.
      \end{enumerate}

\item Hallar a partir de la representación gráfica, las soluciones de $e^{-1/x}=\dfrac{1}{x}$.

\item Representar en una gráfica la función
      \[
      \ f(x)=\left\{
      \begin{array}{ll}
      x^{3}   & \hbox{si $x<0$}    \\
      e^{x}-1 & \hbox{si $x\geq0$} \\
      \end{array}
      \right.
      \]

\end{enumerate}

% nde (se ``pega") la gráfica de una función
% cuando la varia% !TEX root = ../practicas_geogebra.tex
% Author: Alfredo Sánchez Alberca (asalber@ceu.es)
\chapter{Funciones Elementales}

% \section{Fundamentos teóricos}

% En esta práctica se introducen los conceptos básicos sobre funciones reales de variable real, esto es, funciones
% \[f:\mathbb{R}\rightarrow \mathbb{R}.\]

% \subsection{Dominio e imagen}

% El \emph{Dominio} de la función $f$ es el conjunto de los números reales $x$ para los que existe $f(x)$ y se designa mediante $\dom f$.

% La \emph{Imagen} de $f$ es el conjunto de los números reales $y$ para los que existe algún $x\in \mathbb{R}$ tal que $f(x)=y$, y se denota por $\im f$.


% \subsection{Signo y crecimiento}
% El \emph{signo} de la función es positivo $(+)$ en los valores de $x$ para los que $f(x)>0$ y negativo $(-)$ en los que $f(x)<0$.
% Los valores de $x$ en los que la función se anula se conocen como \emph{raíces} de la función.

% Una función $f(x)$ es \emph{creciente} en un intervalo $I$ si $\forall\, x_1, x_2 \in I$ tales que $x_1<x_2$ se verifica que $f(x_1)\leq f(x_2)$.

% Del mismo modo, se dice que una función $f(x)$ es \emph{decreciente} en un intervalo $I$ si $\forall\, x_1, x_2 \in I$ tales que $x_1<x_2$ se verifica que $f(x_1)\geq f(x_2)$. En la figura~\ref{g:crecimiento} se muestran estos conceptos.

% \begin{figure}[h!]
% 	\centering \subfigure[Función creciente.] {\label{g:funcion_creciente}
% 		\scalebox{1}{\input{img/funciones_elementales/funcion_creciente}}}\qquad
% 	\subfigure[Función decreciente.]{\label{g:funcion_decreciente}
% 		\scalebox{1}{\input{img/funciones_elementales/funcion_decreciente}}}
% 	\caption{Crecimiento de una función.}
% 	\label{g:crecimiento}
% \end{figure}


% \subsection{Extremos Relativos}
% Una función $f(x)$ tiene un \emph{máximo relativo} en $x_0$ si existe un entorno $A$ de $x_0$ tal que $\forall x \in A$
% se verifica que $f(x)\leq f(x_0)$.

% Una función $f(x)$ tiene un \emph{mínimo relativo} en $x_0$ si existe un entorno $A$ de $x_0$ tal que $\forall x\in A$
% se verifica que $f(x)\geq f(x_0)$.

% Diremos que la función $f(x)$ tiene un \emph{extremo relativo} en un punto si tiene un \emph{máximo o mínimo relativo}
% en dicho punto. Estos conceptos se muestran en la figura~\ref{g:extremos}.

% \begin{figure}[h!]
% 	\centering \subfigure[Máximo relativo.] {\label{g:maximo}
% 		\scalebox{1}{\input{img/funciones_elementales/maximo}}}\qquad
% 	\subfigure[Mínimo relativo.]{\label{g:minimo}
% 		\scalebox{1}{\input{img/funciones_elementales/minimo}}}
% 	\caption{Extremos relativos de una función.}
% 	\label{g:extremos}
% \end{figure}

% Una función $f(x)$ está \emph{acotada superiormente} si $\exists K\in\mathbb{R}$ tal que $f(x)\leq K$ $\forall x \in \dom f$. Análogamente, se dice que una función $f(x)$ está \emph{acotada inferiormente} si $\exists K\in\mathbb{R}$ tal que $f(x)\geq K$ $\forall x \in \dom f$.

% Una función $f(x)$ está \emph{acotada} si lo está superior e inferiormente, es decir si $\exists K\in\mathbb{R}$ tal que $|f(x)|\leq K$ $\forall x \in \dom f$.


% \subsection{Concavidad}

% De forma intuitiva se puede decir que una función $f(x)$ es \emph{cóncava} en un intervalo $I$ si $\forall\, x_1, x_2
% \in I$, el segmento de extremos $(x_1,f(x_1))$ y $(x_2,f(x_2))$ queda por encima de la gráfica de $f$.

% Análogamente se dirá que es \emph{convexa} si el segmento anterior queda por debajo de la gráfica de $f$.

% Diremos que la función $f(x)$ tiene un \emph{punto de inflexión} en $x_0$ si en ese punto la función pasa de cóncava a
% convexa o de convexa a cóncava. Estos conceptos se ilustran en la figura~\ref{g:concavidad}.

% \begin{figure}[h!]
% 	\centering \subfigure[Función cóncava.] {\label{g:funcion_convexa}
% 		\scalebox{1}{\input{img/funciones_elementales/funcion_convexa}}}\qquad
% 	\subfigure[Función convexa.]{\label{g:funcion_concava}
% 		\scalebox{1}{\input{img/funciones_elementales/funcion_concava}}}
% 	\caption{Concavidad de una función.}
% 	\label{g:concavidad}
% \end{figure}

% \subsection{Asíntotas}

% La recta $x=a$ es una \emph{asíntota vertical} de la función $f(x)$ si al menos uno de los límites laterales de $f(x)$ cuando $x$ tiende hacia $a$ es $+\infty$ o $-\infty$, es decir cuando se verifique alguna de las siguientes igualdades
% \[
% 	\ \lim_{x\rightarrow a^{+}}f(x)=\pm\infty   \quad \textrm{o} \quad
% 	\lim_{x\rightarrow a^{-}}f(x)=\pm\infty
% \]

% La recta $y=b$ es una \emph{asíntota horizontal} de la función $f(x)$ si alguno de los límites de $f(x)$ cuando $x$ tiende hacia $+\infty$ o $-\infty$ es igual a $b$, es decir cuando se verifique
% \[
% 	\ \lim_{x\rightarrow -\infty }f(x)=b    \quad \textrm{o} \quad
% 	\ \lim_{x\rightarrow +\infty }f(x)=b
% \]

% La recta $y=mx+n$ es una \emph{asíntota oblicua} de la función $f(x)$ si alguno de los límites de $f(x)-(mx+n)$ cuando $x$ tiende hacia $+\infty$ o $-\infty$ es igual a 0, es decir si

% \[
% 	\ \lim_{x\rightarrow -\infty }{(f(x)-mx)}=n    \quad \textrm{o} \quad
% 	\ \lim_{x\rightarrow +\infty }{(f(x)-mx)}=n
% \]

% En la figura~\ref{g:asintotas} se muestran los distintos tipos de asíntotas.

% \begin{figure}[h!]
% 	\centering \subfigure[Asíntota horizontal y vertical.] {\label{g:asintotahorizontalyvertical}
% 		\scalebox{1}{\input{img/funciones_elementales/asintota_vertical}}}\qquad\qquad
% 	\subfigure[Asíntota vertical y oblicua.]{\label{g:asintotaoblicua}
% 		\scalebox{1}{\input{img/funciones_elementales/asintota_oblicua}}}
% 	\caption{Tipos de asíntotas de una función.}
% 	\label{g:asintotas}
% \end{figure}


% \subsection{Periodicidad}
% Una función $f(x)$ es \emph{periódica} si existe $h\in\mathbb{R^{+}}$ tal que \[f(x+h)=f(x)\  \forall x\in \dom f\] siendo el período $T$ de la función, el menor valor $h$ que verifique la igualdad anterior.

% En una función periódica, por ejemplo $f(x)=A\sen(wt)$, se denomina \emph{amplitud} al valor de $A$, y es la mitad de la diferencia entre los valores máximos y mínimos de la función. En la figura~\ref{g:periodoyamplitud} se ilustran estos conceptos.

% \begin{figure}[h!]
% 	\centering
% 	\scalebox{0.8}{\input{img/funciones_elementales/funcion_periodica}}
% 	\caption{Periodo y amplitud de una función periódica.}
% 	\label{g:periodoyamplitud}
% \end{figure}

% \clearpage
% \newpage

\section{Ejercicios resueltos}

\begin{enumerate}[leftmargin=*]
\item Se considera la función
      \[
      f(t)=\frac{t^{4} +19\cdot t^{2} - 5}{t^{4} +9\cdot t^{2} - 10}.
      \]

      Representarla gráficamente y determinar a partir de dicha representación:

      \begin{enumerate}
      \item  Dominio.
            \begin{indication}
            \begin{enumerate}
            \item Para representarla gráficamente, introducir la función en la barra de \field{Entrada} de la \field{Vista CAS} y activar la \field{Vista Gráfica}.
            \item Para determinar el dominio tan sólo hay que determinar los valores de $x$ en los que existe la función.
            \item Recordar que, tanto para éste como para el resto de los apartados del ejercicio, pretendemos llegar a conclusiones aproximadas que tan sólo sacamos del análisis de la gráfica.
            \end{enumerate}
            \end{indication}

      \item  Imagen.
            \begin{indication}
            Fijarse en los valores de la variable $y$ hasta los que llega la función.
            \end{indication}

      \item  Asíntotas.
            \begin{indication}
            Son las líneas rectas, ya sea horizontales, verticales u oblicuas, hacia las que tiende la función.
            \end{indication}

      \item  Raíces.
            \begin{indication}
            Son los valores de la variable $x$, si los hay, en los que la función vale 0.
            \end{indication}

      \item Signo.
            \begin{indication}
            Hay que determinar, aproximadamente, por un lado los intervalos de variable $x$ en los que la función es positiva, y por el otro aquellos en
            los que es negativa.
            \end{indication}

      \item  Intervalos de crecimiento y decrecimiento.
            \begin{indication}
            De nuevo, por un lado hay que determinar los intervalos de variable $x$ en los que a medida que crece $x$ también lo hace $y$, que serían
            los intervalos de crecimiento, y también aquellos otros en los que a medida que crece $x$ decrece $y$, que serían los intervalos de
            decremimiento.
            \end{indication}

      \item Intervalos de concavidad y convexidad.
            \begin{indication}
            Para los intervalos de concavidad y convexidad, nos fijamos en el segmento de línea recta que une dos puntos cualquiera del intervalo. Si
            dicho segmento queda por encima de la gráfica, entonces la función es cóncava en el intervalo, mientras que si queda por debajo, entonces es
            convexa en el mismo.
            \end{indication}

      \item Extremos relativos.
            \begin{indication}
            Determinamos, aproximadamente, los puntos en los que se encuentran los máximos y mínimos relativos de la función.
            \end{indication}

      \item Puntos de inflexión.
            \begin{indication}
            Determinamos, aproximadamente, los puntos en los que la función cambia de curvatura, de cóncava a convexa o a la inversa.
            \end{indication}
      \end{enumerate}

\item Representar en una misma gráfica las funciones $2^{x}, e^{x}, 0.7^{x}, 0.5^{x}$. A la vista de las gráficas obtenidas, indicar cuáles
      de las funciones anteriores son crecientes y cuáles son decrecientes.
      \begin{indication}
      Introducir cada función en la barra de \field{Entrada} de la \field{Vista CAS}.
      \end{indication}

      ¿En general, para qué valores de $a$ será la función creciente? ¿Y para qué valores de $a$ será decreciente? Probar con
      distintos valores de $a$ representando gráficamente nuevas funciones si fuera necesario.


\item Representar en una misma gráfica las funciones siguientes, indicando su período y amplitud.
      \begin{enumerate}
      \item $\sen{x}$, $\sen{x}+2$, $\sen{(x+2)}$.
      \item $\sen{2x}$, $2\sen{x}$, $\sen\frac{x}{2}$.
            \begin{indication}
            Introducir cada función en la barra de \field{Entrada} de la \field{Vista CAS}.
            \end{indication}
      \end{enumerate}


\item Representar en una gráfica la función
      \[
      \ f(x)=\left\{
      \begin{array}{cl}
      -2x   & \hbox{si $x\leq0$;} \\
      x^{2} & \hbox{si $x>0$.}    \\
      \end{array}
      \right.
      \]

      \begin{indication}
      Para representar funciones a trozos, Geogebra utiliza el comando
      \begin{center}
            \command{Si(<Condición>, <Entonces>, <Si no>)} 
      \end{center}
      y se pueden anidar varios comandos unos dentro de otros. 
      Utilizando este comando para representar la función anterior, habría que introducir la expresión
      \begin{center}
        Si[x<=0, -2x, x\^2]    
      \end{center}
      \end{indication}
\end{enumerate}


\section{Ejercicios propuestos}
\begin{enumerate}[leftmargin=*]
\item Hallar el dominio de las siguientes funciones a partir de sus representaciones gráficas:

      \begin{enumerate}
      \item $f(x)=\dfrac{x^{2} + x + 1}{x^{3} - x}$
      \item $g(x)=\sqrt[2]{x^{4}-1}$.
      \item $h(x)=\cos{\dfrac{x + 3}{x^{2} + 1}}$.
      \item $l(x)=\arcsen{\dfrac{x}{1+x}}$.
      \end{enumerate}

\item Se considera la función
      \[
      \ f(x)=\frac{x^{3} + x +2}{5x^{3} - 9x^{2} - 4x + 4}.
      \]

      Representarla gráficamente y determinar a partir de dicha representación:

      \begin{enumerate}
      \item Dominio.
      \item Imagen.
      \item Asíntotas.
      \item Raíces.
      \item Signo.
      \item Intervalos de crecimiento y decrecimiento.
      \item Intervalos de concavidad y convexidad.
      \item Extremos relativos.
      \item Puntos de inflexión.
      \end{enumerate}

\item Representar en una misma gráfica las funciones $\log_{10}{x}$, $\log_{2}{x}$, $\log{x}$, $\log_{0.5}{x}$.
      \begin{enumerate}
      \item A la vista de las gráficas obtenidas, indicar cuáles de las funciones anteriores son crecientes y cuáles son decrecientes.
      \item Determinar, a partir de los resultados obtenidos, o representando nuevas funciones si fuera necesario, para qué valores de $a$ será
            creciente la función $\log_{a}{x}$.
      \item Determinar, a partir de los resultados obtenidos, o representando nuevas funciones si fuera necesario, para qué valores de $a$ será
            decreciente la función $\log_{a}{x}$.
      \end{enumerate}

\item Completar las siguientes frases con la palabra igual, o el número de veces que sea mayor o menor en cada caso:
      \begin{enumerate}
      \item La función $\cos{2x}$ tiene un período............ que la función $\cos{x}$.
      \item La función $\cos{2x}$ tiene una amplitud............ que la función $\cos{x}$.
      \item La función $\cos\dfrac{x}{2}$ tiene un período............ que la función $\cos{3x}$.
      \item La función $\cos\dfrac{x}{2}$ tiene una amplitud............ que la función $\cos{3x}$.
      \item La función $3\cos{2x}$ tiene un período............ que la función $\cos\dfrac{x}{2}$.
      \item La función $3\cos{2x}$ tiene una amplitud............ que la función $\cos\dfrac{x}{2}$.
      \end{enumerate}

\item Hallar a partir de la representación gráfica, las soluciones de $e^{-1/x}=\dfrac{1}{x}$.

\item Representar en una gráfica la función
      \[
      \ f(x)=\left\{
      \begin{array}{ll}
      x^{3}   & \hbox{si $x<0$}    \\
      e^{x}-1 & \hbox{si $x\geq0$} \\
      \end{array}
      \right.
      \]

\end{enumerate}

% a un cierto valor, finito o infinito.
% \subsubsection*% !TEX root = ../practicas_geogebra.tex
% Author: Alfredo Sánchez Alberca (asalber@ceu.es)
\chapter{Funciones Elementales}

% \section{Fundamentos teóricos}

% En esta práctica se introducen los conceptos básicos sobre funciones reales de variable real, esto es, funciones
% \[f:\mathbb{R}\rightarrow \mathbb{R}.\]

% \subsection{Dominio e imagen}

% El \emph{Dominio} de la función $f$ es el conjunto de los números reales $x$ para los que existe $f(x)$ y se designa mediante $\dom f$.

% La \emph{Imagen} de $f$ es el conjunto de los números reales $y$ para los que existe algún $x\in \mathbb{R}$ tal que $f(x)=y$, y se denota por $\im f$.


% \subsection{Signo y crecimiento}
% El \emph{signo} de la función es positivo $(+)$ en los valores de $x$ para los que $f(x)>0$ y negativo $(-)$ en los que $f(x)<0$.
% Los valores de $x$ en los que la función se anula se conocen como \emph{raíces} de la función.

% Una función $f(x)$ es \emph{creciente} en un intervalo $I$ si $\forall\, x_1, x_2 \in I$ tales que $x_1<x_2$ se verifica que $f(x_1)\leq f(x_2)$.

% Del mismo modo, se dice que una función $f(x)$ es \emph{decreciente} en un intervalo $I$ si $\forall\, x_1, x_2 \in I$ tales que $x_1<x_2$ se verifica que $f(x_1)\geq f(x_2)$. En la figura~\ref{g:crecimiento} se muestran estos conceptos.

% \begin{figure}[h!]
% 	\centering \subfigure[Función creciente.] {\label{g:funcion_creciente}
% 		\scalebox{1}{\input{img/funciones_elementales/funcion_creciente}}}\qquad
% 	\subfigure[Función decreciente.]{\label{g:funcion_decreciente}
% 		\scalebox{1}{\input{img/funciones_elementales/funcion_decreciente}}}
% 	\caption{Crecimiento de una función.}
% 	\label{g:crecimiento}
% \end{figure}


% \subsection{Extremos Relativos}
% Una función $f(x)$ tiene un \emph{máximo relativo} en $x_0$ si existe un entorno $A$ de $x_0$ tal que $\forall x \in A$
% se verifica que $f(x)\leq f(x_0)$.

% Una función $f(x)$ tiene un \emph{mínimo relativo} en $x_0$ si existe un entorno $A$ de $x_0$ tal que $\forall x\in A$
% se verifica que $f(x)\geq f(x_0)$.

% Diremos que la función $f(x)$ tiene un \emph{extremo relativo} en un punto si tiene un \emph{máximo o mínimo relativo}
% en dicho punto. Estos conceptos se muestran en la figura~\ref{g:extremos}.

% \begin{figure}[h!]
% 	\centering \subfigure[Máximo relativo.] {\label{g:maximo}
% 		\scalebox{1}{\input{img/funciones_elementales/maximo}}}\qquad
% 	\subfigure[Mínimo relativo.]{\label{g:minimo}
% 		\scalebox{1}{\input{img/funciones_elementales/minimo}}}
% 	\caption{Extremos relativos de una función.}
% 	\label{g:extremos}
% \end{figure}

% Una función $f(x)$ está \emph{acotada superiormente} si $\exists K\in\mathbb{R}$ tal que $f(x)\leq K$ $\forall x \in \dom f$. Análogamente, se dice que una función $f(x)$ está \emph{acotada inferiormente} si $\exists K\in\mathbb{R}$ tal que $f(x)\geq K$ $\forall x \in \dom f$.

% Una función $f(x)$ está \emph{acotada} si lo está superior e inferiormente, es decir si $\exists K\in\mathbb{R}$ tal que $|f(x)|\leq K$ $\forall x \in \dom f$.


% \subsection{Concavidad}

% De forma intuitiva se puede decir que una función $f(x)$ es \emph{cóncava} en un intervalo $I$ si $\forall\, x_1, x_2
% \in I$, el segmento de extremos $(x_1,f(x_1))$ y $(x_2,f(x_2))$ queda por encima de la gráfica de $f$.

% Análogamente se dirá que es \emph{convexa} si el segmento anterior queda por debajo de la gráfica de $f$.

% Diremos que la función $f(x)$ tiene un \emph{punto de inflexión} en $x_0$ si en ese punto la función pasa de cóncava a
% convexa o de convexa a cóncava. Estos conceptos se ilustran en la figura~\ref{g:concavidad}.

% \begin{figure}[h!]
% 	\centering \subfigure[Función cóncava.] {\label{g:funcion_convexa}
% 		\scalebox{1}{\input{img/funciones_elementales/funcion_convexa}}}\qquad
% 	\subfigure[Función convexa.]{\label{g:funcion_concava}
% 		\scalebox{1}{\input{img/funciones_elementales/funcion_concava}}}
% 	\caption{Concavidad de una función.}
% 	\label{g:concavidad}
% \end{figure}

% \subsection{Asíntotas}

% La recta $x=a$ es una \emph{asíntota vertical} de la función $f(x)$ si al menos uno de los límites laterales de $f(x)$ cuando $x$ tiende hacia $a$ es $+\infty$ o $-\infty$, es decir cuando se verifique alguna de las siguientes igualdades
% \[
% 	\ \lim_{x\rightarrow a^{+}}f(x)=\pm\infty   \quad \textrm{o} \quad
% 	\lim_{x\rightarrow a^{-}}f(x)=\pm\infty
% \]

% La recta $y=b$ es una \emph{asíntota horizontal} de la función $f(x)$ si alguno de los límites de $f(x)$ cuando $x$ tiende hacia $+\infty$ o $-\infty$ es igual a $b$, es decir cuando se verifique
% \[
% 	\ \lim_{x\rightarrow -\infty }f(x)=b    \quad \textrm{o} \quad
% 	\ \lim_{x\rightarrow +\infty }f(x)=b
% \]

% La recta $y=mx+n$ es una \emph{asíntota oblicua} de la función $f(x)$ si alguno de los límites de $f(x)-(mx+n)$ cuando $x$ tiende hacia $+\infty$ o $-\infty$ es igual a 0, es decir si

% \[
% 	\ \lim_{x\rightarrow -\infty }{(f(x)-mx)}=n    \quad \textrm{o} \quad
% 	\ \lim_{x\rightarrow +\infty }{(f(x)-mx)}=n
% \]

% En la figura~\ref{g:asintotas} se muestran los distintos tipos de asíntotas.

% \begin{figure}[h!]
% 	\centering \subfigure[Asíntota horizontal y vertical.] {\label{g:asintotahorizontalyvertical}
% 		\scalebox{1}{\input{img/funciones_elementales/asintota_vertical}}}\qquad\qquad
% 	\subfigure[Asíntota vertical y oblicua.]{\label{g:asintotaoblicua}
% 		\scalebox{1}{\input{img/funciones_elementales/asintota_oblicua}}}
% 	\caption{Tipos de asíntotas de una función.}
% 	\label{g:asintotas}
% \end{figure}


% \subsection{Periodicidad}
% Una función $f(x)$ es \emph{periódica} si existe $h\in\mathbb{R^{+}}$ tal que \[f(x+h)=f(x)\  \forall x\in \dom f\] siendo el período $T$ de la función, el menor valor $h$ que verifique la igualdad anterior.

% En una función periódica, por ejemplo $f(x)=A\sen(wt)$, se denomina \emph{amplitud} al valor de $A$, y es la mitad de la diferencia entre los valores máximos y mínimos de la función. En la figura~\ref{g:periodoyamplitud} se ilustran estos conceptos.

% \begin{figure}[h!]
% 	\centering
% 	\scalebox{0.8}{\input{img/funciones_elementales/funcion_periodica}}
% 	\caption{Periodo y amplitud de una función periódica.}
% 	\label{g:periodoyamplitud}
% \end{figure}

% \clearpage
% \newpage

\section{Ejercicios resueltos}

\begin{enumerate}[leftmargin=*]
\item Se considera la función
      \[
      f(t)=\frac{t^{4} +19\cdot t^{2} - 5}{t^{4} +9\cdot t^{2} - 10}.
      \]

      Representarla gráficamente y determinar a partir de dicha representación:

      \begin{enumerate}
      \item  Dominio.
            \begin{indication}
            \begin{enumerate}
            \item Para representarla gráficamente, introducir la función en la barra de \field{Entrada} de la \field{Vista CAS} y activar la \field{Vista Gráfica}.
            \item Para determinar el dominio tan sólo hay que determinar los valores de $x$ en los que existe la función.
            \item Recordar que, tanto para éste como para el resto de los apartados del ejercicio, pretendemos llegar a conclusiones aproximadas que tan sólo sacamos del análisis de la gráfica.
            \end{enumerate}
            \end{indication}

      \item  Imagen.
            \begin{indication}
            Fijarse en los valores de la variable $y$ hasta los que llega la función.
            \end{indication}

      \item  Asíntotas.
            \begin{indication}
            Son las líneas rectas, ya sea horizontales, verticales u oblicuas, hacia las que tiende la función.
            \end{indication}

      \item  Raíces.
            \begin{indication}
            Son los valores de la variable $x$, si los hay, en los que la función vale 0.
            \end{indication}

      \item Signo.
            \begin{indication}
            Hay que determinar, aproximadamente, por un lado los intervalos de variable $x$ en los que la función es positiva, y por el otro aquellos en
            los que es negativa.
            \end{indication}

      \item  Intervalos de crecimiento y decrecimiento.
            \begin{indication}
            De nuevo, por un lado hay que determinar los intervalos de variable $x$ en los que a medida que crece $x$ también lo hace $y$, que serían
            los intervalos de crecimiento, y también aquellos otros en los que a medida que crece $x$ decrece $y$, que serían los intervalos de
            decremimiento.
            \end{indication}

      \item Intervalos de concavidad y convexidad.
            \begin{indication}
            Para los intervalos de concavidad y convexidad, nos fijamos en el segmento de línea recta que une dos puntos cualquiera del intervalo. Si
            dicho segmento queda por encima de la gráfica, entonces la función es cóncava en el intervalo, mientras que si queda por debajo, entonces es
            convexa en el mismo.
            \end{indication}

      \item Extremos relativos.
            \begin{indication}
            Determinamos, aproximadamente, los puntos en los que se encuentran los máximos y mínimos relativos de la función.
            \end{indication}

      \item Puntos de inflexión.
            \begin{indication}
            Determinamos, aproximadamente, los puntos en los que la función cambia de curvatura, de cóncava a convexa o a la inversa.
            \end{indication}
      \end{enumerate}

\item Representar en una misma gráfica las funciones $2^{x}, e^{x}, 0.7^{x}, 0.5^{x}$. A la vista de las gráficas obtenidas, indicar cuáles
      de las funciones anteriores son crecientes y cuáles son decrecientes.
      \begin{indication}
      Introducir cada función en la barra de \field{Entrada} de la \field{Vista CAS}.
      \end{indication}

      ¿En general, para qué valores de $a$ será la función creciente? ¿Y para qué valores de $a$ será decreciente? Probar con
      distintos valores de $a$ representando gráficamente nuevas funciones si fuera necesario.


\item Representar en una misma gráfica las funciones siguientes, indicando su período y amplitud.
      \begin{enumerate}
      \item $\sen{x}$, $\sen{x}+2$, $\sen{(x+2)}$.
      \item $\sen{2x}$, $2\sen{x}$, $\sen\frac{x}{2}$.
            \begin{indication}
            Introducir cada función en la barra de \field{Entrada} de la \field{Vista CAS}.
            \end{indication}
      \end{enumerate}


\item Representar en una gráfica la función
      \[
      \ f(x)=\left\{
      \begin{array}{cl}
      -2x   & \hbox{si $x\leq0$;} \\
      x^{2} & \hbox{si $x>0$.}    \\
      \end{array}
      \right.
      \]

      \begin{indication}
      Para representar funciones a trozos, Geogebra utiliza el comando
      \begin{center}
            \command{Si(<Condición>, <Entonces>, <Si no>)} 
      \end{center}
      y se pueden anidar varios comandos unos dentro de otros. 
      Utilizando este comando para representar la función anterior, habría que introducir la expresión
      \begin{center}
        Si[x<=0, -2x, x\^2]    
      \end{center}
      \end{indication}
\end{enumerate}


\section{Ejercicios propuestos}
\begin{enumerate}[leftmargin=*]
\item Hallar el dominio de las siguientes funciones a partir de sus representaciones gráficas:

      \begin{enumerate}
      \item $f(x)=\dfrac{x^{2} + x + 1}{x^{3} - x}$
      \item $g(x)=\sqrt[2]{x^{4}-1}$.
      \item $h(x)=\cos{\dfrac{x + 3}{x^{2} + 1}}$.
      \item $l(x)=\arcsen{\dfrac{x}{1+x}}$.
      \end{enumerate}

\item Se considera la función
      \[
      \ f(x)=\frac{x^{3} + x +2}{5x^{3} - 9x^{2} - 4x + 4}.
      \]

      Representarla gráficamente y determinar a partir de dicha representación:

      \begin{enumerate}
      \item Dominio.
      \item Imagen.
      \item Asíntotas.
      \item Raíces.
      \item Signo.
      \item Intervalos de crecimiento y decrecimiento.
      \item Intervalos de concavidad y convexidad.
      \item Extremos relativos.
      \item Puntos de inflexión.
      \end{enumerate}

\item Representar en una misma gráfica las funciones $\log_{10}{x}$, $\log_{2}{x}$, $\log{x}$, $\log_{0.5}{x}$.
      \begin{enumerate}
      \item A la vista de las gráficas obtenidas, indicar cuáles de las funciones anteriores son crecientes y cuáles son decrecientes.
      \item Determinar, a partir de los resultados obtenidos, o representando nuevas funciones si fuera necesario, para qué valores de $a$ será
            creciente la función $\log_{a}{x}$.
      \item Determinar, a partir de los resultados obtenidos, o representando nuevas funciones si fuera necesario, para qué valores de $a$ será
            decreciente la función $\log_{a}{x}$.
      \end{enumerate}

\item Completar las siguientes frases con la palabra igual, o el número de veces que sea mayor o menor en cada caso:
      \begin{enumerate}
      \item La función $\cos{2x}$ tiene un período............ que la función $\cos{x}$.
      \item La función $\cos{2x}$ tiene una amplitud............ que la función $\cos{x}$.
      \item La función $\cos\dfrac{x}{2}$ tiene un período............ que la función $\cos{3x}$.
      \item La función $\cos\dfrac{x}{2}$ tiene una amplitud............ que la función $\cos{3x}$.
      \item La función $3\cos{2x}$ tiene un período............ que la función $\cos\dfrac{x}{2}$.
      \item La función $3\cos{2x}$ tiene una amplitud............ que la función $\cos\dfrac{x}{2}$.
      \end{enumerate}

\item Hallar a partir de la representación gráfica, las soluciones de $e^{-1/x}=\dfrac{1}{x}$.

\item Representar en una gráfica la función
      \[
      \ f(x)=\left\{
      \begin{array}{ll}
      x^{3}   & \hbox{si $x<0$}    \\
      e^{x}-1 & \hbox{si $x\geq0$} \\
      \end{array}
      \right.
      \]

\end{enumerate}

% verticales}
% La recta $x=a$ % !TEX root = ../practicas_geogebra.tex
% Author: Alfredo Sánchez Alberca (asalber@ceu.es)
\chapter{Funciones Elementales}

% \section{Fundamentos teóricos}

% En esta práctica se introducen los conceptos básicos sobre funciones reales de variable real, esto es, funciones
% \[f:\mathbb{R}\rightarrow \mathbb{R}.\]

% \subsection{Dominio e imagen}

% El \emph{Dominio} de la función $f$ es el conjunto de los números reales $x$ para los que existe $f(x)$ y se designa mediante $\dom f$.

% La \emph{Imagen} de $f$ es el conjunto de los números reales $y$ para los que existe algún $x\in \mathbb{R}$ tal que $f(x)=y$, y se denota por $\im f$.


% \subsection{Signo y crecimiento}
% El \emph{signo} de la función es positivo $(+)$ en los valores de $x$ para los que $f(x)>0$ y negativo $(-)$ en los que $f(x)<0$.
% Los valores de $x$ en los que la función se anula se conocen como \emph{raíces} de la función.

% Una función $f(x)$ es \emph{creciente} en un intervalo $I$ si $\forall\, x_1, x_2 \in I$ tales que $x_1<x_2$ se verifica que $f(x_1)\leq f(x_2)$.

% Del mismo modo, se dice que una función $f(x)$ es \emph{decreciente} en un intervalo $I$ si $\forall\, x_1, x_2 \in I$ tales que $x_1<x_2$ se verifica que $f(x_1)\geq f(x_2)$. En la figura~\ref{g:crecimiento} se muestran estos conceptos.

% \begin{figure}[h!]
% 	\centering \subfigure[Función creciente.] {\label{g:funcion_creciente}
% 		\scalebox{1}{\input{img/funciones_elementales/funcion_creciente}}}\qquad
% 	\subfigure[Función decreciente.]{\label{g:funcion_decreciente}
% 		\scalebox{1}{\input{img/funciones_elementales/funcion_decreciente}}}
% 	\caption{Crecimiento de una función.}
% 	\label{g:crecimiento}
% \end{figure}


% \subsection{Extremos Relativos}
% Una función $f(x)$ tiene un \emph{máximo relativo} en $x_0$ si existe un entorno $A$ de $x_0$ tal que $\forall x \in A$
% se verifica que $f(x)\leq f(x_0)$.

% Una función $f(x)$ tiene un \emph{mínimo relativo} en $x_0$ si existe un entorno $A$ de $x_0$ tal que $\forall x\in A$
% se verifica que $f(x)\geq f(x_0)$.

% Diremos que la función $f(x)$ tiene un \emph{extremo relativo} en un punto si tiene un \emph{máximo o mínimo relativo}
% en dicho punto. Estos conceptos se muestran en la figura~\ref{g:extremos}.

% \begin{figure}[h!]
% 	\centering \subfigure[Máximo relativo.] {\label{g:maximo}
% 		\scalebox{1}{\input{img/funciones_elementales/maximo}}}\qquad
% 	\subfigure[Mínimo relativo.]{\label{g:minimo}
% 		\scalebox{1}{\input{img/funciones_elementales/minimo}}}
% 	\caption{Extremos relativos de una función.}
% 	\label{g:extremos}
% \end{figure}

% Una función $f(x)$ está \emph{acotada superiormente} si $\exists K\in\mathbb{R}$ tal que $f(x)\leq K$ $\forall x \in \dom f$. Análogamente, se dice que una función $f(x)$ está \emph{acotada inferiormente} si $\exists K\in\mathbb{R}$ tal que $f(x)\geq K$ $\forall x \in \dom f$.

% Una función $f(x)$ está \emph{acotada} si lo está superior e inferiormente, es decir si $\exists K\in\mathbb{R}$ tal que $|f(x)|\leq K$ $\forall x \in \dom f$.


% \subsection{Concavidad}

% De forma intuitiva se puede decir que una función $f(x)$ es \emph{cóncava} en un intervalo $I$ si $\forall\, x_1, x_2
% \in I$, el segmento de extremos $(x_1,f(x_1))$ y $(x_2,f(x_2))$ queda por encima de la gráfica de $f$.

% Análogamente se dirá que es \emph{convexa} si el segmento anterior queda por debajo de la gráfica de $f$.

% Diremos que la función $f(x)$ tiene un \emph{punto de inflexión} en $x_0$ si en ese punto la función pasa de cóncava a
% convexa o de convexa a cóncava. Estos conceptos se ilustran en la figura~\ref{g:concavidad}.

% \begin{figure}[h!]
% 	\centering \subfigure[Función cóncava.] {\label{g:funcion_convexa}
% 		\scalebox{1}{\input{img/funciones_elementales/funcion_convexa}}}\qquad
% 	\subfigure[Función convexa.]{\label{g:funcion_concava}
% 		\scalebox{1}{\input{img/funciones_elementales/funcion_concava}}}
% 	\caption{Concavidad de una función.}
% 	\label{g:concavidad}
% \end{figure}

% \subsection{Asíntotas}

% La recta $x=a$ es una \emph{asíntota vertical} de la función $f(x)$ si al menos uno de los límites laterales de $f(x)$ cuando $x$ tiende hacia $a$ es $+\infty$ o $-\infty$, es decir cuando se verifique alguna de las siguientes igualdades
% \[
% 	\ \lim_{x\rightarrow a^{+}}f(x)=\pm\infty   \quad \textrm{o} \quad
% 	\lim_{x\rightarrow a^{-}}f(x)=\pm\infty
% \]

% La recta $y=b$ es una \emph{asíntota horizontal} de la función $f(x)$ si alguno de los límites de $f(x)$ cuando $x$ tiende hacia $+\infty$ o $-\infty$ es igual a $b$, es decir cuando se verifique
% \[
% 	\ \lim_{x\rightarrow -\infty }f(x)=b    \quad \textrm{o} \quad
% 	\ \lim_{x\rightarrow +\infty }f(x)=b
% \]

% La recta $y=mx+n$ es una \emph{asíntota oblicua} de la función $f(x)$ si alguno de los límites de $f(x)-(mx+n)$ cuando $x$ tiende hacia $+\infty$ o $-\infty$ es igual a 0, es decir si

% \[
% 	\ \lim_{x\rightarrow -\infty }{(f(x)-mx)}=n    \quad \textrm{o} \quad
% 	\ \lim_{x\rightarrow +\infty }{(f(x)-mx)}=n
% \]

% En la figura~\ref{g:asintotas} se muestran los distintos tipos de asíntotas.

% \begin{figure}[h!]
% 	\centering \subfigure[Asíntota horizontal y vertical.] {\label{g:asintotahorizontalyvertical}
% 		\scalebox{1}{\input{img/funciones_elementales/asintota_vertical}}}\qquad\qquad
% 	\subfigure[Asíntota vertical y oblicua.]{\label{g:asintotaoblicua}
% 		\scalebox{1}{\input{img/funciones_elementales/asintota_oblicua}}}
% 	\caption{Tipos de asíntotas de una función.}
% 	\label{g:asintotas}
% \end{figure}


% \subsection{Periodicidad}
% Una función $f(x)$ es \emph{periódica} si existe $h\in\mathbb{R^{+}}$ tal que \[f(x+h)=f(x)\  \forall x\in \dom f\] siendo el período $T$ de la función, el menor valor $h$ que verifique la igualdad anterior.

% En una función periódica, por ejemplo $f(x)=A\sen(wt)$, se denomina \emph{amplitud} al valor de $A$, y es la mitad de la diferencia entre los valores máximos y mínimos de la función. En la figura~\ref{g:periodoyamplitud} se ilustran estos conceptos.

% \begin{figure}[h!]
% 	\centering
% 	\scalebox{0.8}{\input{img/funciones_elementales/funcion_periodica}}
% 	\caption{Periodo y amplitud de una función periódica.}
% 	\label{g:periodoyamplitud}
% \end{figure}

% \clearpage
% \newpage

\section{Ejercicios resueltos}

\begin{enumerate}[leftmargin=*]
\item Se considera la función
      \[
      f(t)=\frac{t^{4} +19\cdot t^{2} - 5}{t^{4} +9\cdot t^{2} - 10}.
      \]

      Representarla gráficamente y determinar a partir de dicha representación:

      \begin{enumerate}
      \item  Dominio.
            \begin{indication}
            \begin{enumerate}
            \item Para representarla gráficamente, introducir la función en la barra de \field{Entrada} de la \field{Vista CAS} y activar la \field{Vista Gráfica}.
            \item Para determinar el dominio tan sólo hay que determinar los valores de $x$ en los que existe la función.
            \item Recordar que, tanto para éste como para el resto de los apartados del ejercicio, pretendemos llegar a conclusiones aproximadas que tan sólo sacamos del análisis de la gráfica.
            \end{enumerate}
            \end{indication}

      \item  Imagen.
            \begin{indication}
            Fijarse en los valores de la variable $y$ hasta los que llega la función.
            \end{indication}

      \item  Asíntotas.
            \begin{indication}
            Son las líneas rectas, ya sea horizontales, verticales u oblicuas, hacia las que tiende la función.
            \end{indication}

      \item  Raíces.
            \begin{indication}
            Son los valores de la variable $x$, si los hay, en los que la función vale 0.
            \end{indication}

      \item Signo.
            \begin{indication}
            Hay que determinar, aproximadamente, por un lado los intervalos de variable $x$ en los que la función es positiva, y por el otro aquellos en
            los que es negativa.
            \end{indication}

      \item  Intervalos de crecimiento y decrecimiento.
            \begin{indication}
            De nuevo, por un lado hay que determinar los intervalos de variable $x$ en los que a medida que crece $x$ también lo hace $y$, que serían
            los intervalos de crecimiento, y también aquellos otros en los que a medida que crece $x$ decrece $y$, que serían los intervalos de
            decremimiento.
            \end{indication}

      \item Intervalos de concavidad y convexidad.
            \begin{indication}
            Para los intervalos de concavidad y convexidad, nos fijamos en el segmento de línea recta que une dos puntos cualquiera del intervalo. Si
            dicho segmento queda por encima de la gráfica, entonces la función es cóncava en el intervalo, mientras que si queda por debajo, entonces es
            convexa en el mismo.
            \end{indication}

      \item Extremos relativos.
            \begin{indication}
            Determinamos, aproximadamente, los puntos en los que se encuentran los máximos y mínimos relativos de la función.
            \end{indication}

      \item Puntos de inflexión.
            \begin{indication}
            Determinamos, aproximadamente, los puntos en los que la función cambia de curvatura, de cóncava a convexa o a la inversa.
            \end{indication}
      \end{enumerate}

\item Representar en una misma gráfica las funciones $2^{x}, e^{x}, 0.7^{x}, 0.5^{x}$. A la vista de las gráficas obtenidas, indicar cuáles
      de las funciones anteriores son crecientes y cuáles son decrecientes.
      \begin{indication}
      Introducir cada función en la barra de \field{Entrada} de la \field{Vista CAS}.
      \end{indication}

      ¿En general, para qué valores de $a$ será la función creciente? ¿Y para qué valores de $a$ será decreciente? Probar con
      distintos valores de $a$ representando gráficamente nuevas funciones si fuera necesario.


\item Representar en una misma gráfica las funciones siguientes, indicando su período y amplitud.
      \begin{enumerate}
      \item $\sen{x}$, $\sen{x}+2$, $\sen{(x+2)}$.
      \item $\sen{2x}$, $2\sen{x}$, $\sen\frac{x}{2}$.
            \begin{indication}
            Introducir cada función en la barra de \field{Entrada} de la \field{Vista CAS}.
            \end{indication}
      \end{enumerate}


\item Representar en una gráfica la función
      \[
      \ f(x)=\left\{
      \begin{array}{cl}
      -2x   & \hbox{si $x\leq0$;} \\
      x^{2} & \hbox{si $x>0$.}    \\
      \end{array}
      \right.
      \]

      \begin{indication}
      Para representar funciones a trozos, Geogebra utiliza el comando
      \begin{center}
            \command{Si(<Condición>, <Entonces>, <Si no>)} 
      \end{center}
      y se pueden anidar varios comandos unos dentro de otros. 
      Utilizando este comando para representar la función anterior, habría que introducir la expresión
      \begin{center}
        Si[x<=0, -2x, x\^2]    
      \end{center}
      \end{indication}
\end{enumerate}


\section{Ejercicios propuestos}
\begin{enumerate}[leftmargin=*]
\item Hallar el dominio de las siguientes funciones a partir de sus representaciones gráficas:

      \begin{enumerate}
      \item $f(x)=\dfrac{x^{2} + x + 1}{x^{3} - x}$
      \item $g(x)=\sqrt[2]{x^{4}-1}$.
      \item $h(x)=\cos{\dfrac{x + 3}{x^{2} + 1}}$.
      \item $l(x)=\arcsen{\dfrac{x}{1+x}}$.
      \end{enumerate}

\item Se considera la función
      \[
      \ f(x)=\frac{x^{3} + x +2}{5x^{3} - 9x^{2} - 4x + 4}.
      \]

      Representarla gráficamente y determinar a partir de dicha representación:

      \begin{enumerate}
      \item Dominio.
      \item Imagen.
      \item Asíntotas.
      \item Raíces.
      \item Signo.
      \item Intervalos de crecimiento y decrecimiento.
      \item Intervalos de concavidad y convexidad.
      \item Extremos relativos.
      \item Puntos de inflexión.
      \end{enumerate}

\item Representar en una misma gráfica las funciones $\log_{10}{x}$, $\log_{2}{x}$, $\log{x}$, $\log_{0.5}{x}$.
      \begin{enumerate}
      \item A la vista de las gráficas obtenidas, indicar cuáles de las funciones anteriores son crecientes y cuáles son decrecientes.
      \item Determinar, a partir de los resultados obtenidos, o representando nuevas funciones si fuera necesario, para qué valores de $a$ será
            creciente la función $\log_{a}{x}$.
      \item Determinar, a partir de los resultados obtenidos, o representando nuevas funciones si fuera necesario, para qué valores de $a$ será
            decreciente la función $\log_{a}{x}$.
      \end{enumerate}

\item Completar las siguientes frases con la palabra igual, o el número de veces que sea mayor o menor en cada caso:
      \begin{enumerate}
      \item La función $\cos{2x}$ tiene un período............ que la función $\cos{x}$.
      \item La función $\cos{2x}$ tiene una amplitud............ que la función $\cos{x}$.
      \item La función $\cos\dfrac{x}{2}$ tiene un período............ que la función $\cos{3x}$.
      \item La función $\cos\dfrac{x}{2}$ tiene una amplitud............ que la función $\cos{3x}$.
      \item La función $3\cos{2x}$ tiene un período............ que la función $\cos\dfrac{x}{2}$.
      \item La función $3\cos{2x}$ tiene una amplitud............ que la función $\cos\dfrac{x}{2}$.
      \end{enumerate}

\item Hallar a partir de la representación gráfica, las soluciones de $e^{-1/x}=\dfrac{1}{x}$.

\item Representar en una gráfica la función
      \[
      \ f(x)=\left\{
      \begin{array}{ll}
      x^{3}   & \hbox{si $x<0$}    \\
      e^{x}-1 & \hbox{si $x\geq0$} \\
      \end{array}
      \right.
      \]

\end{enumerate}

% h{Asíntota Vertical} de la función $f(x)$
% si al menos uno% !TEX root = ../practicas_geogebra.tex
% Author: Alfredo Sánchez Alberca (asalber@ceu.es)
\chapter{Funciones Elementales}

% \section{Fundamentos teóricos}

% En esta práctica se introducen los conceptos básicos sobre funciones reales de variable real, esto es, funciones
% \[f:\mathbb{R}\rightarrow \mathbb{R}.\]

% \subsection{Dominio e imagen}

% El \emph{Dominio} de la función $f$ es el conjunto de los números reales $x$ para los que existe $f(x)$ y se designa mediante $\dom f$.

% La \emph{Imagen} de $f$ es el conjunto de los números reales $y$ para los que existe algún $x\in \mathbb{R}$ tal que $f(x)=y$, y se denota por $\im f$.


% \subsection{Signo y crecimiento}
% El \emph{signo} de la función es positivo $(+)$ en los valores de $x$ para los que $f(x)>0$ y negativo $(-)$ en los que $f(x)<0$.
% Los valores de $x$ en los que la función se anula se conocen como \emph{raíces} de la función.

% Una función $f(x)$ es \emph{creciente} en un intervalo $I$ si $\forall\, x_1, x_2 \in I$ tales que $x_1<x_2$ se verifica que $f(x_1)\leq f(x_2)$.

% Del mismo modo, se dice que una función $f(x)$ es \emph{decreciente} en un intervalo $I$ si $\forall\, x_1, x_2 \in I$ tales que $x_1<x_2$ se verifica que $f(x_1)\geq f(x_2)$. En la figura~\ref{g:crecimiento} se muestran estos conceptos.

% \begin{figure}[h!]
% 	\centering \subfigure[Función creciente.] {\label{g:funcion_creciente}
% 		\scalebox{1}{\input{img/funciones_elementales/funcion_creciente}}}\qquad
% 	\subfigure[Función decreciente.]{\label{g:funcion_decreciente}
% 		\scalebox{1}{\input{img/funciones_elementales/funcion_decreciente}}}
% 	\caption{Crecimiento de una función.}
% 	\label{g:crecimiento}
% \end{figure}


% \subsection{Extremos Relativos}
% Una función $f(x)$ tiene un \emph{máximo relativo} en $x_0$ si existe un entorno $A$ de $x_0$ tal que $\forall x \in A$
% se verifica que $f(x)\leq f(x_0)$.

% Una función $f(x)$ tiene un \emph{mínimo relativo} en $x_0$ si existe un entorno $A$ de $x_0$ tal que $\forall x\in A$
% se verifica que $f(x)\geq f(x_0)$.

% Diremos que la función $f(x)$ tiene un \emph{extremo relativo} en un punto si tiene un \emph{máximo o mínimo relativo}
% en dicho punto. Estos conceptos se muestran en la figura~\ref{g:extremos}.

% \begin{figure}[h!]
% 	\centering \subfigure[Máximo relativo.] {\label{g:maximo}
% 		\scalebox{1}{\input{img/funciones_elementales/maximo}}}\qquad
% 	\subfigure[Mínimo relativo.]{\label{g:minimo}
% 		\scalebox{1}{\input{img/funciones_elementales/minimo}}}
% 	\caption{Extremos relativos de una función.}
% 	\label{g:extremos}
% \end{figure}

% Una función $f(x)$ está \emph{acotada superiormente} si $\exists K\in\mathbb{R}$ tal que $f(x)\leq K$ $\forall x \in \dom f$. Análogamente, se dice que una función $f(x)$ está \emph{acotada inferiormente} si $\exists K\in\mathbb{R}$ tal que $f(x)\geq K$ $\forall x \in \dom f$.

% Una función $f(x)$ está \emph{acotada} si lo está superior e inferiormente, es decir si $\exists K\in\mathbb{R}$ tal que $|f(x)|\leq K$ $\forall x \in \dom f$.


% \subsection{Concavidad}

% De forma intuitiva se puede decir que una función $f(x)$ es \emph{cóncava} en un intervalo $I$ si $\forall\, x_1, x_2
% \in I$, el segmento de extremos $(x_1,f(x_1))$ y $(x_2,f(x_2))$ queda por encima de la gráfica de $f$.

% Análogamente se dirá que es \emph{convexa} si el segmento anterior queda por debajo de la gráfica de $f$.

% Diremos que la función $f(x)$ tiene un \emph{punto de inflexión} en $x_0$ si en ese punto la función pasa de cóncava a
% convexa o de convexa a cóncava. Estos conceptos se ilustran en la figura~\ref{g:concavidad}.

% \begin{figure}[h!]
% 	\centering \subfigure[Función cóncava.] {\label{g:funcion_convexa}
% 		\scalebox{1}{\input{img/funciones_elementales/funcion_convexa}}}\qquad
% 	\subfigure[Función convexa.]{\label{g:funcion_concava}
% 		\scalebox{1}{\input{img/funciones_elementales/funcion_concava}}}
% 	\caption{Concavidad de una función.}
% 	\label{g:concavidad}
% \end{figure}

% \subsection{Asíntotas}

% La recta $x=a$ es una \emph{asíntota vertical} de la función $f(x)$ si al menos uno de los límites laterales de $f(x)$ cuando $x$ tiende hacia $a$ es $+\infty$ o $-\infty$, es decir cuando se verifique alguna de las siguientes igualdades
% \[
% 	\ \lim_{x\rightarrow a^{+}}f(x)=\pm\infty   \quad \textrm{o} \quad
% 	\lim_{x\rightarrow a^{-}}f(x)=\pm\infty
% \]

% La recta $y=b$ es una \emph{asíntota horizontal} de la función $f(x)$ si alguno de los límites de $f(x)$ cuando $x$ tiende hacia $+\infty$ o $-\infty$ es igual a $b$, es decir cuando se verifique
% \[
% 	\ \lim_{x\rightarrow -\infty }f(x)=b    \quad \textrm{o} \quad
% 	\ \lim_{x\rightarrow +\infty }f(x)=b
% \]

% La recta $y=mx+n$ es una \emph{asíntota oblicua} de la función $f(x)$ si alguno de los límites de $f(x)-(mx+n)$ cuando $x$ tiende hacia $+\infty$ o $-\infty$ es igual a 0, es decir si

% \[
% 	\ \lim_{x\rightarrow -\infty }{(f(x)-mx)}=n    \quad \textrm{o} \quad
% 	\ \lim_{x\rightarrow +\infty }{(f(x)-mx)}=n
% \]

% En la figura~\ref{g:asintotas} se muestran los distintos tipos de asíntotas.

% \begin{figure}[h!]
% 	\centering \subfigure[Asíntota horizontal y vertical.] {\label{g:asintotahorizontalyvertical}
% 		\scalebox{1}{\input{img/funciones_elementales/asintota_vertical}}}\qquad\qquad
% 	\subfigure[Asíntota vertical y oblicua.]{\label{g:asintotaoblicua}
% 		\scalebox{1}{\input{img/funciones_elementales/asintota_oblicua}}}
% 	\caption{Tipos de asíntotas de una función.}
% 	\label{g:asintotas}
% \end{figure}


% \subsection{Periodicidad}
% Una función $f(x)$ es \emph{periódica} si existe $h\in\mathbb{R^{+}}$ tal que \[f(x+h)=f(x)\  \forall x\in \dom f\] siendo el período $T$ de la función, el menor valor $h$ que verifique la igualdad anterior.

% En una función periódica, por ejemplo $f(x)=A\sen(wt)$, se denomina \emph{amplitud} al valor de $A$, y es la mitad de la diferencia entre los valores máximos y mínimos de la función. En la figura~\ref{g:periodoyamplitud} se ilustran estos conceptos.

% \begin{figure}[h!]
% 	\centering
% 	\scalebox{0.8}{\input{img/funciones_elementales/funcion_periodica}}
% 	\caption{Periodo y amplitud de una función periódica.}
% 	\label{g:periodoyamplitud}
% \end{figure}

% \clearpage
% \newpage

\section{Ejercicios resueltos}

\begin{enumerate}[leftmargin=*]
\item Se considera la función
      \[
      f(t)=\frac{t^{4} +19\cdot t^{2} - 5}{t^{4} +9\cdot t^{2} - 10}.
      \]

      Representarla gráficamente y determinar a partir de dicha representación:

      \begin{enumerate}
      \item  Dominio.
            \begin{indication}
            \begin{enumerate}
            \item Para representarla gráficamente, introducir la función en la barra de \field{Entrada} de la \field{Vista CAS} y activar la \field{Vista Gráfica}.
            \item Para determinar el dominio tan sólo hay que determinar los valores de $x$ en los que existe la función.
            \item Recordar que, tanto para éste como para el resto de los apartados del ejercicio, pretendemos llegar a conclusiones aproximadas que tan sólo sacamos del análisis de la gráfica.
            \end{enumerate}
            \end{indication}

      \item  Imagen.
            \begin{indication}
            Fijarse en los valores de la variable $y$ hasta los que llega la función.
            \end{indication}

      \item  Asíntotas.
            \begin{indication}
            Son las líneas rectas, ya sea horizontales, verticales u oblicuas, hacia las que tiende la función.
            \end{indication}

      \item  Raíces.
            \begin{indication}
            Son los valores de la variable $x$, si los hay, en los que la función vale 0.
            \end{indication}

      \item Signo.
            \begin{indication}
            Hay que determinar, aproximadamente, por un lado los intervalos de variable $x$ en los que la función es positiva, y por el otro aquellos en
            los que es negativa.
            \end{indication}

      \item  Intervalos de crecimiento y decrecimiento.
            \begin{indication}
            De nuevo, por un lado hay que determinar los intervalos de variable $x$ en los que a medida que crece $x$ también lo hace $y$, que serían
            los intervalos de crecimiento, y también aquellos otros en los que a medida que crece $x$ decrece $y$, que serían los intervalos de
            decremimiento.
            \end{indication}

      \item Intervalos de concavidad y convexidad.
            \begin{indication}
            Para los intervalos de concavidad y convexidad, nos fijamos en el segmento de línea recta que une dos puntos cualquiera del intervalo. Si
            dicho segmento queda por encima de la gráfica, entonces la función es cóncava en el intervalo, mientras que si queda por debajo, entonces es
            convexa en el mismo.
            \end{indication}

      \item Extremos relativos.
            \begin{indication}
            Determinamos, aproximadamente, los puntos en los que se encuentran los máximos y mínimos relativos de la función.
            \end{indication}

      \item Puntos de inflexión.
            \begin{indication}
            Determinamos, aproximadamente, los puntos en los que la función cambia de curvatura, de cóncava a convexa o a la inversa.
            \end{indication}
      \end{enumerate}

\item Representar en una misma gráfica las funciones $2^{x}, e^{x}, 0.7^{x}, 0.5^{x}$. A la vista de las gráficas obtenidas, indicar cuáles
      de las funciones anteriores son crecientes y cuáles son decrecientes.
      \begin{indication}
      Introducir cada función en la barra de \field{Entrada} de la \field{Vista CAS}.
      \end{indication}

      ¿En general, para qué valores de $a$ será la función creciente? ¿Y para qué valores de $a$ será decreciente? Probar con
      distintos valores de $a$ representando gráficamente nuevas funciones si fuera necesario.


\item Representar en una misma gráfica las funciones siguientes, indicando su período y amplitud.
      \begin{enumerate}
      \item $\sen{x}$, $\sen{x}+2$, $\sen{(x+2)}$.
      \item $\sen{2x}$, $2\sen{x}$, $\sen\frac{x}{2}$.
            \begin{indication}
            Introducir cada función en la barra de \field{Entrada} de la \field{Vista CAS}.
            \end{indication}
      \end{enumerate}


\item Representar en una gráfica la función
      \[
      \ f(x)=\left\{
      \begin{array}{cl}
      -2x   & \hbox{si $x\leq0$;} \\
      x^{2} & \hbox{si $x>0$.}    \\
      \end{array}
      \right.
      \]

      \begin{indication}
      Para representar funciones a trozos, Geogebra utiliza el comando
      \begin{center}
            \command{Si(<Condición>, <Entonces>, <Si no>)} 
      \end{center}
      y se pueden anidar varios comandos unos dentro de otros. 
      Utilizando este comando para representar la función anterior, habría que introducir la expresión
      \begin{center}
        Si[x<=0, -2x, x\^2]    
      \end{center}
      \end{indication}
\end{enumerate}


\section{Ejercicios propuestos}
\begin{enumerate}[leftmargin=*]
\item Hallar el dominio de las siguientes funciones a partir de sus representaciones gráficas:

      \begin{enumerate}
      \item $f(x)=\dfrac{x^{2} + x + 1}{x^{3} - x}$
      \item $g(x)=\sqrt[2]{x^{4}-1}$.
      \item $h(x)=\cos{\dfrac{x + 3}{x^{2} + 1}}$.
      \item $l(x)=\arcsen{\dfrac{x}{1+x}}$.
      \end{enumerate}

\item Se considera la función
      \[
      \ f(x)=\frac{x^{3} + x +2}{5x^{3} - 9x^{2} - 4x + 4}.
      \]

      Representarla gráficamente y determinar a partir de dicha representación:

      \begin{enumerate}
      \item Dominio.
      \item Imagen.
      \item Asíntotas.
      \item Raíces.
      \item Signo.
      \item Intervalos de crecimiento y decrecimiento.
      \item Intervalos de concavidad y convexidad.
      \item Extremos relativos.
      \item Puntos de inflexión.
      \end{enumerate}

\item Representar en una misma gráfica las funciones $\log_{10}{x}$, $\log_{2}{x}$, $\log{x}$, $\log_{0.5}{x}$.
      \begin{enumerate}
      \item A la vista de las gráficas obtenidas, indicar cuáles de las funciones anteriores son crecientes y cuáles son decrecientes.
      \item Determinar, a partir de los resultados obtenidos, o representando nuevas funciones si fuera necesario, para qué valores de $a$ será
            creciente la función $\log_{a}{x}$.
      \item Determinar, a partir de los resultados obtenidos, o representando nuevas funciones si fuera necesario, para qué valores de $a$ será
            decreciente la función $\log_{a}{x}$.
      \end{enumerate}

\item Completar las siguientes frases con la palabra igual, o el número de veces que sea mayor o menor en cada caso:
      \begin{enumerate}
      \item La función $\cos{2x}$ tiene un período............ que la función $\cos{x}$.
      \item La función $\cos{2x}$ tiene una amplitud............ que la función $\cos{x}$.
      \item La función $\cos\dfrac{x}{2}$ tiene un período............ que la función $\cos{3x}$.
      \item La función $\cos\dfrac{x}{2}$ tiene una amplitud............ que la función $\cos{3x}$.
      \item La función $3\cos{2x}$ tiene un período............ que la función $\cos\dfrac{x}{2}$.
      \item La función $3\cos{2x}$ tiene una amplitud............ que la función $\cos\dfrac{x}{2}$.
      \end{enumerate}

\item Hallar a partir de la representación gráfica, las soluciones de $e^{-1/x}=\dfrac{1}{x}$.

\item Representar en una gráfica la función
      \[
      \ f(x)=\left\{
      \begin{array}{ll}
      x^{3}   & \hbox{si $x<0$}    \\
      e^{x}-1 & \hbox{si $x\geq0$} \\
      \end{array}
      \right.
      \]

\end{enumerate}

% ites laterales de $f$ en $a$ es $+\infty$
% ó $+\infty$. Es% !TEX root = ../practicas_geogebra.tex
% Author: Alfredo Sánchez Alberca (asalber@ceu.es)
\chapter{Funciones Elementales}

% \section{Fundamentos teóricos}

% En esta práctica se introducen los conceptos básicos sobre funciones reales de variable real, esto es, funciones
% \[f:\mathbb{R}\rightarrow \mathbb{R}.\]

% \subsection{Dominio e imagen}

% El \emph{Dominio} de la función $f$ es el conjunto de los números reales $x$ para los que existe $f(x)$ y se designa mediante $\dom f$.

% La \emph{Imagen} de $f$ es el conjunto de los números reales $y$ para los que existe algún $x\in \mathbb{R}$ tal que $f(x)=y$, y se denota por $\im f$.


% \subsection{Signo y crecimiento}
% El \emph{signo} de la función es positivo $(+)$ en los valores de $x$ para los que $f(x)>0$ y negativo $(-)$ en los que $f(x)<0$.
% Los valores de $x$ en los que la función se anula se conocen como \emph{raíces} de la función.

% Una función $f(x)$ es \emph{creciente} en un intervalo $I$ si $\forall\, x_1, x_2 \in I$ tales que $x_1<x_2$ se verifica que $f(x_1)\leq f(x_2)$.

% Del mismo modo, se dice que una función $f(x)$ es \emph{decreciente} en un intervalo $I$ si $\forall\, x_1, x_2 \in I$ tales que $x_1<x_2$ se verifica que $f(x_1)\geq f(x_2)$. En la figura~\ref{g:crecimiento} se muestran estos conceptos.

% \begin{figure}[h!]
% 	\centering \subfigure[Función creciente.] {\label{g:funcion_creciente}
% 		\scalebox{1}{\input{img/funciones_elementales/funcion_creciente}}}\qquad
% 	\subfigure[Función decreciente.]{\label{g:funcion_decreciente}
% 		\scalebox{1}{\input{img/funciones_elementales/funcion_decreciente}}}
% 	\caption{Crecimiento de una función.}
% 	\label{g:crecimiento}
% \end{figure}


% \subsection{Extremos Relativos}
% Una función $f(x)$ tiene un \emph{máximo relativo} en $x_0$ si existe un entorno $A$ de $x_0$ tal que $\forall x \in A$
% se verifica que $f(x)\leq f(x_0)$.

% Una función $f(x)$ tiene un \emph{mínimo relativo} en $x_0$ si existe un entorno $A$ de $x_0$ tal que $\forall x\in A$
% se verifica que $f(x)\geq f(x_0)$.

% Diremos que la función $f(x)$ tiene un \emph{extremo relativo} en un punto si tiene un \emph{máximo o mínimo relativo}
% en dicho punto. Estos conceptos se muestran en la figura~\ref{g:extremos}.

% \begin{figure}[h!]
% 	\centering \subfigure[Máximo relativo.] {\label{g:maximo}
% 		\scalebox{1}{\input{img/funciones_elementales/maximo}}}\qquad
% 	\subfigure[Mínimo relativo.]{\label{g:minimo}
% 		\scalebox{1}{\input{img/funciones_elementales/minimo}}}
% 	\caption{Extremos relativos de una función.}
% 	\label{g:extremos}
% \end{figure}

% Una función $f(x)$ está \emph{acotada superiormente} si $\exists K\in\mathbb{R}$ tal que $f(x)\leq K$ $\forall x \in \dom f$. Análogamente, se dice que una función $f(x)$ está \emph{acotada inferiormente} si $\exists K\in\mathbb{R}$ tal que $f(x)\geq K$ $\forall x \in \dom f$.

% Una función $f(x)$ está \emph{acotada} si lo está superior e inferiormente, es decir si $\exists K\in\mathbb{R}$ tal que $|f(x)|\leq K$ $\forall x \in \dom f$.


% \subsection{Concavidad}

% De forma intuitiva se puede decir que una función $f(x)$ es \emph{cóncava} en un intervalo $I$ si $\forall\, x_1, x_2
% \in I$, el segmento de extremos $(x_1,f(x_1))$ y $(x_2,f(x_2))$ queda por encima de la gráfica de $f$.

% Análogamente se dirá que es \emph{convexa} si el segmento anterior queda por debajo de la gráfica de $f$.

% Diremos que la función $f(x)$ tiene un \emph{punto de inflexión} en $x_0$ si en ese punto la función pasa de cóncava a
% convexa o de convexa a cóncava. Estos conceptos se ilustran en la figura~\ref{g:concavidad}.

% \begin{figure}[h!]
% 	\centering \subfigure[Función cóncava.] {\label{g:funcion_convexa}
% 		\scalebox{1}{\input{img/funciones_elementales/funcion_convexa}}}\qquad
% 	\subfigure[Función convexa.]{\label{g:funcion_concava}
% 		\scalebox{1}{\input{img/funciones_elementales/funcion_concava}}}
% 	\caption{Concavidad de una función.}
% 	\label{g:concavidad}
% \end{figure}

% \subsection{Asíntotas}

% La recta $x=a$ es una \emph{asíntota vertical} de la función $f(x)$ si al menos uno de los límites laterales de $f(x)$ cuando $x$ tiende hacia $a$ es $+\infty$ o $-\infty$, es decir cuando se verifique alguna de las siguientes igualdades
% \[
% 	\ \lim_{x\rightarrow a^{+}}f(x)=\pm\infty   \quad \textrm{o} \quad
% 	\lim_{x\rightarrow a^{-}}f(x)=\pm\infty
% \]

% La recta $y=b$ es una \emph{asíntota horizontal} de la función $f(x)$ si alguno de los límites de $f(x)$ cuando $x$ tiende hacia $+\infty$ o $-\infty$ es igual a $b$, es decir cuando se verifique
% \[
% 	\ \lim_{x\rightarrow -\infty }f(x)=b    \quad \textrm{o} \quad
% 	\ \lim_{x\rightarrow +\infty }f(x)=b
% \]

% La recta $y=mx+n$ es una \emph{asíntota oblicua} de la función $f(x)$ si alguno de los límites de $f(x)-(mx+n)$ cuando $x$ tiende hacia $+\infty$ o $-\infty$ es igual a 0, es decir si

% \[
% 	\ \lim_{x\rightarrow -\infty }{(f(x)-mx)}=n    \quad \textrm{o} \quad
% 	\ \lim_{x\rightarrow +\infty }{(f(x)-mx)}=n
% \]

% En la figura~\ref{g:asintotas} se muestran los distintos tipos de asíntotas.

% \begin{figure}[h!]
% 	\centering \subfigure[Asíntota horizontal y vertical.] {\label{g:asintotahorizontalyvertical}
% 		\scalebox{1}{\input{img/funciones_elementales/asintota_vertical}}}\qquad\qquad
% 	\subfigure[Asíntota vertical y oblicua.]{\label{g:asintotaoblicua}
% 		\scalebox{1}{\input{img/funciones_elementales/asintota_oblicua}}}
% 	\caption{Tipos de asíntotas de una función.}
% 	\label{g:asintotas}
% \end{figure}


% \subsection{Periodicidad}
% Una función $f(x)$ es \emph{periódica} si existe $h\in\mathbb{R^{+}}$ tal que \[f(x+h)=f(x)\  \forall x\in \dom f\] siendo el período $T$ de la función, el menor valor $h$ que verifique la igualdad anterior.

% En una función periódica, por ejemplo $f(x)=A\sen(wt)$, se denomina \emph{amplitud} al valor de $A$, y es la mitad de la diferencia entre los valores máximos y mínimos de la función. En la figura~\ref{g:periodoyamplitud} se ilustran estos conceptos.

% \begin{figure}[h!]
% 	\centering
% 	\scalebox{0.8}{\input{img/funciones_elementales/funcion_periodica}}
% 	\caption{Periodo y amplitud de una función periódica.}
% 	\label{g:periodoyamplitud}
% \end{figure}

% \clearpage
% \newpage

\section{Ejercicios resueltos}

\begin{enumerate}[leftmargin=*]
\item Se considera la función
      \[
      f(t)=\frac{t^{4} +19\cdot t^{2} - 5}{t^{4} +9\cdot t^{2} - 10}.
      \]

      Representarla gráficamente y determinar a partir de dicha representación:

      \begin{enumerate}
      \item  Dominio.
            \begin{indication}
            \begin{enumerate}
            \item Para representarla gráficamente, introducir la función en la barra de \field{Entrada} de la \field{Vista CAS} y activar la \field{Vista Gráfica}.
            \item Para determinar el dominio tan sólo hay que determinar los valores de $x$ en los que existe la función.
            \item Recordar que, tanto para éste como para el resto de los apartados del ejercicio, pretendemos llegar a conclusiones aproximadas que tan sólo sacamos del análisis de la gráfica.
            \end{enumerate}
            \end{indication}

      \item  Imagen.
            \begin{indication}
            Fijarse en los valores de la variable $y$ hasta los que llega la función.
            \end{indication}

      \item  Asíntotas.
            \begin{indication}
            Son las líneas rectas, ya sea horizontales, verticales u oblicuas, hacia las que tiende la función.
            \end{indication}

      \item  Raíces.
            \begin{indication}
            Son los valores de la variable $x$, si los hay, en los que la función vale 0.
            \end{indication}

      \item Signo.
            \begin{indication}
            Hay que determinar, aproximadamente, por un lado los intervalos de variable $x$ en los que la función es positiva, y por el otro aquellos en
            los que es negativa.
            \end{indication}

      \item  Intervalos de crecimiento y decrecimiento.
            \begin{indication}
            De nuevo, por un lado hay que determinar los intervalos de variable $x$ en los que a medida que crece $x$ también lo hace $y$, que serían
            los intervalos de crecimiento, y también aquellos otros en los que a medida que crece $x$ decrece $y$, que serían los intervalos de
            decremimiento.
            \end{indication}

      \item Intervalos de concavidad y convexidad.
            \begin{indication}
            Para los intervalos de concavidad y convexidad, nos fijamos en el segmento de línea recta que une dos puntos cualquiera del intervalo. Si
            dicho segmento queda por encima de la gráfica, entonces la función es cóncava en el intervalo, mientras que si queda por debajo, entonces es
            convexa en el mismo.
            \end{indication}

      \item Extremos relativos.
            \begin{indication}
            Determinamos, aproximadamente, los puntos en los que se encuentran los máximos y mínimos relativos de la función.
            \end{indication}

      \item Puntos de inflexión.
            \begin{indication}
            Determinamos, aproximadamente, los puntos en los que la función cambia de curvatura, de cóncava a convexa o a la inversa.
            \end{indication}
      \end{enumerate}

\item Representar en una misma gráfica las funciones $2^{x}, e^{x}, 0.7^{x}, 0.5^{x}$. A la vista de las gráficas obtenidas, indicar cuáles
      de las funciones anteriores son crecientes y cuáles son decrecientes.
      \begin{indication}
      Introducir cada función en la barra de \field{Entrada} de la \field{Vista CAS}.
      \end{indication}

      ¿En general, para qué valores de $a$ será la función creciente? ¿Y para qué valores de $a$ será decreciente? Probar con
      distintos valores de $a$ representando gráficamente nuevas funciones si fuera necesario.


\item Representar en una misma gráfica las funciones siguientes, indicando su período y amplitud.
      \begin{enumerate}
      \item $\sen{x}$, $\sen{x}+2$, $\sen{(x+2)}$.
      \item $\sen{2x}$, $2\sen{x}$, $\sen\frac{x}{2}$.
            \begin{indication}
            Introducir cada función en la barra de \field{Entrada} de la \field{Vista CAS}.
            \end{indication}
      \end{enumerate}


\item Representar en una gráfica la función
      \[
      \ f(x)=\left\{
      \begin{array}{cl}
      -2x   & \hbox{si $x\leq0$;} \\
      x^{2} & \hbox{si $x>0$.}    \\
      \end{array}
      \right.
      \]

      \begin{indication}
      Para representar funciones a trozos, Geogebra utiliza el comando
      \begin{center}
            \command{Si(<Condición>, <Entonces>, <Si no>)} 
      \end{center}
      y se pueden anidar varios comandos unos dentro de otros. 
      Utilizando este comando para representar la función anterior, habría que introducir la expresión
      \begin{center}
        Si[x<=0, -2x, x\^2]    
      \end{center}
      \end{indication}
\end{enumerate}


\section{Ejercicios propuestos}
\begin{enumerate}[leftmargin=*]
\item Hallar el dominio de las siguientes funciones a partir de sus representaciones gráficas:

      \begin{enumerate}
      \item $f(x)=\dfrac{x^{2} + x + 1}{x^{3} - x}$
      \item $g(x)=\sqrt[2]{x^{4}-1}$.
      \item $h(x)=\cos{\dfrac{x + 3}{x^{2} + 1}}$.
      \item $l(x)=\arcsen{\dfrac{x}{1+x}}$.
      \end{enumerate}

\item Se considera la función
      \[
      \ f(x)=\frac{x^{3} + x +2}{5x^{3} - 9x^{2} - 4x + 4}.
      \]

      Representarla gráficamente y determinar a partir de dicha representación:

      \begin{enumerate}
      \item Dominio.
      \item Imagen.
      \item Asíntotas.
      \item Raíces.
      \item Signo.
      \item Intervalos de crecimiento y decrecimiento.
      \item Intervalos de concavidad y convexidad.
      \item Extremos relativos.
      \item Puntos de inflexión.
      \end{enumerate}

\item Representar en una misma gráfica las funciones $\log_{10}{x}$, $\log_{2}{x}$, $\log{x}$, $\log_{0.5}{x}$.
      \begin{enumerate}
      \item A la vista de las gráficas obtenidas, indicar cuáles de las funciones anteriores son crecientes y cuáles son decrecientes.
      \item Determinar, a partir de los resultados obtenidos, o representando nuevas funciones si fuera necesario, para qué valores de $a$ será
            creciente la función $\log_{a}{x}$.
      \item Determinar, a partir de los resultados obtenidos, o representando nuevas funciones si fuera necesario, para qué valores de $a$ será
            decreciente la función $\log_{a}{x}$.
      \end{enumerate}

\item Completar las siguientes frases con la palabra igual, o el número de veces que sea mayor o menor en cada caso:
      \begin{enumerate}
      \item La función $\cos{2x}$ tiene un período............ que la función $\cos{x}$.
      \item La función $\cos{2x}$ tiene una amplitud............ que la función $\cos{x}$.
      \item La función $\cos\dfrac{x}{2}$ tiene un período............ que la función $\cos{3x}$.
      \item La función $\cos\dfrac{x}{2}$ tiene una amplitud............ que la función $\cos{3x}$.
      \item La función $3\cos{2x}$ tiene un período............ que la función $\cos\dfrac{x}{2}$.
      \item La función $3\cos{2x}$ tiene una amplitud............ que la función $\cos\dfrac{x}{2}$.
      \end{enumerate}

\item Hallar a partir de la representación gráfica, las soluciones de $e^{-1/x}=\dfrac{1}{x}$.

\item Representar en una gráfica la función
      \[
      \ f(x)=\left\{
      \begin{array}{ll}
      x^{3}   & \hbox{si $x<0$}    \\
      e^{x}-1 & \hbox{si $x\geq0$} \\
      \end{array}
      \right.
      \]

\end{enumerate}



% \[
% \mathop {\lim }\limits_{x \to a} f(x) =  \pm \infty
% \]

% \subsubsection*{Asíntotas Horizontales}
% La recta $y=b$ es una \emph{Asíntota Horizontal} de la función
% $f(x)$ si se cumple:
% \[
% \mathop {\lim }\limits_{x \to  + \infty } f(x) = b\quad
% \text{ó}\quad\mathop {\lim }\limits_{x \to  - \infty } f(x) = b
% \]


% \subsubsection*{Asíntotas Oblicuas}

% La recta $y=mx+n$, donde $m\neq0$, es \emph{Asíntota Oblicua} de la
% función $f(x)$ si:


% \[% !TEX root = ../practicas_geogebra.tex

% \m% !TEX root = ../practicas_geogebra.tex
% limits_{x \to  + \infty } \left[ {f(x) - \left( {mx
% + % !TEX root = ../practicas_geogebra.tex
% ight] = 0\quad\text{ó}\quad\mathop {\lim }\limits_{x
% \t% !TEX root = ../practicas_geogebra.tex
% left[ {f(x) - \left( {mx + n} \right)} \right] = 0
% \]% !TEX root = ../practicas_geogebra.tex



% La% !TEX root = ../practicas_geogebra.tex
%  práctica de $m$ y $n$ se realiza del siguiente
% mo% !TEX root = ../practicas_geogebra.tex


% \[
% m = \mathop {\lim }\limits_{x \to  + \infty } \frac{{f(x)}} {x}
% \]

% \[
% n = \mathop {\lim }\limits_{x \to  + \infty } \left[ {f(x) - mx}
% \right]
% \]
% o bien lo mismo con los límites en $-\infty$:
% \[
% m = \mathop {\lim }\limits_{x \to  - \infty } \frac{{f(x)}} {x}
% \]

% \[
% n = \mathop {\lim }\limits_{x \to  - \infty } \left[ {f(x) - mx}
% \right]
% \]

% En cualquiera de los casos, si obtenemos un valor real para $m$ (no
% puede ser ni $+\infty$ ni $-\infty$) distinto de $0$, procedemos
% después a calcular $n$, que también debe ser real (sí que puede ser
% $0$).

% Si $m=\pm\infty$ entonces la función crece (decrece) más deprisa que
% cualquier recta, y si $m=0$ la función crece (decrece) más despacio
% que cualquier recta, y en cualquiera de los dos casos decimos que la
% función tiene una \emph{Rama Parabólica}.

% \subsection{Continuidad de una función en un punto}
% Diremos que una función $f(x)$ es continua en un punto $a\in
% \mathbb{R}$, si se cumple
% \[ \lim_{x\rightarrow a}f(x)=f(a),\]
% donde $f(a)\in \mathbb{R}$.

% La definición anterior implica a su vez que se cumplan estas tres
% condiciones:

% \begin{itemize}

% \item Existe el límite de $f$ en $x=a$.

% \item La función está definida en $x=a$; es decir, existe $f(a)$.

% \item Los dos valores anteriores coinciden.

% \end{itemize}

% Si la función $f$ no es continua en $x=a$, diremos que es
% \emph{discontinua} en el punto $a$, o bien que $f$ tiene una
% \emph{discontinuidad} en $a$.

% Intuitivamente, una función es continua cuando puede dibujarse su
% gráfica sin levantar el lápiz.

% \subsubsection*{Continuidad lateral en un punto}

% Si nos restringimos a los valores que toma una función a la derecha
% de un punto $x=a$, o a la izquierda, se habla de continuidad por la
% derecha o por la izquierda según la siguiente definición.

% Una función es \emph{continua por la derecha} en un punto $x=a$, y
% lo notaremos como $f$ continua en $a^+$, si existe el límite por la
% derecha en dicho punto y coincide con el valor de la función en el
% mismo:
% \[
% \mathop {\lim }\limits_{x \to a^ +  } f\left( x \right) = f\left( a
% \right)
% \]

% De igual manera, la función es \emph{continua por la izquierda} en
% un punto $x=a$, y lo notaremos como $f$ continua en $a^-$, si existe
% el límite por la izquierda en dicho punto y coincide con el valor de
% la función en el mismo:

% \[
% \mathop {\lim }\limits_{x \to a^ -  } f\left( x \right) = f\left( a
% \right)
% \]


% \subsubsection*{Propiedades de la continuidad en un punto}

% Como consecuencia de la definición de continuidad en un punto,
% podrían demostrarse toda una serie de teoremas, algunos de ellos
% especialmente importantes.

% \begin{itemize}

% \item \textbf{Álgebra de funciones continuas}.
%       Si $f$ y $g$ son funciones continuas en $x=a$, entonces $f\pm g$ y
%       $f\cdot g$ son también continuas en $x=a$. Si además $g(a)\neq 0$,
%       entonces $f/g$ también es continua en $x=a$.

% \item \textbf{Continuidad de funciones compuestas}. Si $f$ es continua en
%       $x=a$ y $g$ es continua en $b=f(a)$, entonces la función compuesta
%       $g\circ f$ es continua en $x=a$.

% \item \textbf{Continuidad y cálculo de límites}. Sean $f$ y $g$ dos
%       funciones tales que existe $\mathop {\lim }\limits_{x \to a} f(x) =
%       l$ $\in \mathbb{R}$ y $g$ es una función continua en $l$. Entonces:

%       \[
%       \mathop {\lim }\limits_{x \to a} g\left( {f\left( x \right)} \right)
%       = g\left( l \right)
%       \]

% \end{itemize}

% \subsubsection*{Tipos de discontinuidades}
% Puesto que la condición de continuidad puede no satisfacerse por
% distintos motivos, existen distintos tipos de discontinuidades:


% \begin{itemize}
% \item \textbf{Discontinuidad evitable}. Se dice que $f(x)$ tiene una \emph{discontinuidad evitable} en el punto $a$, si existe el límite de la función  pero no coincide con el valor de la función en el punto (bien porque sea diferente, bien por que la función no esté definida en dicho punto), es decir
%       \[\lim_{x\rightarrow a}f(x)=l\neq f(a).\]

% \item \textbf{Discontinuidad de salto}. Se dice que $f(x)$ tiene una \emph{discontinuidad de salto} en el punto $a$, si existe el límite de la función por la izquierda  y por la derecha pero son diferentes, es decir,
%       \[
%       \lim_{x\rightarrow a^-}f(x)=l_1\neq l_2=\lim_{x\rightarrow a^+}f(x).
%       \]
%       A la diferencia entre ambos límites $l_1-l_2$, se le llama
%       \emph{amplitud del salto}.

% \item \textbf{Discontinuidad esencial}. Se dice que $f(x)$ tiene una \emph{discontinuidad esencial} en el punto $a$, si no existe alguno de los límites laterales de la función.
% \end{itemize}

% \newpage

\section{Ejercicios resueltos}
\begin{enumerate}[leftmargin=*]
\item Dada la función
      \[
      f(x)=\left( 1+\frac{2}{x}\right) ^{x/2},
      \]
      se pide:

      \begin{enumerate}
      \item Dibujar su gráfica, y a la vista de misma conjeturar el resultado de los siguientes límites:
            \begin{multicols}{2}
            \begin{enumerate}
            \item  $\lim\limits_{x\rightarrow -\,2^{-}}\ f(x)$
            \item  $\lim\limits_{x\rightarrow -\,2^{+}}\ f(x)$
            \item  $\lim\limits_{x\rightarrow -\,\infty }\ f(x)$
            \item  $\lim\limits_{x\rightarrow +\,\infty }\ f(x)$
            \item  $\lim\limits_{x\rightarrow 2}\ f(x)$
            \item  $\lim\limits_{x\rightarrow 0}\ f(x)$
            \end{enumerate}
            \end{multicols}

            \begin{indication}
            \begin{enumerate}
            \item Para representarla gráficamente, introducir la función \command{f(x)=(1+2/x)\^{}(x/2)} en la barra de \field{Entrada} de la \field{Vista Algebraica} y activar la \field{Vista Gráfica}.
            \item Para predecir cuáles pueden ser los valores de los límites pedidos, crear un deslizador introduciendo la constante \command{a} en la barra de \field{Entrada} y después introducir el punto \command{A=(a,f(a))} para dibujar el punto sobre la gráfica de la función. Desplazar el deslizador y observar el valor de la coordenada $y$ en el punto $A$ cuando $x$ tiende a cada uno de los valores de los límites.
            \end{enumerate}
            \end{indication}

      \item Calcular los límites anteriores. ¿Coinciden los resultados con los conjeturados?.
            \begin{indication}
            \begin{enumerate}
            \item Para calcular el límite por la izquierda introducir el comando \command{LímiteIzquierda(<funcion>,<valor>)} en la barra de \field{Entrada}.
            \item Para calcular el límite por la derecha introducir el comando \command{LímiteDerecha(<funcion>,<valor>)} en la barra de \field{Entrada}.
            \item Para calcular el límite global introducir el comando \command{Límite(<funcion>,<valor>)}en la barra de \field{Entrada}.
            \end{enumerate}
            \end{indication}
      \end{enumerate}


\item Dada la función
      \[
      g(x)=
      \begin{cases}
      \dfrac{x}{x-2}    & \mbox{si $x\leq 0$;} \\
      \dfrac{x^2}{2x-6} & \mbox{si $x>0$;}
      \end{cases}
      \]
      \begin{enumerate}
      \item Dibujar la gráfica de $g$ y determinar gráficamente si existen asíntotas.
            \begin{indication}
            \begin{enumerate}
            \item Para representarla gráficamente, introducir la función \command{g(x)=Si(x<=0, x/(x-2), x\^{}2/(2x-6))} en la barra de \field{Entrada} de la \field{Vista Algebraica} y activar la \field{Vista Gráfica}.
            \item Para ver si existen asíntotas verticales, horizontales y oblicuas, hay que ver si existen rectas verticales, horizontales su oblicuas a las que la gráfica de $g$ se aproxima cada vez más (aunque nunca lleguen a tocarse).
            \end{enumerate}
            \end{indication}

      \item Calcular las asíntotas verticales de $g$ y dibujarlas.
            \begin{indication}
            \begin{enumerate}
            \item El único punto donde la función no está definida es $x=3$.
                  Para ver si existe asíntota vertical en ese punto hay que calcular los límites laterales $\lim_{x\rightarrow 3^-}g(x)$ y $\lim_{x\rightarrow 3^+}g(x)$.
            \item Para calcular el límite por la izquierda introducir el comando \command{LímiteIzquierda(g, 3)} en la barra de \field{Entrada}.
            \item Para calcular el límite por la derecha introducir el comando \command{LímiteDerecha(g, 3)} en la barra de \field{Entrada}.
            \item Como $\lim_{x\rightarrow 3^-}g(x)=-\infty$ y $\lim_{x\rightarrow 3^+}g(x)=\infty$, existe una asíntota vertical en $x=3$.
                  Para dibujarla introducir la expresión \command{x=3} en la barra de \field{Entrada}.
            \end{enumerate}
            \end{indication}

      \item Calcular las asíntotas horizontales de $g$ y dibujarlas.
            \begin{indication}
            \begin{enumerate}
            \item Para ver si existen asíntotas horizontales hay que calcular los límites $\lim_{x\rightarrow -\infty} g(x)$ y $\lim_{x\rightarrow \infty} g(x)$.
            \item Para calcular el límite en $-\infty$ introducir el comando \command{Límite(g, -inf)} en la barra de \field{Entrada}.
            \item Para calcular el límite en $\infty$ introducir el comando \command{Límite(g, inf)} en la barra de \field{Entrada}.
            \item Como $\lim_{x\rightarrow -\infty} g(x)=1$, existe una asíntota horizontal $y=1$.
                  Para dibujarla introducir la expresión \command{x=3} en la barra de \field{Entrada}.
            \item Como $\lim_{x\rightarrow \infty} g(x)=\infty$, no hay asíntota horizontal por la derecha.
            \end{enumerate}
            \end{indication}

      \item Calcular las asíntotas oblicuas de $g$.
            \begin{indication}
            \begin{enumerate}
            \item Por la izquierda no hay asíntota oblicua puesto que hay una asíntota horizontal.
                  Para ver si existen asíntota oblicua por la derecha hay que calcular $\lim_{x\rightarrow \infty}\frac{g(x)}{x}$.
                  Para ello introducir el comando \command{Límite(g/x, inf)} en la barra de \field{Entrada}.
            \item Como $\lim_{x\rightarrow \infty}\frac{g(x)}{x}=0.5$, existe asíntota oblicua por la derecha y su pendiente es $0.5$.
            \item Para determinar el término independiente hay que calcular el límite $\lim_{x\rightarrow \infty}g(x)-0.5x$.
                  Para ello introducir el comando \command{Límite(g-0.5x, inf)} en la barra de \field{Entrada}.
            \item Como $\lim_{x\rightarrow \infty}g(x)-0.5x=1.5$ entonces la ecuación de la asíntota oblicua es $y=0.5x+1.5$.
                  Para dibujarla introducir la expresión \command{y=0.5x+1.5} en la barra de \field{Entrada}.
            \end{enumerate}
            \end{indication}
      \end{enumerate}

\item Reprentar gráficamente las siguientes funciones y clasificar sus discontinuidades en los puntos que se indica.
      \begin{enumerate}
      \item  $f(x)=\dfrac{\sen x}{x}$ en $x=0$.

            \begin{indication}
            \begin{enumerate}
            \item Para representarla gráficamente, introducir la función \command{f(x)=sen(x)/x} en la barra de \field{Entrada} de la \field{Vista Algebraica} y activar la \field{Vista Gráfica}.
            \item Para calcular el límite $\lim_{x\rightarrow 0^-}f(x)$ introducir el comando \command{LímiteIzquierda(f, 0)} en la barra de \field{Entrada}.
            \item Para calcular el límite $\lim_{x\rightarrow 0^+}f(x)$ introducir el comando \command{LímiteDerecha(f, 0)} en la barra de \field{Entrada}.
            \item Como $\lim_{x\rightarrow 0^-}f(x)=\lim_{x\rightarrow 0^+}f(x)=1$, $f$ tiene una discontinuidad evitable en $x=0$.
            \end{enumerate}
            \end{indication}
      \item $g(x)=\dfrac{1}{2^{1/x}}$ en $x=0$.
            \begin{indication}
            \begin{enumerate}
            \item Para representarla gráficamente, introducir la función \command{f(x)=sen(x)/x} en la barra de \field{Entrada} de la \field{Vista Algebraica}.
            \item Para calcular el límite $\lim_{x\rightarrow 0^-}g(x)$ introducir el comando \command{LímiteIzquierda(g, 0)} en la barra de \field{Entrada}.
            \item Para calcular el límite $\lim_{x\rightarrow 0^+}g(x)$ introducir el comando \command{LímiteDerecha(g, 0)} en la barra de \field{Entrada}.
            \item Como $\lim_{x\rightarrow 0^-}g(x)=\infty$, $g$ tiene una discontinuidad evitable en $x=0$.
            \end{enumerate}
            \end{indication}
      \item $h(x)=\dfrac{1}{1+e^{\frac{1}{1-x}}}$ en $x=1$.
            \begin{indication}
            \begin{enumerate}
            \item Para representarla gráficamente, introducir la función \command{f(x)=sen(x)/x} en la barra de \field{Entrada} de la \field{Vista Algebraica}.
            \item Para calcular el límite $\lim_{x\rightarrow 1^-}h(x)$ introducir el comando \command{LímiteIzquierda(f, 0)} en la barra de \field{Entrada}.
            \item Para calcular el límite $\lim_{x\rightarrow 1^+}h(x)$ introducir el comando \command{LímiteDerecha(f, 0)} en la barra de \field{Entrada}.
            \item Como $\lim_{x\rightarrow 1^-}h(x)=0$ y $\lim_{x\rightarrow 1^+}f(x)=1$, $h$ tiene una discontinuidad de salto en $x=1$.
            \end{enumerate}
            \end{indication}
      \end{enumerate}


\item  Representar gráficamente y clasificar las discontinuidades de la función
      \[
      f(x)=
      \left\{
      \begin{array}{ll}
      \dfrac{x+1}{x^2-1},       & \hbox{si $x<0$;}     \\
      \dfrac{1}{e^{1/(x^2-1)}}, & \hbox{si $x\geq 0$.} \\
      \end{array}
      \right.
      \]

      \begin{indication}
      \begin{enumerate}
      \item Para representarla gráficamente, introducir la función \command{f(x)=Si(x<0, (x+1)/(x\^{}2-1), 1/e\^{}1/(x\^{}2-1))} en la barra de \field{Entrada} de la \field{Vista Algebraica} y activar la \field{Vista Gráfica}.
      \item En primer lugar hay que encontrar los puntos que quedan fuera del dominio de cada uno de los tramos.
            Para ello, hay que analizar dónde se anulan los denominadores presentes en las definiciones de cada trozo.
            Para ver donde se anula $x^2-1$ introducir la ecuación $x^1-1=0$ en la barra de \field{Entrada} y hacer clic sobre el botón \button{Resolver}.
      \item Como la ecuación anterior tiene soluciones $x=-1$ y $x=1$, en estos puntos la función no está definida y es, por tanto, discontinua.
            Además de estos dos puntos hay que estudiar la posible discontinuidad en $x=0$ que es donde cambia la definición de la función.
      \item Para calcular el límite $\lim_{x\rightarrow -1^-}f(x)$ introducir el comando \command{LímiteIzquierda(f, -1)} en la barra de \field{Entrada}.
      \item Para calcular el límite $\lim_{x\rightarrow -1^+}f(x)$ introducir el comando \command{LímiteDerecha(f, -1)} en la barra de \field{Entrada}.
      \item Como $\lim_{x\rightarrow -1^-}f(x)=\lim_{x\rightarrow -1^+}f(x)=-0.5$, $f$ tiene una discontinuidad evitable en $x=-1$.
      \item Para calcular el límite $\lim_{x\rightarrow 0^-}f(x)$ introducir el comando \command{LímiteIzquierda(f, 0)} en la barra de \field{Entrada}.
      \item Para calcular el límite $\lim_{x\rightarrow 0^+}f(x)$ introducir el comando \command{LímiteDerecha(f, 0)} en la barra de \field{Entrada}.
      \item Como $\lim_{x\rightarrow 0^-}f(x)=-1$ y $\lim_{x\rightarrow 0^+}f(x)=e$, $f$ tiene una discontinuidad de salto en $x=0$.
      \item Para calcular el límite $\lim_{x\rightarrow 1^-}f(x)$ introducir el comando \command{LímiteIzquierda(f, 1)} en la barra de \field{Entrada}.
      \item Para calcular el límite $\lim_{x\rightarrow 1^+}f(x)$ introducir el comando \command{LímiteDerecha(f, 1)} en la barra de \field{Entrada}.
      \item Como $\lim_{x\rightarrow 1^-}f(x)=\infty$, $f$ tiene una discontinuidad esencial en $x=0$.
      \end{enumerate}
      \end{indication}
\end{enumerate}


\section{Ejercicios propuestos}
\begin{enumerate}[leftmargin=*]
\item  Calcular los siguientes límites si existen:
      \begin{multicols}{2}
      \begin{enumerate}
      \item  $\displaystyle \lim_{x\rightarrow 1}\dfrac{x^3-3x+2}{x^4-4x+3}$.
      \item  $\displaystyle \lim_{x\rightarrow a}\dfrac{\sen x-\sen a}{x-a}$.
      \item $\displaystyle \lim_{x\rightarrow\infty}\dfrac{x^2-3x+2}{e^{2x}}$.
      \item $\displaystyle \lim_{x\rightarrow\infty}\dfrac{\log(x^2-1)}{x+2}$.
      \item $\displaystyle \lim_{x\rightarrow 1}\dfrac{\log(1/x)}{\tg(x+\dfrac{\pi}{2})}$.
      \item $\displaystyle \lim_{x\rightarrow a}\dfrac{x^n-a^n}{x-a}\quad n\in \mathbb{N}$.
      \item $ \displaystyle \lim_{x\rightarrow 1}\dfrac{\sqrt[n]{x}-1}{\sqrt[m]{x}-1}\quad n,m \in \mathbb{Z}$.
      \item $\displaystyle \lim_{x\rightarrow 0}\dfrac{\tg x-\sen x}{x^3}$.
      \item $\displaystyle \lim_{x\rightarrow \pi/4}\dfrac{\sen x-\cos x}{1-\tg x}$.
      \item $\displaystyle \lim_{x\rightarrow 0}x^2e^{1/x^2}$.
      \item $\displaystyle \lim_{x\rightarrow \infty}\left(1+\dfrac{a}{x}\right)^x$.
      \item $\displaystyle \lim_{x\rightarrow \infty} \sqrt[x]{x^2}$.
      \item $\displaystyle \lim_{x\rightarrow 0}\left(\dfrac{1}{x}\right)^{\tg x}$.
      \item $\displaystyle \lim_{x\rightarrow 0}(\cos x)^{1/\mbox{\footnotesize sen}\, x}$.
      \item $\displaystyle \lim_{x\rightarrow 0}\dfrac{6}{4+e^{-1/x}}$.
      \item $\displaystyle \lim_{x\rightarrow \infty}\left(\sqrt{x^2+x+1}-\sqrt{x^2-2x-1}\right)$.
      \item $\displaystyle \lim_{x\rightarrow \pi/2}\sec x-\tg x$.
      \end{enumerate}
      \end{multicols}

\item Dada la función:
      \[
      \renewcommand{\arraystretch}{2}
      f(x) = \left\{
      \begin{array}{*{20}c}{
      \dfrac{{x^2  + 1}}{{x + 3}}}     & {\text{si}\;x < 0}           \\
      {\dfrac{1}{{e^{1/(x^2  - 1)} }}} & {\text{si}\;x \geqslant 0\;} \\
      \end{array}
      \right.
      \]
      Calcular todas sus asíntotas.

\item  Las siguientes funciones no están definidas en $x=0$.
      Determinar, cuando sea posible, su valor en dicho punto de modo que sean continuas.
      \begin{multicols}{2}
      \begin{enumerate}
      \item  $f(x)=\dfrac{(1+x)^n-1}{x}$.
      \item  $h(x)=\dfrac{e^x-e^{-x}}{x}$.
      \item  $j(x)=\dfrac{\log(1+x)-\log(1-x)}{x}$.
      \item  $k(x)=x^2\sen\dfrac{1}{x}$.
      \end{enumerate}
      \end{multicols}

\end{enumerate}
