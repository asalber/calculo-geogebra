%$HeadURL: https://practicas-derive.googlecode.com/svn/trunk/integral_riemann.tex $
%$LastChangedDate: 2009-11-16 16:11:46 +0100 (lun, 16 nov 2009) $
%$LastChangedRevision: 5 $
%$LastChangedBy: asalber $

\chapter[La integral de Riemman: Cálculo de áreas y volúmenes]{La integral de Riemann:\\ Cálculo de áreas y volúmenes}

\section{Fundamentos teóricos}
Junto al concepto de derivada, el de integral es otro de los más
importantes del cálculo matemático. Aunque dicho concepto surge, en
principio, como técnica para el cálculo de áreas, el teorema
fundamental del cálculo establece su relación con la derivada, de
manera que, en cierto sentido, la diferenciación y la integración
son operaciones inversas.

En esta práctica se estudia la integral de Riemann, que permite
calcular áreas por debajo de funciones acotadas en un intervalo.

\subsection*{Integral de Riemann}

Se llama \emph{partición} de un intervalo $[a,b]\subset\mathbb{R}$,
a una colección finita de puntos del intervalo,
$P=\{x_{0},x_{1},...,x_{n}\}$,  tales que
$x_{0}=a<x_{1}<...<x_{n}=b$, con lo que el intervalo $[a,b]$ queda
dividido en $n$ subintervalos $[x_{i},x_{i+1}]$, $i=0,...,n-1$.

Dada una función $f:[a,b]\rightarrow\mathbb{R}$ acotada y una
partición $P=\{x_{0},x_{1},...,x_{n}\}$ de $[a,b]$, se llama
\emph{suma inferior} de $f$ en relación a $P$, y se designa por
$L(P,f)$, a:

\[
\ L(P,f)=\sum_{i=1}^{n} m_{i}(x_{i}-x_{i-1})
\]
donde $  m_{i}=\inf\{f(x):x_{i-1}\leq x \leq x_{i}\}$.

Análogamente se llama \emph{suma superior} de $f$ en relación a $P$,
y se designa por $U(P,f)$, a:

\[
\ U(P,f)=\sum_{i=1}^{n} M_{i}(x_{i}-x_{i-1})
\]
donde $ M_{i}=\sup\{f(x):x_{i-1}\leq x \leq x_{i}\}$.

La \emph{suma inferior} y la \emph{suma superior} así definidas representan las sumas de las áreas de los rectángulos que tienen por bases los subintervalos de la partición, y por alturas los valores mínimo y máximo respectivamente de la función $f$ en los subintervalos considerados, tal y como se muestra en la figura~\ref{g:sumassupinf}. Así, los valores de $L(P,f)$ y $U(P,f)$ serán siempre menores y mayores respectivamente, que el área encerrada por la función $f$ y el eje de abscisas en el intervalo $[a,b]$.

\begin{figure}[htbp]
\centering \subfigure[Suma inferior $L(P,f)$.]{
\label{g:sumainferior}
\scalebox{1}{\input{img/integrales/suma_inferior}}}\qquad\qquad
\subfigure[Suma superior$U(P,f)$.]{
\label{g:sumasuperior}
\scalebox{1}{\input{img/integrales/suma_superior}}}
\caption{Áreas medidas por las sumas superior e inferior
correspondientes a una partición.} \label{g:sumassupinf}
\end{figure}

Una función $f:[a,b]\rightarrow\mathbb{R}$ acotada es
\emph{integrable} en el intervalo $[a,b]$ si se verifica que:

\[
\ \sup\{L(P,f): P \textrm{ partición de } [a,b]\}=\inf\{U(P,f): P \textrm{
partición de }[a,b]\}
\]
y ese número se designa por $\int_{a}^{b}f(x)\,dx$ o simplemente por
$\int_{a}^{b}f$.


\subsubsection*{Propiedades de la Integral}

\begin{enumerate}

\item \textbf{Linealidad}

Dadas dos funciones $f$ y $g$ integrables en $[a,b]$ y $k \in
\mathbb{R}$ se verifica que:

\[
\ \int_{a}^{b}(f(x)+g(x))\,dx=\int_{a}^{b}f(x)\,dx+\int_{a}^{b}g(x)\,dx
\]
y
\[
\ \int_{a}^{b}{kf(x)}\,dx=k\int_{a}^{b}{f(x)}\,dx
\]

\item \textbf{Monotonía}

Dadas dos funciones $f$ y $g$ integrables en $[a,b]$ y tales que
$f(x)\leq g(x)$ $\forall x \in [a,b]$ se verifica que:


\[
\ \int_{a}^{b}{f(x)\,dx} \leq \int_{a}^{b}{g(x)\,dx}
\]

\item \textbf{Acotación}

Si $f$ es una función integrable en el intervalo $[a,b]$, existen
$m,M\in\mathbb{R}$ tales que:

\[
\ m(b-a)\leq\int_{a}^{b}{f(x)\,dx} \leq \ M(b-a)
\]

\item \textbf{Aditividad}

Si $f$ es una función acotada en $[a,b]$ y $c\in(a,b)$, entonces $f$
es integrable en $[a,b]$ si y sólo si lo es en $[a,c]$ y en $[c,b]$,
verificándose además:

\[
\ \int_{a}^{b}{f(x)\,dx} = \int_{a}^{c}{f(x)\,dx}+\int_{c}^{b}{f(x)\,dx}
\]\\

\end{enumerate}

\subsubsection*{Teorema Fundamental del Cálculo}

Sea $f : [a,b]\rightarrow\mathbb{R}$ continua y sea $G$ una función
continua en $[a,b]$. Entonces $G$ es derivable en $(a,b)$ y
$G'(x)=f(x)$ para todo $x\in(a,b)$ si y sólo si:

\[
\ G(x)-G(a) = \int_{a}^{x}f(t)\,dt
\]

\subsubsection*{Regla de Barrow}

Si $f$ es una función continua en $[a,b]$ y $G$ es continua en
$[a,b]$, derivable en $(a,b)$ y tal que $G'(x)=f(x)$ para todo
$x\in(a,b)$ entonces:

\[
\  \int_{a}^{b}{f} = G(b)-G(a)
\]


De aquí se deduce que:

\[
\  \int_{a}^{b}{f} = -\int_{b}^{a}{f}
\]


\subsection*{Integrales impropias}

La integral $ \int_{a}^{b}{f(x)\,dx}$ se llama \emph{impropia} si el intervalo
$(a,b)$ no está acotado o si la función $f(x)$ no está acotada en el intervalo
$(a,b)$.

Si el intervalo $(a,b)$ no está acotado, se denomina integral impropia de primera especie mientras que si la función no está acotada en el intervalo se denomina integral impropia de segunda especie.

\subsection*{Cálculo de áreas}
Una de las principales aplicaciones de la integral es el cálculo de áreas.

\subsubsection*{Area de una región plana encerrada por dos curvas}

Si $f$ y $g$ son dos funciones integrables en el intervalo $[a,b]$ y
se verifica que $g(x)\leq f(x)$ $\forall x\in[a,b]$, entonces el
área de la región plana limitada por las curvas $y=f(x)$, $y=g(x)$,
y las rectas $x=a$ y $x=b$ viene dada por:

\[
\ A = \int_{a}^{b}{(f(x)- g(x))\,dx}
\]\\

\noindent \textbf{Observaciones}

\begin{enumerate}

\item El intervalo $(a,b)$ puede ser infinito y la definición sería análoga, pero en ese caso es preciso que la integral impropia sea convergente.

\item Si $f(x)\geq0$ y $g(x)=0$ al calcular la integral entre $a$ y $b$ se obtiene el área encerrada por la función $f(x)$ y el eje de abscisas entre las rectas verticales $x=a$ y $x=b$ (figura~\ref{g:integraldefinida}).

\begin{figure}[h!]
\begin{center}
\scalebox{1}{\input{img/integrales/integral_definida}}
\caption{Cálculo de área encerrada por la función $f(x)$ y el eje de
abscisas entre las rectas verticales $x=a$ y $x=b$  mediante la
integral definida.} \label{g:integral_definida}
\end{center}
\end{figure}

\item Si $f(x)\geq 0$ $\forall x\in[a,c]$ y $f(x)\leq 0$ $\forall x\in[c,b]$, el área de la región plana encerrada por $f$, las rectas verticales $x=a$ y $x=b$ y el eje de abscisas se calcula
mediante:
\[
\ A= \int_{a}^{c}{f(x)\,dx} - \int_{c}^{b}{f(x)\,dx}.
\]

\item Si las curvas $y=f(x)$ e $y=g(x)$ se cortan en los puntos de abscisas $a$ y $b$, no cortándose en ningún otro punto cuya abscisa esté comprendida entre $a$ y $b$, el área encerrada por dichas curvas entre esos puntos de corte puede calcularse
mediante:
\[
\ A= \int_{a}^{b}{|f(x)-g(x)|dx}
\]
\end{enumerate}


\subsection*{Cálculo de Volúmenes}

\subsubsection*{Volumen de un sólido}
Si se considera un cuerpo que al ser cortado por un plano
perpendicular al eje $OX$ da lugar, en cada punto de abscisa $x$, a
una sección de área $A(x)$, el volumen de dicho cuerpo comprendido
entre los planos perpendiculares al eje $OX$ en los puntos de
abscisas $a$ y $b$ es:

\[
\ V = \int_{a}^{b}{A(x)\,dx}
\]

\subsubsection*{Volumen de un cuerpo de revolución}
Si se hace girar la curva $y=f(x)$ alrededor del eje $OX$ se genera un sólido
de revolución cuyas secciones perpendiculares al eje $OX$ tienen áreas
$A(x)=\pi(f(x))^{2}$, y cuyo volumen comprendido entre las abscisas $a$ y $b$
será:

\[
\ V = \int_{a}^{b}{\pi(f(x))^{2}\,dx}=
\pi\int_{a}^{b}{(f(x))^{2}\,dx}
\]


En general, el volumen del cuerpo de revolución engendrado al girar
alrededor del eje $OX$ la región plana limitada por las curvas
$y=f(x)$, $y=g(x)$ y las rectas verticales $x=a$ y $x=b$ es:

\[
\ V = \int_{a}^{b}{\pi|(f(x))^{2}-(g(x))^{2}|\,dx}
\]

De manera análoga se calcula el volumen del cuerpo de revolución
engendrado por la rotación de una curva $x=f(y)$ alrededor del eje
$OY$, entre los planos $y=a$ e $y=b$, mediante:

\[
\ V = \int_{a}^{b}{\pi(f(y))^{2}dy} = \pi \int_{a}^{b}{(f(y))^{2}dy}
\]

\newpage

\section{Ejercicios resueltos}
\begin{enumerate}[leftmargin=*]
\item Calcular las siguientes integrales:
\begin{enumerate}
\item $ \dint^{0}_{-\frac{1}{2}}{\frac{x^{3}}{x^{2}+x+1}}\,dx$
\begin{indication}
{Introducir:
\[
\frac{x^{3}}{x^{2}+x+1}
\]
en la línea de edición. Utilizar el menú \menu{Cálculo->Integrales}
y elegir \option{Integral Definida}. Introducir $-\frac{1}{2}$ en
\option{Límite Inferior} y 0 en \option{Límite Superior}, y pinchar
en \button{Simplificar}. }
\end{indication}

\item $ \dint^{4}_{2}{\frac{\sqrt{16-x^{2}}}{x}\,dx}$

\begin{indication}
{Seguir las indicaciones realizadas en el apartado $a)$,
introduciendo:

\[
\frac{\sqrt{16-x^{2}}}{x}
\]
en la línea de edición y los valores 2 y 4 en \option{Límite
Inferior} y \option{Límite Superior} respectivamente.}
\end{indication}

\item $ \dint^{\frac{\pi}{2}}_{0}{\frac{dx}{3+\cos{2x}}}$

\begin{indication}
{Seguir las indicaciones realizadas en el apartado $a)$,
introduciendo:

\[
\frac{1}{3+\cos{2x}}
\]

en la línea de edición y los valores 0 y $\pi/2$ en \option{Límite
Inferior} y \option{Límite Superior} respectivamente.}
\end{indication}
\end{enumerate}

\item Calcular la siguiente integral
\[
\  \dint_{2}^{\infty}{x^{2}e^{-x}\,dx}.
\]

\begin{indication}
{Seguir las indicaciones realizadas en el apartado $a)$ del
ejercicio $1$, introduciendo $x^{2}e^{-x}$ en la línea de edición y
los valores $2$ e ${\infty}$ en \option{Límite Inferior}
 y \option{Límite Superior} respectivamente.}
\end{indication}

\item Representar la parábola $y=x^{2}-7x+6$, y calcular el área
limitada por dicha parábola, el eje de abscisas y las rectas $x=2$ y $x=6$.

\begin{indication}
{
\begin{enumerate}
\item Se introduce $x^{2}-7x+6$ en la línea de edición, se pincha en el botón \button{Ventana 2D} de la barra de botones para acceder al entorno de
gráficos de dos dimensiones, y una vez allí se pincha en el botón
\button{Representar Expresión}.

\item Para volver a la Ventana de Álgebra se pincha en el botón
\button{Activar la Ventana de Álgebra}, y una vez en ella se realiza
lo indicado en el apartado anterior sucesivamente con las
expresiones $x=2$ y $x=6$ para obtener sus representaciones
gráficas.

\item Puede observarse en la gráfica que, entre $x=2$ y $x=6$,
la parábola $y=x^{2}-7x+6$ se encuentra por debajo del eje de
abscisas, por lo que si se calcula el valor de la integral definida
de $x^{2}-7x+6$ entre esos límites el resultado será negativo. Para
hallar el área encerrada habrá que cambiar el signo al resultado.

\item Seleccionar $x^{2}-7x+6$. Utilizar el menú \menu{Cálculo->Integrales} y elegir
\option{Integral Definida}. Introducir $2$ en \option{Límite
Inferior} y $6$ en \option{Límite Superior}, y pinchar en
\button{Simplificar}. El área buscada será el número obtenido
cambiado de signo.

\item También podría haberse hallado el área pedida calculando la integral definida de $-(x^{2}-7x+6)$, o la de
$|x^{2}-7x+6|$, entre $x=2$ y $x=6$.
\end{enumerate}
}
\end{indication}

\item Calcular el área encerrada por la curva $y=x^{2}+2x-2$, las rectas $x=-3$ y $x=2$ y el eje de abscisas.
\begin{indication}
{
\begin{enumerate}
\item Como la función no tiene el mismo signo en todo el intervalo, habrá que descomponer
 el intervalo de integración en intervalos donde la función tenga el mismo
 signo y calcular la suma de las integrales en cada uno de estos intervalos, teniendo en
cuenta que cuando el signo de la función
 sea negativo en un intervalo, se deberá cambiar el signo de la integral
 según se indicó en el ejercicio anterior. También se puede calcular el area
 directamente, sin descomponer el intervalo, integrando $|y(x)|$.

\item Se introduce $x^{2}+2x-2$ en la línea de edición, se pincha en el botón
 \button{Ventana 2D} de la barra de botones para acceder al entorno de gráficos de dos dimensiones, y
una vez allí se pincha en el botón \button{Representar Expresión}.

\item Para volver a la Ventana de Álgebra se pincha en el botón
\button{Activar la Ventana de Álgebra}, y una vez en ella se realiza
lo indicado en el apartado anterior sucesivamente con las
expresiones $x=-3$ y $x=2$ para obtener sus representaciones
gráficas.

\item Puede observarse en la gráfica, que entre $x=-3$ y $x=2$,
la parábola $y=x^{2}+2x-2$ no se encuentra totalmente por encima ni
por debajo del eje de abscisas, por lo que si se calcula el valor de
la integral definida de $x^{2}+2x-2$ entre esos límites, el
resultado no será el área buscada.

\item Para hallar dicha área, habrá que calcular en primer lugar los valores
de $x$ en los que la parábola cambia de signo. Para ello se
selecciona la expresión $x^{2}+2x-2$ y se utiliza el menú
\menu{Resolver->Expresión}. Se elige \option{Método Algebraico} y
\option{Dominio Real}, y se pincha en \button{Resolver} para obtener
los valores de $x$ en que la parábola corta al eje de abscisas.
Dichos valores son $-\sqrt{3}-1$ y $\sqrt{3}-1$. En la gráfica se
observa que la parábola está por encima del eje de abscisas entre
$-3$ y $-\sqrt{3}-1$, y entre $\sqrt{3}-1$ y $2$, y está por debajo
del eje de abscisas entre $-\sqrt{3}-1$ y $\sqrt{3}-1$.

\item Para calcular el área encerrada entre la parábola y el eje de
abscisas, habrá que sumar los valores de las integrales entre $-3$ y
$-\sqrt{3}-1$, entre $\sqrt{3}-1$ y $2$, y entre $-\sqrt{3}-1$ y
$\sqrt{3}-1$, éste último cambiado de signo. Éstas integrales se
hacen siguiendo las indicaciones incluidas en el ejercicio $1$
apartado $a)$.

\item El cálculo del área se podía haber hecho de una forma más rápida
calculando el valor de la integral de $|x^{2}+2x-2|$ entre $x=-3$ y
$x=2$, sin necesidad de saber en qué zonas es positiva o negativa la
función que se desea integrar.

\end{enumerate}
}
\end{indication}

\item Calcular el área encerrada por la curva $y=x^{2}e^{-x}$ y el eje de abscisas.
\begin{indication}
{
\begin{enumerate}

\item Se representa la función $x^{2}e^{-x}$ siguiendo las indicaciones realizadas en el ejercicio
$3$, y se observa en la gráfica que la zona encerrada por la función
y el eje de abscisas se extiende desde el origen de coordenadas
hasta el $\infty$.

\item Para volver a la Ventana de Álgebra se pincha en el botón
\texttt{Activar la Ventana de Álgebra}, y una vez en ella se
calculan los valores de $x$ en que se anula la función $x^{2}e^{-x}$
para confirmar lo observado en la gráfica. Para ello se selecciona
la expresión $x^{2}e^{-x}$ y se utiliza el menú
\menu{Resolver->Expresión}. Se elige \option{Método Algebraico} y
\option{Dominio Real}, y se pincha en \button{Resolver} para obtener
los valores de $x$ en que se anula la función. Dichos valores son
$0$ e $\infty$.

\item Para calcular el área encerrada por la curva $y=x^{2}e^{-x}$ y el eje de abscisas, se
calcula la integral de $x^{2}e^{-x}$ entre $0$ e $\infty$, siguiendo
las indicaciones incluidas en el ejercicio $1$ apartado $a)$.
\end{enumerate}
}
\end{indication}
\item Hallar el área comprendida entre las parábolas $y=6x-x^{2}$ e $y=x^{2}-2x$

\begin{indication}
{
\begin{enumerate}
\item Se representan las parábolas $y=6x-x^2$ e $y=x^{2}-2x$ siguiendo
las indicaciones realizadas en el ejercicio $3$, y se observa en la
gráfica que, en la zona encerrada entre ambas parábolas, la función
que está por encima es $y=6x-x^2$.

\item Se vuelve a la Ventana de Álgebra, y una vez en ella se
calculan los valores de $x$ correspondientes a los puntos de
intersección de las parábolas. Para ello se utiliza el menú
\menu{Resolver->Sistema}, se introduce $2$ en el campo
\option{Número} y se pincha en el botón \button{Sí}. Se escriben las
ecuaciones de las parábolas en los campos $1$ y $2$ de la ventana
siguiente, y se pincha en \button{Resolver}, con lo que se obtienen
las coordenadas de los puntos de intersección de las parábolas. Las
abscisas de dichos puntos son $0$ y $4$ respectivamente.

\item Para calcular el área comprendida entre las parábolas, se
 calcula la integral de $(6x-x^{2})-(x^2-2x)$ entre $0$ y $4$, siguiendo
 las indicaciones incluidas en el ejercicio $1$ apartado $a)$.
\end{enumerate}
}
\end{indication}

\item  Representar gráficamente la región del primer cuadrante limitada
por la parábola $y^{2}=8x$, la recta $x=2$ y el eje $OX$, y hallar el volumen
generado en la rotación alrededor del eje $OX$ de la región anterior.

\begin{indication}
{
\begin{enumerate}
\item Se introduce $y^{2}=8x$ en la línea de edición, se pincha en el botón
 \button{Ventana 2D} de la barra de botones para acceder al entorno de gráficos de dos dimensiones, y
una vez allí se pincha en el botón \button{Representar Expresión}.

\item Para volver a la Ventana de Álgebra se pincha en el botón
\button{Activar la Ventana de Álgebra}, y una vez en ella se realiza
lo indicado en el apartado anterior con la expresión $x=2$ y se
observa en la gráfica la región del primer cuadrante limitada por la
parábola $y^{2}=8x$, la recta $x=2$ y el eje $OX$.

\item Para calcular el volumen generado en la rotación alrededor del eje $OX$ de la región anterior, se
 calcula la integral de $\pi8x$ entre $0$ y $2$, siguiendo
 las indicaciones incluidas en el ejercicio $1$ apartado $a)$.
\end{enumerate}
}
\end{indication}
\end{enumerate}

\section{Ejercicios propuestos}
\begin{enumerate}[leftmargin=*]
\item Hallar el área encerrada la parábola $y=9-x^{2}$ y la recta $y=-x$.
\item Hallar el área encerrada por la curva $y=e^{-|x|}$ y su asíntota.
\item Hallar el volumen generado en la rotación alrededor del eje $OX$ de la región plana limitada por la parábola $y=2x^{2}$, las rectas $x=0$, $x=5$ y el eje $OX$, representando previamente dicha región plana.
\item Hallar el volumen generado en la rotación alrededor del eje $OY$ del área limitada por la parábola $y^{2}=8x$ y la recta $x=2$.
\end{enumerate}