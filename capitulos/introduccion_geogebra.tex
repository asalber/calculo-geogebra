% !TEX root = ../practicas_geogebra.tex
% Author: Alfredo Sánchez Alberca (asalber@ceu.es)
\chapter{Introducción a Geogebra}

\section{Introducción}
La gran potencia de cálculo alcanzada por los ordenadores en las últimas décadas, ha convertido a los mismos en poderosas herramientas al servicio de todas aquellas disciplinas que, como las matemáticas, requieren cálculos largos y complejos.

Geogebra \renewcommand{\thefootnote}{\fnsymbol{footnote}}\footnote{Esta practica está basada en la versión 6.1 de Geogebra} es uno de los programas de cálculo numérico y simbólico más utilizados.
Aparte de sus capacidades el cálculo numérico, vectorial y matricial, también permite realizar representaciones gráficas, lo cual permite resolver multitud de problemas de álgebra, análisis, cálculo, geometría e incluso estadística.
La ventaja de Derive frente a otros programas habituales de cálculo como Mathematica, Mapple o MATLAB, radica en su sencillez y simplicidad de uso, lo cual lo hace idóneo para la enseñanza de las matemáticas.

% \begin{center}
% 	\includegraphics[scale=0.6]{img/introduccion_derive/introduccion}
% \end{center}

% El objetivo de esta práctica es introducir al alumno en la utilización de este programa, enseñándole a realizar las operaciones básicas más habituales.


% \section{Funciones básicas}
% \subsection*{Arranque}
% Como cualquier otra aplicación de Windows, para arrancar el programa hay que pulsar sobre la opción correspondiente del menú \menu{Inicio>Programas}, o bien sobre el icono de escritorio

% \begin{center}
% 	\includegraphics[scale=0.4]{img/introduccion_derive/icono}
% \end{center}

% Cuando el programa arranca, en la pantalla aparece la ventana principal del programa que se conoce como \emph{ventana de Álgebra} (figura \ref{g:principal}).

% \begin{figure}[h!]
% 	\begin{center}
% 		\includegraphics[scale=0.6]{img/introduccion_derive/principal}
% 		\caption{Ventana principal de Derive.} \label{g:principal} 
% 	\end{center}
% \end{figure}

% Como cualquier otra ventana de aplicación de Windows, la ventana principal tiene una barra de título, una barra de menús con las distintas funciones que puede hacer Derive (cálculo de límites, derivadas, integrales, representaciones gráficas, etc.), una barra
% de botones que son atajos a las opciones más habituales de los menús, y una barra de estado en la parte inferior que nos indica lo que hace el programa en cada instante.
% Además, por defecto, en la parte inferior de la ventana aparece el editor de expresiones, que pasamos a describir a continuación.

% \subsection*{Edición de expresiones}
% Antes de realizar cualquier cálculo sobre una expresión matemática, lo primero es escribir dicha expresión y aprender a manipularla.

% \subsection*{Introducción de expresiones}
% Para introducir una expresión se utiliza el editor de expresiones (figura~\ref{g:editor}), el cual aparece directamente en la parte baja de la ventana de Álgebra. 

% \begin{figure}[h!]
% 	\begin{center}
% 		\includegraphics[scale=0.6]{img/introduccion_derive/authorexpression}
% 		\caption{Editor de expresiones.} \label{g:editor} 
% 	\end{center} 
% \end{figure}

% El editor de expresiones está compuesto por una línea de edición, que se utiliza para dar forma a las expresiones matemáticas (también permite introducir comentarios de texto) que vamos a utilizar con el programa, una barra con las letras del alfabeto griego, a menudo presentes en las expresiones matemáticas, y una barra de símbolos
% matemáticos con los operadores más habituales (suma, resta, producto, división, paréntesis, raíz cuadrada) y las constantes que más se utilizan (número $e$, número $\pi$...).

% En el editor de expresiones podemos escribir números, letras (que serán variables), símbolos y operadores aritméticos y relacionales. 
% Los operadores más habituales en la construcción de expresiones son los que aparecen en la siguiente tabla:

% \begin{center}
% 	\begin{tabular}{cc}
% 		\tcrule
% 		\textbf{Símbolo} & \textbf{Operador} \\
% 		\texttt{+}        & suma              \\
% 		\texttt{-}        & resta             \\
% 		\texttt{*}        & producto          \\
% 		\texttt{/}        & división         \\
% 		\texttt{\^{}}     & potencia          \\
% 		\bcrule
% 	\end{tabular}
% \end{center}

% A la hora de escribir una expresión hay que tener en cuenta que Derive tiene establecido un orden de prioridad en la evaluación de los operadores.
% En primer lugar evalúa las funciones y constantes predefinidas, después evalúa las potencias, después productos y cocientes (ambos con igual prioridad y de izquierda a derecha), y por último sumas y restas (ambas con igual prioridad y de izquierda a derecha).
% Para forzar la evaluación de una subexpresión, saltándose el orden de evaluación de Derive, se utilizan paréntesis.
% Así, como se ve en el siguiente ejemplo, dependiendo de cómo se introduzca una expresión pueden obtenerse resultados diferentes.


% \begin{center}\renewcommand{\arraystretch}{2}
% 	\begin{tabular}{cc}
% 		\tcrule
% 		\textbf{Expresión introducida} & \textbf{Expresión evaluada} \\
% 		\texttt{4x-1/x-5}               & $4x-\dfrac{1}{x}-5$          \\
% 		\texttt{(4x-1)/x-5}             & $\dfrac{4x-1}{x}-5$          \\
% 		\texttt{4x-1/(x-5)}             & $4x-\dfrac{1}{x-5}$          \\
% 		\texttt{(4x-1)/(x-5)}           & $\dfrac{4x-1}{x-5}$          \\
% 		\bcrule
% 	\end{tabular}
% \end{center}

% Cada vez que introducimos una expresión, esta aparece en la ventana de Algebra etiquetada con un número precedido del símbolo de almohadilla \verb"#", tal y como se muestra en la figura~\ref{g:expresiones}.
% Posteriormente, cada vez que queramos hacer referencia a dicha expresión podremos utilizar su etiqueta en lugar de volver a escribir la expresión.

% Es posible seleccionar cualquier expresión o subexpresión de la ventana Algebra con el ratón o bien con las teclas del cursor.

% La tecla \texttt{F3} permite introducir la expresión que tengamos seleccionada en el editor de expresiones.

% \subsubsection*{Modificación de expresiones}
% Una vez introducida una expresión, podemos volver a editarla para realizar cualquier corrección o cambio mediante el menú \menu{Editar>Expresión} y aparecerá la ventana del editor de expresiones con la expresión seleccionada.

% \subsubsection*{Eliminación de expresiones}
% Para eliminar una expresión de la ventana de Algebra, basta con seleccionarla y utilizar el menú \menu{Editar>Borrar} y la expresión seleccionada desaparecerá automáticamente, mientras que el resto de las expresiones se reenumeran  automáticamente.
% También es posible eliminar bloques completos de expresiones consecutivas seleccionando previamente el bloque de expresiones a eliminar.

% \textbf{¡Importante!}: Si hemos eliminado alguna expresión por equivocación, es posible recuperarla mediante el menú \menu{Editar>Recuperar}.

% \subsubsection*{Reordenación de expresiones}
% Es posible cambiar la posición que ocupa una expresión en la ventana de Álgebra marcándola y arrastrándola mediante el ratón hasta la posición que queremos que ocupe. Al cambiar la posición de una expresión, inmediatamente se reenumeran las expresiones de la ventana de Álgebra.

% \subsubsection*{Introducción de comentarios}
% Hay dos formas diferentes para introducir un comentario en la secuencia de expresiones. La primera consiste en utilizar la línea de edición escribiendo el texto del comentario entre comillas, y, si procedemos de esta manera, el comentario aparecerá como una expresión más, con su correspondiente etiqueta de ordenación.
% La segunda es mediante el menú \menu{Insertar>Objeto de Texto}, y de esta forma el comentario aparece sin etiqueta de ordenación ya que se trata de un objeto más insertado en el archivo, como también lo sería una gráfica, un dibujo, una fotografía o una hoja de cálculo\ldots

% \subsubsection*{Nombres de variables}
% Por defecto Derive utiliza una sola letra para representar una variable, de manera que la expresión \texttt{xy}, no se interpreta como una variable de nombre $xy$, sino como el producto de la variable $x$ por la variable $y$.
% Además, por defecto, no distingue entre mayúsculas y minúsculas.
% Por ejemplo, Derive interpretará que queremos trabajar con la función coseno tanto si
% introducimos en la línea de edición $\cos(x)$ como si introducimos $\cos(X)$.
% No obstante, es posible hacer que el programa utilice variables con más de una letra y distinga entre mayúsculas y minúsculas mediante el menú \menu{Opciones>Ajustes de
% Modo>Introducción}.

% \subsubsection*{Definición de constantes y funciones}
% Es posible definir constantes y funciones mediante el operador de definición \texttt{:=}.
% Para definir una constante basta con escribir el nombre de la constante seguido de \texttt{:=} y el valor de dicha constante.
% Por ejemplo para definir la constante de la aceleración de la gravedad, escribiríamos \texttt{g:=9.8}.
% Por otro lado, para definir una función se escribe el nombre de la función seguido de la lista de variables de la misma separadas por comas y entre paréntesis; después se escribe \texttt{:=} y por último la expresión que define la función.
% Así, por ejemplo, para definir la función que calcula el área de un triángulo de base $b$ y altura $h$, escribiríamos \texttt{a(b,h):=(b*h)/2} (ver figura~\ref{g:expresiones}).

% Con respecto a la definición de funciones, o de constantes, resultan especialmente importantes dos matizaciones:

% \begin{itemize}
% 	\item Si hemos definido una función o una constante, la definición permanece activa durante toda la sesión de trabajo con el documento, incluso si borramos la expresión en la que hemos procedido a la definición (al borrar en la pantalla no borramos la  memoria interna en la que se almacenan las definiciones de las constantes y funciones).
% 	      Para cambiar una definición previa no quedará más remedio que redefinir (g:=9.812 ), o dejar la asignación en blanco si lo que queremos es borrar la definición (g:= ).
	      	      	      	      	      
% 	\item En las definiciones de funciones sí que, por defecto, Derive distingue entre minúsculas y mayúsculas. De tal forma que, por ejemplo, distinguirá entre $a(b,h)$ y $A(b,h)$.
	      	      	      	      	      
% \end{itemize}

% \subsubsection*{Funciones y constantes predefinidas} Derive tiene ya implementadas la mayoría de la funciones elementales y constantes que suelen utilizarse en los cálculos matemáticos.
% La sintaxis de algunas de estas funciones y constantes se muestra en la tabla~\ref{t:funcioneselementales}, aunque, muy a menudo, en lugar de utilizar dicha sintaxis se utilizan los operadores y constantes que aparecen en la barra de símbolos. Por ejemplo, se puede observar cómo cambia el aspecto de la letra \texttt{e} introducida en la línea de edición como un variable más, o si en su lugar utilizamos \verb"#"\texttt{e}, o la \emph{e} que aparece en la barra de símbolos.
% En los dos últimos casos lo que hemos introducido en la línea de edición es la constante de Euler, base de los logaritmos naturales.

% Para conocer todas las funciones predefinidas de Derive lo mejor es utilizar el menú \menu{Ayuda>En Línea} y visitar la sección \texttt{Funciones y Constantes Internas}.

% \begin{table}[h!]
% 	\centering
% 	\begin{tabular}{cl}
% 		\tcrule
% 		\textbf{Syntaxis}   & \textbf{Constante o función}                  \\
% 		\verb"#"\command{e} & Constante de Euler $e=2.71828\ldots$           \\
% 		\command{pi}        & El número $\pi=3.14159\ldots$                 \\
% 		\verb"#"\command{i} & El número imaginario $i=\sqrt{-1}$            \\
% 		\command{inf}       & Infinito $\infty$                              \\
% 		\command{exp(x)}    & Función exponencial $e^x$                     \\
% 		\command{log(x,a)}  & Función logarítmica con base $a$, $\log_a x$ \\
% 		\command{ln(x)}     & Función logaritmo neperiano $\ln x$           \\
% 		\command{sqrt(x)}   & Función raíz cuadrada $\sqrt{x}$             \\
% 		\command{sin(x)}    & Función seno $\sin x$                         \\
% 		\command{cos(x)}    & Función coseno $\cos x$                       \\
% 		\command{tan(x)}    & Función tangente $\tan x$                     \\
% 		\command{asin(x)}   & Función arcoseno $\arcsin x$                  \\
% 		\command{acos(x)}   & Función arcocoseno $\arccos x$                \\
% 		\command{atan(x)}   & Función arcotangente $\arctan x$              \\
% 		\bcrule
% 	\end{tabular}
% 	\caption{Sintaxis de algunas funciones elementales y constantes predefinidas en Derive.} \label{t:funcioneselementales}
% \end{table}

% \textbf{¡Importante!}: en las funciones predefinidas, Derive, por defecto, no distingue entre mayúsculas y minúsculas. 
% Por ejemplo, opera con la función coseno tanto si introducimos $\cos(x)$, $\operatorname {Cos}(x)$, o $\operatorname {COS}(x)$.

% \subsubsection*{Vectores y matrices}
% Derive también permite la manipulación de vectores y matrices.
% Para crear un vector se utiliza el menú \menu{Introducir>Vector}.
% Al seleccionar este menú aparece un cuadro de diálogo donde debemos introducir el número de elementos del vector, y tras pulsar \texttt{Sí} aparece otro cuadro de diálogo donde deben introducirse las componentes del mismo.

% Otra forma de introducir vectores es mediante la línea de edición, introduciendo entre corchetes las componentes del vector separadas por comas.
% Por ejemplo, para introducir el vector $(x,y,z)$ escribiríamos \texttt{[x,y,z]} (ver figura~\ref{g:expresiones}).

% Para crear matrices se utiliza el menú \menu{Introducir>Matriz}.
% Con este menú aparece un cuadro de diálogo donde debemos introducir las filas y las columnas de nuestra matriz, y tras pulsar \texttt{Sí}, aparece otro cuadro de diálogo donde deben introducirse las componentes de la misma.

% Otra forma de introducir matrices es mediante la línea de edición, introduciendo entre corchetes los vectores fila que componen la matriz separados por comas, teniendo en cuenta que, como se explica anteriormente, cada vector debe ir escrito a su vez entre corchetes.
% Así, para introducir por ejemplo la matriz
% \[
% 	\left(
% 	\begin{array}{ccc}
% 		1 & 2 & 3 \\
% 		a & b & c \\
% 	\end{array}
% 	\right)
% \]
% escribiríamos \texttt{[[1,2,3],[a,b,c]]} (ver figura~\ref{g:expresiones}).

% \subsubsection*{Anotaciones}
% Es posible asociar a cada expresión una pequeña anotación, o nota.
% Para ello se selecciona la expresión y se utiliza el menú \menu{Editar>Anotacion}. Dicha anotación aparecerá en la barra de estado cada vez que seleccionemos la expresión y también es posible imprimirlo junto a la expresión.

% \begin{figure}[h!]
% 	\begin{center}
% 		\includegraphics[scale=0.6]{img/introduccion_derive/expresiones}
% 		\caption{Ventana de Algebra con distintos tipos de expresiones.}
% 		\label{g:expresiones}
% 	\end{center}
% \end{figure}


% \subsection*{Manipulación de archivos}
% Las expresiones y los cálculos realizados dentro de la ventana de Álgebra suelen almacenarse en archivos.

% \subsubsection*{Guardar un archivo}
% Para crear un archivo donde se guarden las expresiones de la ventana de Álgebra se utiliza el menú \menu{Archivo>Guardar}, y en el cuadro de diálogo que aparece se le da nombre al archivo y se selecciona la carpeta donde queremos guardarlo. Derive le pone
% automáticamente la extensión \texttt{*.dfw} a sus archivos.
% Una vez creado el archivo, su nombre aparecerá en la barra de título de la ventana de Derive. Posteriormente, para guardar cambios en una ventana de Álgebra, bastará con seleccionar de nuevo el menú \menu{Archivo>Guardar}, de manera que el archivo se actualizará.

% \subsubsection*{Recuperar un archivo}
% Para recuperar en una ventana de Álgebra el contenido de un archivo se utiliza el menú \menu{Archivo>Abrir}, y en en cuadro de diálogo que aparece se selecciona el archivo deseado.
% Automáticamente el contenido del archivo aparece en una ventana nueva de Álgebra.

% Otra forma de abrir archivos es mediante el menú \menu{Archivo>Leer>Math}, que se utiliza para almacenar en memoria la definición de nuevas funciones, presentes en los archivos con extensión *.mth, que expanden el potencial de cálculo del núcleo
% del programa, el cual queda operativo nada más arrancar Derive. 
% Al igual que antes aparece un cuadro de diálogo donde debemos seleccionar el archivo que queremos abrir, sólo que ahora, el contenido del archivo no aparece en una nueva ventana de Álgebra, sino que se añade en la ventana de Álgebra activa, a continuación de las expresiones existentes. Otra forma de proceder con igual resultado es mediante el menú \menu{Archivo>Leer>Utilidades}, que también permite acceder hasta, y cargar en memoria, los archivos con extensión *.mth, pero en este caso el conjunto de expresiones
% que componen dichos archivos no aparece en la pantalla, aunque sí que, al estar cargadas en la memoria del ordenador, serán operativas.

% \subsubsection*{Cerrar y abrir nuevas ventanas de Álgebra}
% Cuando terminemos una sesión de trabajo, podemos cerrar la ventana de Álgebra correspondiente mediante el menú \menu{Archivo>Cerrar}. 
% Por otro lado, en cualquier momento de una sesión de trabajo podemos abrir, añadidas a la que aparece por defecto, tantas ventanas de Álgebra como estimemos oportunas
% mediante el menú \menu{Archivo>Nuevo}. 
% El programa trabaja con cada una de las ventanas de Álgebra activas de forma completamente independiente, lo cual implica, entre otras cosas, que podremos utilizar los mismos nombres de variables en todas las ventanas abiertas sin interferencia entre las mismas. 

% \subsubsection*{Impresión}
% Para imprimir el contenido de una ventana de Álgebra, o bien una gráfica, se utiliza el menú \menu{Archivo>Imprimir}. 
% En el caso de una ventana de Álgebra aparecerá un cuadro de diálogo donde se puede seleccionar \texttt{Todo}, para imprimir todo el contenido de la ventana, \texttt{Páginas} para imprimir un rango de páginas o \texttt{Selección} para imprimir la zona previamente seleccionada de la ventana. No obstante, antes de imprimir, conviene utilizar el menú \menu{Archivo>Vista Previa} para ver por pantalla cómo quedaría la hoja impresa.
% Si todo está bien, bastaría con pulsar el botón \texttt{Imprimir} para que aparezca el cuadro de diálogo de impresión y desde ahí enviarlo a la impresora.
% La orientación y los márgenes pueden cambiarse con el menú \menu{Archivo>Configurar Página}, mientras que otras opciones como el tipo de letra, o el encabezado y pie de página se controlan mediante el menú \menu{Opciones>Impresión>Cabecera y Pie}.

% \subsection*{Simplificación de expresiones}
% Derive incorpora varios sistemas de simplificación de expresiones.
% El más sencillo es la simplificación básica, que puede realizarse mediante el menú \menu{Simplificar>Normal}.
% Este menú permite realizar simplificaciones simples como por ejemplo convertir la expresión $x+x$ en la expresión $2x$. 
% Sin embargo, no permite pasar de un binomio como $(x+1)^2$ a su desarrollo $x^2+2x+1$, ya que no está claro cuál de las dos expresiones es más simple.
% Para obtener el desarrollo de este binomio se utiliza el menú \menu{Simplificar>Expandir} que permite expandir una expresión con respecto sus variables. Por el contrario, si lo que queremos es pasar del desarrollo a la forma del binomio, se utiliza el menú \menu{Simplificar>Factorizar} que permite factorizar una expresión con respecto a sus variables.

% En cualquiera de estas simplificaciones, Derive trabaja por defecto en modo exacto y por eso devuelve expresiones fraccionarias.
% Para obtener el valor de una expresión en modo aproximado, con decimales, se utiliza el menú \menu{Simplificar>Aproximar}.
% Con este menú aparece un cuadro de diálogo donde debemos introducir el número de
% decimales que queremos para la aproximación.

% Por último, es posible sustituir cualquier variable de una expresión por un valor u otra expresión mediante el menú \menu{Simplificar>Sustituir Variable}.
% En el cuadro de diálogo que aparece se elige la variable a sustituir y se introduce la
% expresión o el valor de sustitución en \texttt{Nuevo Valor}.

% \subsection*{Representaciones gráficas}
% Derive permite representar gráficamente funciones en 2 y 3 dimensiones.

% \subsubsection*{Gráficas en 2 dimensiones}
% Para representar una función o expresión de una variable, se selecciona la expresión y se utiliza el menú \menu{Ventana>Nueva Ventana 2D}.
% Automáticamente aparece una ventana de gráficas en 2 dimensiones con unos ejes cartesianos, y para que aparezca la gráfica de la función, basta con pulsar el menú
% \menu{Insertar>Gráfica} de esta ventana, o pulsar en su correspondiente botón de la barra de botones. 
% En la figura~\ref{g:2d-plot} se muestra un ejemplo de gráfica en 2 dimensiones.

% Si queremos que la gráfica, una vez obtenida, también aparezca en la ventana de Álgebra como un objeto más de la misma, desde la ventana 2D, utilizamos el menú \menu{Archivo>Incrustar}.

% \begin{figure}[h!]
% 	\begin{center}
% 		\includegraphics[scale=0.6]{img/introduccion_derive/2d-plot}
% 		\caption{Ventana de gráficas en 2 dimensiones.} \label{g:2d-plot}
% 	\end{center}
% \end{figure}

% Es posible representar más de una función en una misma gráfica, seleccionando la nueva expresión en la ventana de Álgebra, y pulsando de nuevo el menú \menu{Insertar>Gráfica} en la ventana de gráficos en 2 dimensiones en que queramos que aparezca la
% representación gráfica de la expresión seleccionada.
% Cuando se quieren representar varias funciones, a veces resulta más cómodo mostrar al mismo tiempo la ventana de Álgebra y la de gráficas mediante el menú \menu{Ventana>Mosaico Vertical}, tal y como se muestra en la figura~\ref{g:expresionesygraficas}.

% \begin{figure}[h!]
% 	\begin{center}
% 		\includegraphics[scale=0.6]{img/introduccion_derive/expresionesygraficas}
% 		\caption{Ventana de Álgebra y de gráficas en 2 dimensiones en una
% 			misma pantalla.} \label{g:expresionesygraficas}
% 	\end{center}
% \end{figure}

% También es posible borrar gráficas mediante el menú \menu{Editar>Borrar Gráfica}. Si se elige la opción \texttt{Primera} se borra la primera gráfica dibujada, si se elige
% la opción \texttt{Última} se borra la última, y si se elige la opción \texttt{Anteriores} borra todas las gráficas excepto la última.

% En la ventana de gráficas en 2 dimensiones existen distintos menús que permiten cambiar el aspecto de la gráfica representada.
% Una posibilidad muy interesante es cambiar la escala de los ejes mediante el menú \menu{Seleccionar>Relación de Aspecto}.

% También es posible ampliar la representación gráfica de una determinada zona del gráfico mediante el menú \menu{Seleccionar>Rango de la Gráfica}, introduciendo las
% coordenadas de la zona que queremos ampliar, aunque es más práctico utilizar el botón \texttt{Seleccionar el rango}, y después utilizar el ratón para delimitar la zona que queremos ampliar.

% En la ventana de gráficas en 2 dimensiones aparece una cruz que representa al cursor. Las coordenadas del cursor siempre aparecen en la barra de estado.
% Cuando se pulsa la tecla \texttt{F3}, la cruz se transforma en un cuadradito y se pasa a \emph{modo de traza}.
% En este modo, al mover el cursor con las flechas del teclado, el cursor sigue la trayectoria de la función representada, con lo que podemos averiguar los valores que toma la misma en la barra de estado, tal y como se muestra en la figura~\ref{g:modotraza}.

% \begin{figure}[h!]
% 	\begin{center}
% 		\includegraphics[scale=0.6]{img/introduccion_derive/modotraza}
% 		\caption{Ventana de gráficas en 2 dimensiones en modo de traza con
% 			una gráfica ampliada.} \label{g:modotraza}
% 	\end{center}
% \end{figure}

% Es posible centrar la gráfica de una función en cualquier punto mediante el menú \menu{Seleccionar>Rango de la Gráfica>Longitud/Centro}, aunque, de nuevo, tal vez sea más
% operativo hacerlo mediante los botones \texttt{Centrar en el cursor} y \texttt{Centrar en el origen}.

% \subsubsection*{Gráficas en 3 dimensiones}
% Para representar una función o expresión de dos variables, se selecciona la expresión y se utiliza el menú \menu{Ventana>Nueva Ventana 3D}.
% Automáticamente aparece una ventana de gráficas en 3 dimensiones con unos ejes cartesianos, y para que aparezca la gráfica de la función, basta con pulsar el menú \menu{Insertar>Gráfica} de esta ventana.
% En la figura~\ref{g:3d-plot} se muestra un ejemplo de gráfica en 3 dimensiones.

% \begin{figure}[h!]
% 	\begin{center}
% 		\includegraphics[scale=0.6]{img/introduccion_derive/3d-plot}
% 		\caption{Ventana de gráficas en 3 dimensiones.} \label{g:3d-plot}
% 	\end{center}
% \end{figure}

% Al igual que en el caso de las gráficas de 2 dimensiones, existen distintos menús que permiten cambiar el aspecto de la gráfica representada.
% De todos ellos, sólo comentaremos el menú \menu{Editar>Gráfica>Número de Paneles} que permite cambiar la resolución del gráfico, y el menú \menu{Seleccionar>Posición de
% Ojo} que permite cambiar la posición desde donde se mira la gráfica.
