% !TEX root = ../practicas_geogebra.tex
% Author: Alfredo Sánchez Alberca (asalber@ceu.es)
\chapter{Funciones Elementales}

% \section{Fundamentos teóricos}

% En esta práctica se introducen los conceptos básicos sobre funciones reales de variable real, esto es, funciones
% \[f:\mathbb{R}\rightarrow \mathbb{R}.\]

% \subsection{Dominio e imagen}

% El \emph{Dominio} de la función $f$ es el conjunto de los números reales $x$ para los que existe $f(x)$ y se designa mediante $\dom f$.

% La \emph{Imagen} de $f$ es el conjunto de los números reales $y$ para los que existe algún $x\in \mathbb{R}$ tal que $f(x)=y$, y se denota por $\im f$.


% \subsection{Signo y crecimiento}
% El \emph{signo} de la función es positivo $(+)$ en los valores de $x$ para los que $f(x)>0$ y negativo $(-)$ en los que $f(x)<0$.
% Los valores de $x$ en los que la función se anula se conocen como \emph{raíces} de la función.

% Una función $f(x)$ es \emph{creciente} en un intervalo $I$ si $\forall\, x_1, x_2 \in I$ tales que $x_1<x_2$ se verifica que $f(x_1)\leq f(x_2)$.

% Del mismo modo, se dice que una función $f(x)$ es \emph{decreciente} en un intervalo $I$ si $\forall\, x_1, x_2 \in I$ tales que $x_1<x_2$ se verifica que $f(x_1)\geq f(x_2)$. En la figura~\ref{g:crecimiento} se muestran estos conceptos.

% \begin{figure}[h!]
% 	\centering \subfigure[Función creciente.] {\label{g:funcion_creciente}
% 		\scalebox{1}{\input{img/funciones_elementales/funcion_creciente}}}\qquad
% 	\subfigure[Función decreciente.]{\label{g:funcion_decreciente}
% 		\scalebox{1}{\input{img/funciones_elementales/funcion_decreciente}}}
% 	\caption{Crecimiento de una función.}
% 	\label{g:crecimiento}
% \end{figure}


% \subsection{Extremos Relativos}
% Una función $f(x)$ tiene un \emph{máximo relativo} en $x_0$ si existe un entorno $A$ de $x_0$ tal que $\forall x \in A$
% se verifica que $f(x)\leq f(x_0)$.

% Una función $f(x)$ tiene un \emph{mínimo relativo} en $x_0$ si existe un entorno $A$ de $x_0$ tal que $\forall x\in A$
% se verifica que $f(x)\geq f(x_0)$.

% Diremos que la función $f(x)$ tiene un \emph{extremo relativo} en un punto si tiene un \emph{máximo o mínimo relativo}
% en dicho punto. Estos conceptos se muestran en la figura~\ref{g:extremos}.

% \begin{figure}[h!]
% 	\centering \subfigure[Máximo relativo.] {\label{g:maximo}
% 		\scalebox{1}{\input{img/funciones_elementales/maximo}}}\qquad
% 	\subfigure[Mínimo relativo.]{\label{g:minimo}
% 		\scalebox{1}{\input{img/funciones_elementales/minimo}}}
% 	\caption{Extremos relativos de una función.}
% 	\label{g:extremos}
% \end{figure}

% Una función $f(x)$ está \emph{acotada superiormente} si $\exists K\in\mathbb{R}$ tal que $f(x)\leq K$ $\forall x \in \dom f$. Análogamente, se dice que una función $f(x)$ está \emph{acotada inferiormente} si $\exists K\in\mathbb{R}$ tal que $f(x)\geq K$ $\forall x \in \dom f$.

% Una función $f(x)$ está \emph{acotada} si lo está superior e inferiormente, es decir si $\exists K\in\mathbb{R}$ tal que $|f(x)|\leq K$ $\forall x \in \dom f$.


% \subsection{Concavidad}

% De forma intuitiva se puede decir que una función $f(x)$ es \emph{cóncava} en un intervalo $I$ si $\forall\, x_1, x_2
% \in I$, el segmento de extremos $(x_1,f(x_1))$ y $(x_2,f(x_2))$ queda por encima de la gráfica de $f$.

% Análogamente se dirá que es \emph{convexa} si el segmento anterior queda por debajo de la gráfica de $f$.

% Diremos que la función $f(x)$ tiene un \emph{punto de inflexión} en $x_0$ si en ese punto la función pasa de cóncava a
% convexa o de convexa a cóncava. Estos conceptos se ilustran en la figura~\ref{g:concavidad}.

% \begin{figure}[h!]
% 	\centering \subfigure[Función cóncava.] {\label{g:funcion_convexa}
% 		\scalebox{1}{\input{img/funciones_elementales/funcion_convexa}}}\qquad
% 	\subfigure[Función convexa.]{\label{g:funcion_concava}
% 		\scalebox{1}{\input{img/funciones_elementales/funcion_concava}}}
% 	\caption{Concavidad de una función.}
% 	\label{g:concavidad}
% \end{figure}

% \subsection{Asíntotas}

% La recta $x=a$ es una \emph{asíntota vertical} de la función $f(x)$ si al menos uno de los límites laterales de $f(x)$ cuando $x$ tiende hacia $a$ es $+\infty$ o $-\infty$, es decir cuando se verifique alguna de las siguientes igualdades
% \[
% 	\ \lim_{x\rightarrow a^{+}}f(x)=\pm\infty   \quad \textrm{o} \quad
% 	\lim_{x\rightarrow a^{-}}f(x)=\pm\infty
% \]

% La recta $y=b$ es una \emph{asíntota horizontal} de la función $f(x)$ si alguno de los límites de $f(x)$ cuando $x$ tiende hacia $+\infty$ o $-\infty$ es igual a $b$, es decir cuando se verifique
% \[
% 	\ \lim_{x\rightarrow -\infty }f(x)=b    \quad \textrm{o} \quad
% 	\ \lim_{x\rightarrow +\infty }f(x)=b
% \]

% La recta $y=mx+n$ es una \emph{asíntota oblicua} de la función $f(x)$ si alguno de los límites de $f(x)-(mx+n)$ cuando $x$ tiende hacia $+\infty$ o $-\infty$ es igual a 0, es decir si

% \[
% 	\ \lim_{x\rightarrow -\infty }{(f(x)-mx)}=n    \quad \textrm{o} \quad
% 	\ \lim_{x\rightarrow +\infty }{(f(x)-mx)}=n
% \]

% En la figura~\ref{g:asintotas} se muestran los distintos tipos de asíntotas.

% \begin{figure}[h!]
% 	\centering \subfigure[Asíntota horizontal y vertical.] {\label{g:asintotahorizontalyvertical}
% 		\scalebox{1}{\input{img/funciones_elementales/asintota_vertical}}}\qquad\qquad
% 	\subfigure[Asíntota vertical y oblicua.]{\label{g:asintotaoblicua}
% 		\scalebox{1}{\input{img/funciones_elementales/asintota_oblicua}}}
% 	\caption{Tipos de asíntotas de una función.}
% 	\label{g:asintotas}
% \end{figure}


% \subsection{Periodicidad}
% Una función $f(x)$ es \emph{periódica} si existe $h\in\mathbb{R^{+}}$ tal que \[f(x+h)=f(x)\  \forall x\in \dom f\] siendo el período $T$ de la función, el menor valor $h$ que verifique la igualdad anterior.

% En una función periódica, por ejemplo $f(x)=A\sen(wt)$, se denomina \emph{amplitud} al valor de $A$, y es la mitad de la diferencia entre los valores máximos y mínimos de la función. En la figura~\ref{g:periodoyamplitud} se ilustran estos conceptos.

% \begin{figure}[h!]
% 	\centering
% 	\scalebox{0.8}{\input{img/funciones_elementales/funcion_periodica}}
% 	\caption{Periodo y amplitud de una función periódica.}
% 	\label{g:periodoyamplitud}
% \end{figure}

% \clearpage
% \newpage

\section{Ejercicios resueltos}

\begin{enumerate}[leftmargin=*]
\item Se considera la función
      \[
      f(t)=\frac{t^{4} +19\cdot t^{2} - 5}{t^{4} +9\cdot t^{2} - 10}.
      \]

      Representarla gráficamente y determinar a partir de dicha representación:

      \begin{enumerate}
      \item  Dominio.
            \begin{indication}
            \begin{enumerate}
            \item Para representarla gráficamente, introducir la función en la barra de \field{Entrada} de la \field{Vista CAS} y activar la \field{Vista Gráfica}.
            \item Para determinar el dominio tan sólo hay que determinar los valores de $x$ en los que existe la función.
            \item Recordar que, tanto para éste como para el resto de los apartados del ejercicio, pretendemos llegar a conclusiones aproximadas que tan sólo sacamos del análisis de la gráfica.
            \end{enumerate}
            \end{indication}

      \item  Imagen.
            \begin{indication}
            Fijarse en los valores de la variable $y$ hasta los que llega la función.
            \end{indication}

      \item  Asíntotas.
            \begin{indication}
            Son las líneas rectas, ya sea horizontales, verticales u oblicuas, hacia las que tiende la función.
            \end{indication}

      \item  Raíces.
            \begin{indication}
            Son los valores de la variable $x$, si los hay, en los que la función vale 0.
            \end{indication}

      \item Signo.
            \begin{indication}
            Hay que determinar, aproximadamente, por un lado los intervalos de variable $x$ en los que la función es positiva, y por el otro aquellos en
            los que es negativa.
            \end{indication}

      \item  Intervalos de crecimiento y decrecimiento.
            \begin{indication}
            De nuevo, por un lado hay que determinar los intervalos de variable $x$ en los que a medida que crece $x$ también lo hace $y$, que serían
            los intervalos de crecimiento, y también aquellos otros en los que a medida que crece $x$ decrece $y$, que serían los intervalos de
            decremimiento.
            \end{indication}

      \item Intervalos de concavidad y convexidad.
            \begin{indication}
            Para los intervalos de concavidad y convexidad, nos fijamos en el segmento de línea recta que une dos puntos cualquiera del intervalo. Si
            dicho segmento queda por encima de la gráfica, entonces la función es cóncava en el intervalo, mientras que si queda por debajo, entonces es
            convexa en el mismo.
            \end{indication}

      \item Extremos relativos.
            \begin{indication}
            Determinamos, aproximadamente, los puntos en los que se encuentran los máximos y mínimos relativos de la función.
            \end{indication}

      \item Puntos de inflexión.
            \begin{indication}
            Determinamos, aproximadamente, los puntos en los que la función cambia de curvatura, de cóncava a convexa o a la inversa.
            \end{indication}
      \end{enumerate}

\item Representar en una misma gráfica las funciones $2^{x}, e^{x}, 0.7^{x}, 0.5^{x}$. A la vista de las gráficas obtenidas, indicar cuáles
      de las funciones anteriores son crecientes y cuáles son decrecientes.
      \begin{indication}
      Introducir cada función en la barra de \field{Entrada} de la \field{Vista CAS}.
      \end{indication}

      ¿En general, para qué valores de $a$ será la función creciente? ¿Y para qué valores de $a$ será decreciente? Probar con
      distintos valores de $a$ representando gráficamente nuevas funciones si fuera necesario.


\item Representar en una misma gráfica las funciones siguientes, indicando su período y amplitud.
      \begin{enumerate}
      \item $\sen{x}$, $\sen{x}+2$, $\sen{(x+2)}$.
      \item $\sen{2x}$, $2\sen{x}$, $\sen\frac{x}{2}$.
            \begin{indication}
            Introducir cada función en la barra de \field{Entrada} de la \field{Vista CAS}.
            \end{indication}
      \end{enumerate}


\item Representar en una gráfica la función
      \[
      \ f(x)=\left\{
      \begin{array}{cl}
      -2x   & \hbox{si $x\leq0$;} \\
      x^{2} & \hbox{si $x>0$.}    \\
      \end{array}
      \right.
      \]

      \begin{indication}
      Para representar funciones a trozos, Geogebra utiliza el comando
      \begin{center}
            \command{Si(<Condición>, <Entonces>, <Si no>)} 
      \end{center}
      y se pueden anidar varios comandos unos dentro de otros. 
      Utilizando este comando para representar la función anterior, habría que introducir la expresión
      \begin{center}
        Si[x<=0, -2x, x\^2]    
      \end{center}
      \end{indication}
\end{enumerate}


\section{Ejercicios propuestos}
\begin{enumerate}[leftmargin=*]
\item Hallar el dominio de las siguientes funciones a partir de sus representaciones gráficas:

      \begin{enumerate}
      \item $f(x)=\dfrac{x^{2} + x + 1}{x^{3} - x}$
      \item $g(x)=\sqrt[2]{x^{4}-1}$.
      \item $h(x)=\cos{\dfrac{x + 3}{x^{2} + 1}}$.
      \item $l(x)=\arcsen{\dfrac{x}{1+x}}$.
      \end{enumerate}

\item Se considera la función
      \[
      \ f(x)=\frac{x^{3} + x +2}{5x^{3} - 9x^{2} - 4x + 4}.
      \]

      Representarla gráficamente y determinar a partir de dicha representación:

      \begin{enumerate}
      \item Dominio.
      \item Imagen.
      \item Asíntotas.
      \item Raíces.
      \item Signo.
      \item Intervalos de crecimiento y decrecimiento.
      \item Intervalos de concavidad y convexidad.
      \item Extremos relativos.
      \item Puntos de inflexión.
      \end{enumerate}

\item Representar en una misma gráfica las funciones $\log_{10}{x}$, $\log_{2}{x}$, $\log{x}$, $\log_{0.5}{x}$.
      \begin{enumerate}
      \item A la vista de las gráficas obtenidas, indicar cuáles de las funciones anteriores son crecientes y cuáles son decrecientes.
      \item Determinar, a partir de los resultados obtenidos, o representando nuevas funciones si fuera necesario, para qué valores de $a$ será
            creciente la función $\log_{a}{x}$.
      \item Determinar, a partir de los resultados obtenidos, o representando nuevas funciones si fuera necesario, para qué valores de $a$ será
            decreciente la función $\log_{a}{x}$.
      \end{enumerate}

\item Completar las siguientes frases con la palabra igual, o el número de veces que sea mayor o menor en cada caso:
      \begin{enumerate}
      \item La función $\cos{2x}$ tiene un período............ que la función $\cos{x}$.
      \item La función $\cos{2x}$ tiene una amplitud............ que la función $\cos{x}$.
      \item La función $\cos\dfrac{x}{2}$ tiene un período............ que la función $\cos{3x}$.
      \item La función $\cos\dfrac{x}{2}$ tiene una amplitud............ que la función $\cos{3x}$.
      \item La función $3\cos{2x}$ tiene un período............ que la función $\cos\dfrac{x}{2}$.
      \item La función $3\cos{2x}$ tiene una amplitud............ que la función $\cos\dfrac{x}{2}$.
      \end{enumerate}

\item Hallar a partir de la representación gráfica, las soluciones de $e^{-1/x}=\dfrac{1}{x}$.

\item Representar en una gráfica la función
      \[
      \ f(x)=\left\{
      \begin{array}{ll}
      x^{3}   & \hbox{si $x<0$}    \\
      e^{x}-1 & \hbox{si $x\geq0$} \\
      \end{array}
      \right.
      \]

\end{enumerate}
