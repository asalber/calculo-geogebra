% Author: Alfredo Sánchez Alberca (asalber@ceu.es)
\chapter{Integrales}

\section{Fundamentos teóricos}
Junto al concepto de derivada, el de integral es otro de los más importantes
del cálculo matemático. Aunque dicho concepto surge en principio, como técnica
para el cálculo de áreas, el teorema fundamental del cálculo establece su
relación con la derivada, de manera que, en cierto sentido, la diferenciación
y la integración son operaciones inversas.

En esta práctica se introduce el concepto de integral como antiderivada, y
también el de integral de Riemann, que permite calcular áreas por debajo de
funciones acotadas en un intervalo.

\subsection{Primitivas e Integrales}
\subsubsection*{Función Primitiva}

Se dice que la función $F(X)$ es una \emph{función primitiva} de
$f(x)$ si se verifica que $F'(x)=f(x)$ $\forall x \in \dom f$.

Como dos funciones que difieran en una constante tienen la misma
derivada, si $F(x)$ es una función primitiva de $f(x)$ también lo será toda función de la forma $F(x)+k$ $\forall k \in \mathbb{R}$.\\


\subsubsection*{Función integral indefinida}

Se llama \emph{función integral indefinida} de la función $f(x)$ al
conjunto de todas sus funciones primitivas y se representa como:

\[
\ \int{f(x)}\,dx=F(x)+C
\]
siendo $F(x)$ una función primitiva de $f(x)$ y $C$ una constante arbitraria.\\


\subsubsection*{Linealidad de la integral}

Dadas dos funciones $f(x)$ y $g(x)$ que admiten primitiva, y una
constante $k \in \mathbb{R}$ se verifica que:

\[
\ \int{(f(x)+g(x))}\,dx=\int{f(x)}\,dx+\int{g(x)}\,dx
\]
y:
\[
\ \int{kf(x)}\,dx=k\int{f(x)}\,dx
\]\\


\subsection{Integral de Riemann}

Se llama \emph{partición} de un intervalo $[a,b]\subset\mathbb{R}$,
a una colección finita de puntos del intervalo,
$P=\{x_{0},x_{1},...,x_{n}\}$,  tales que
$x_{0}=a<x_{1}<...<x_{n}=b$, con lo que el intervalo $[a,b]$ queda
dividido en $n$ subintervalos $[x_{i},x_{i+1}]$, $i=0,...,n-1$.

Dada una función $f:[a,b]\rightarrow\mathbb{R}$ acotada y una
partición $P=\{x_{0},x_{1},...,x_{n}\}$ de $[a,b]$, se llama
\emph{suma inferior} de $f$ en relación a $P$, y se designa por
$L(P,f)$, a:

\[
\ L(P,f)=\sum_{i=1}^{n} m_{i}(x_{i}-x_{i-1})
\]
donde $  m_{i}=\inf\{f(x):x_{i-1}\leq x \leq x_{i}\}$.

Análogamente se llama \emph{suma superior} de $f$ en relación a $P$,
y se designa por $U(P,f)$, a:

\[
\ U(P,f)=\sum_{i=1}^{n} M_{i}(x_{i}-x_{i-1})
\]
donde $ M_{i}=\sup\{f(x):x_{i-1}\leq x \leq x_{i}\}$.

La \emph{suma inferior} y la \emph{suma superior} así definidas
representan las sumas de las áreas de los rectángulos que tienen por
bases los subintervalos de la partición, y por alturas los valores
mínimo y máximo respectivamente de la función $f$ en los
subintervalos considerados, tal y como se muestra en la
figura~\ref{g:sumassupinf}. Así, los valores de $L(P,f)$ y $U(P,f)$
serán siempre menores y mayores respectivamente, que el área
encerrada por la función $f$ y el eje de abscisas en el intervalo
$[a,b]$.

\begin{figure}[htbp]
\centering \subfigure[Suma inferior $L(P,f)$.]{
\label{g:sumainferior}
\scalebox{1}{\input{img/integrales/suma_inferior}}}\qquad\qquad
\subfigure[Suma superior$U(P,f)$.]{
\label{g:sumasuperior}
\scalebox{1}{\input{img/integrales/suma_superior}}}
\caption{Áreas medidas por las sumas superior e inferior
correspondientes a una partición.} \label{g:sumassupinf}
\end{figure}

Una función $f:[a,b]\rightarrow\mathbb{R}$ acotada es
\emph{integrable} en el intervalo $[a,b]$ si se verifica que:

\[
\ \sup\{L(P,f): P \textrm{ partición de } [a,b]\}=\inf\{U(P,f): P
\textrm{ partición de }[a,b]\}
\]
y ese número se designa por $\int_{a}^{b}f(x)\,dx$ o simplemente por
$\int_{a}^{b}f$.


\subsubsection*{Propiedades de la Integral}

\begin{enumerate}

\item \textbf{Linealidad}

Dadas dos funciones $f$ y $g$ integrables en $[a,b]$ y $k \in
\mathbb{R}$ se verifica que:

\[
\
\int_{a}^{b}(f(x)+g(x))\,dx=\int_{a}^{b}f(x)\,dx+\int_{a}^{b}g(x)\,dx
\]
y
\[
\ \int_{a}^{b}{kf(x)}\,dx=k\int_{a}^{b}{f(x)}\,dx
\]

\item \textbf{Monotonía}

Dadas dos funciones $f$ y $g$ integrables en $[a,b]$ y tales que
$f(x)\leq g(x)$ $\forall x \in [a,b]$ se verifica que:


\[
\ \int_{a}^{b}{f(x)\,dx} \leq \int_{a}^{b}{g(x)\,dx}
\]

\item \textbf{Acotación}

Si $f$ es una función integrable en el intervalo $[a,b]$, existen
$m,M\in\mathbb{R}$ tales que:

\[
\ m(b-a)\leq\int_{a}^{b}{f(x)\,dx} \leq \ M(b-a)
\]

\item \textbf{Aditividad}

Si $f$ es una función acotada en $[a,b]$ y $c\in(a,b)$, entonces $f$
es integrable en $[a,b]$ si y sólo si lo es en $[a,c]$ y en $[c,b]$,
verificándose además:

\[
\ \int_{a}^{b}{f(x)\,dx} =
\int_{a}^{c}{f(x)\,dx}+\int_{c}^{b}{f(x)\,dx}
\]\\

\end{enumerate}

\subsubsection*{Teorema Fundamental del Cálculo}

Sea $f : [a,b]\rightarrow\mathbb{R}$ continua y sea $G$ una función
continua en $[a,b]$. Entonces $G$ es derivable en $(a,b)$ y
$G'(x)=f(x)$ para todo $x\in(a,b)$ si y sólo si:

\[
\ G(x)-G(a) = \int_{a}^{x}f(t)\,dt
\]

\subsubsection*{Regla de Barrow}

Si $f$ es una función continua en $[a,b]$ y $G$ es continua en
$[a,b]$, derivable en $(a,b)$ y tal que $G'(x)=f(x)$ para todo
$x\in(a,b)$ entonces:

\[
\  \int_{a}^{b}{f} = G(b)-G(a)
\]


De aquí se deduce que:

\[
\  \int_{a}^{b}{f} = -\int_{b}^{a}{f}
\]


\subsection{Integrales impropias}

La integral $ \int_{a}^{b}{f(x)\,dx}$ se llama \emph{impropia} si el
intervalo $(a,b)$ no está acotado o si la función $f(x)$ no está
acotada en el intervalo $(a,b)$.

Si el intervalo $(a,b)$ no está acotado, se denomina integral
impropia de primera especie mientras que si la función no está
acotada en el intervalo se denomina integral impropia de segunda
especie.

\subsection{Cálculo de áreas}
Una de las principales aplicaciones de la integral es el cálculo de
áreas.

\subsubsection*{Área de una región plana encerrada por dos curvas}

Si $f$ y $g$ son dos funciones integrables en el intervalo $[a,b]$ y
se verifica que $g(x)\leq f(x)$ $\forall x\in[a,b]$, entonces el
área de la región plana limitada por las curvas $y=f(x)$, $y=g(x)$,
y las rectas $x=a$ y $x=b$ viene dada por:

\[
\ A = \int_{a}^{b}{(f(x)- g(x))\,dx}
\]\\

\noindent \textbf{Observaciones}

\begin{enumerate}

\item El intervalo $(a,b)$ puede ser infinito y la definición sería análoga, pero en ese caso es preciso que la integral impropia sea convergente.

\item Si $f(x)\geq0$ y $g(x)=0$ al calcular la integral entre $a$ y $b$ se obtiene el área encerrada por la función $f(x)$ y el eje de abscisas entre las rectas verticales $x=a$ y $x=b$ (figura~\ref{g:integral_definida}).

\begin{figure}[h!]
\begin{center}
\scalebox{1}{\input{img/integrales/integral_definida}}
\caption{Cálculo de área encerrada por la función $f(x)$ y el eje de
abscisas entre las rectas verticales $x=a$ y $x=b$  mediante la
integral definida.} \label{g:integral_definida}
\end{center}
\end{figure}

\item Si $f(x)\geq 0$ $\forall x\in[a,c]$ y $f(x)\leq 0$ $\forall x\in[c,b]$, el área de la región plana encerrada por $f$, las rectas verticales $x=a$ y $x=b$ y el eje de abscisas se calcula
mediante:
\[
\ A= \int_{a}^{c}{f(x)\,dx} - \int_{c}^{b}{f(x)\,dx}.
\]

\item Si las curvas $y=f(x)$ e $y=g(x)$ se cortan en los puntos de abscisas $a$ y $b$, no cortándose en ningún otro punto cuya abscisa esté comprendida entre $a$ y $b$, el área encerrada por dichas curvas entre esos puntos de corte puede calcularse
mediante:
\[
\ A= \int_{a}^{b}{|f(x)-g(x)|dx}
\]
\end{enumerate}


\subsection{Cálculo de Volúmenes}

\subsubsection*{Volumen de un sólido}
Si se considera un cuerpo que al ser cortado por un plano
perpendicular al eje $OX$ da lugar, en cada punto de abscisa $x$, a
una sección de área $A(x)$, el volumen de dicho cuerpo comprendido
entre los planos perpendiculares al eje $OX$ en los puntos de
abscisas $a$ y $b$ es:

\[
\ V = \int_{a}^{b}{A(x)\,dx}
\]

\subsubsection*{Volumen de un cuerpo de revolución}
Si se hace girar la curva $y=f(x)$ alrededor del eje $OX$ se genera
un sólido de revolución cuyas secciones perpendiculares al eje $OX$
tienen áreas $A(x)=\pi(f(x))^{2}$, y cuyo volumen comprendido entre
las abscisas $a$ y $b$ será:

\[
\ V = \int_{a}^{b}{\pi(f(x))^{2}\,dx}=
\pi\int_{a}^{b}{(f(x))^{2}\,dx}
\]


En general, el volumen del cuerpo de revolución engendrado al girar
alrededor del eje $OX$ la región plana limitada por las curvas
$y=f(x)$, $y=g(x)$ y las rectas verticales $x=a$ y $x=b$ es:

\[
\ V = \int_{a}^{b}{\pi|(f(x))^{2}-(g(x))^{2}|\,dx}
\]

De manera análoga se calcula el volumen del cuerpo de revolución
engendrado por la rotación de una curva $x=f(y)$ alrededor del eje
$OY$, entre los planos $y=a$ e $y=b$, mediante:

\[
\ V = \int_{a}^{b}{\pi(f(y))^{2}dy} = \pi \int_{a}^{b}{(f(y))^{2}dy}
\]


\section{Ejercicios resueltos}

\begin{enumerate}[leftmargin=*]
\item Calcular las siguientes integrales:
\begin{enumerate}
\item $ \int{x^2 \log x\,dx}$
\begin{indicacion}
\begin{enumerate}
\item Introducir la expresión \command{x\^{}2 log(x)} en la ventana de Álgebra.
\item Seleccionar el menú \menu{Calculo > Integrales} o hacer clic en el botón \button{Calcular integral}.
\item En el cuadro de diálogo que aparece marcar la opción \option{Indefinida}, introducir la constante \command{C} en el campo \field{Constante} y hacer clic en el botón \button{Simplificar}.
\end{enumerate}
Una manera más rápida de calcular la integral es introduciendo la expresión \command{INT(x\^{}2 log(x), x, C)} en la ventana de Álgebra y haciendo clic en el botón \button{Simplificar}.
\end{indicacion}

\item $\displaystyle \int \frac{\log(\log x)}{x}\,dx$
\begin{indicacion}
Introducir la expresión \command{INT(log(log(x)), x, C)} en la ventana de Álgebra y haciendo clic en el botón \button{Simplicar}.
\end{indicacion}

\item $\displaystyle \int \frac{5x^{2}+4x+1}{x^{5}-2x^{4}+2x^{3}-2x^{2}+x}\,dx$
\begin{indicacion}
Introducir la expresión \command{INT((5x\^{}2+4x+1)/(x\^{}5-2x\^{}4+2x\^{}3-2x\^{}2+x), x, C)} en la ventana de Álgebra y haciendo clic en el botón \button{Simplificar}.
\end{indicacion}

\item $\displaystyle \int \frac{6x+5}{(x^{2}+x+1)^{2}}\,dx$
\begin{indicacion}
Introducir la expresión \command{INT((6x+5)/((x\^{}2+x+1)\^{}2), x, C)} en la ventana de Álgebra y haciendo clic en el botón \button{Simplificar}.
\end{indicacion}
\end{enumerate}


\item Calcular las siguientes integrales definidas:
\begin{enumerate}
\item $\displaystyle \int_{-\frac{1}{2}}^0 \frac{x^{3}}{x^{2}+x+1}\,dx$
\begin{indicacion}
\begin{enumerate}
\item Introducir la expresión \command{x\^{}3/(x\^{}2+x+1)} en la ventana de Álgebra.
\item Seleccionar el menú \menu{Cálculo > Integrales} o hacer clic en el botón \button{Calcular integral}.
\item En el cuadro de diálogo que aparece marcar la opción \option{Definida}, introducir \command{-1/2} en el campo \field{Límite inferior}, introducir \command{0} en el campo \field{Límite superior} y hacer clic en el botón \button{Simplificar}.
\end{enumerate}
Una manera más rápida de calcular la integral es introduciendo la expresión \command{INT(x\^{}3/(x\^{}2+x+1), x, -1/2, 0)} en la ventana de Álgebra y haciendo clic en el botón \button{Simplificar}.
\end{indicacion}

\item $\displaystyle \int_2^4 \frac{\sqrt{16-x^{2}}}{x}\,dx$
\begin{indicacion}
Introducir la expresión \command{INT(sqrt(16-x\^{}2)/x, x, 2, 4)} en la ventana de Álgebra y hacer clic en el botón \button{Simplificar}.
\end{indicacion}

\item $\displaystyle \int_0^{\frac{\pi}{2}} \frac{dx}{3+\cos(2x)}$
\begin{indicacion}
Introducir la expresión \command{INT(sqrt(1/(3+cos(2x), x, 0, pi/2)} en la ventana de Álgebra y hacer clic en el botón \button{Simplificar}.
\end{indicacion}
\end{enumerate}

\item Calcular la siguiente integral impropia
$\int_2^{\infty} x^2e^{-x}\,dx$.
\begin{indicacion}
Introducir la expresión \command{INT(x\^{}2 exp(-x), x, 2, inf)} en la ventana de Álgebra y hacer clic en el botón \button{Simplificar}.
\end{indicacion}



\item Representar la parábola $y=x^{2}-7x+6$, y calcular el área limitada por dicha parábola, el eje de abscisas y las rectas $x=2$ y $x=6$.
\begin{indicacion}
Para dibujar la gráfica de la parábola:
\begin{enumerate}
\item Definir la función en la ventana de Álgebra introduciendo la expresión \command{f(x):=x\^{}2-7x+6}.
\item Abrir una nueva ventana gráfica con el menú \menu{Ventana > Nueva Ventana 2D} y seleccionar el menú \menu{Ventana -> Mosaico Vertical} para ver la ventana de Álgebra y la ventana gráfica a la vez.
\item Hacer clic en el botón \button{Representar expresión} en la ventana gráfica.
\end{enumerate}
Para calcular el área de la región introducir la expresión \command{INT(ABS(f(x)), x, 2, 7)} en la ventana de Álgebra y hacer clic en el botón \button{Simplificar}.

Para dibujar la región introducir la expresión \command{x>2 $\wedge$ x<7 $\wedge$ y>MIN(0,f(x)) $\wedge$ y<MAX(0,f(x))} en la ventana de Álgebra y hacer clic en el botón \button{Representar expresión} en la ventana gráfica.
\end{indicacion}

\item Calcular y dibujar el área comprendida entres las funciones $\sen x$ y $\cos x$ en el intervalo $[0,2\pi]$.
\begin{indicacion}
Para dibujar la región:
\begin{enumerate}
\item Definir la primera función introduciendo la expresión \command{f(x):=sin x} en la ventana de Álgebra.
\item Abrir una nueva ventana gráfica con el menú \menu{Ventana > Nueva Ventana 2D} y seleccionar el menú \menu{Ventana -> Mosaico Vertical} para ver la ventana de Álgebra y la ventana gráfica a la vez.
\item Hacer clic en el botón \button{Representar Expresión} en la ventana gráfica.
\item Definir la segunda función introduciendo la expresión \command{f(x):=cos x} en la ventana de Álgebra y haciendo clic en el botón \button{Representar Expresión} en la ventana gráfica.
\item Introducir la expresión \command{x>0 $\wedge$ x<2pi $\wedge$ y>MIN(f(x),g(x)) $\wedge$ y<MAX(f(x),g(x))} en la ventana de Álgebra y hacer clic en el botón \button{Plot} en la ventana gráfica.
\end{enumerate}
Para calcular el área de la región introducir la expresión \command{INT(ABS(f(x)-g(x)), x, 0, 2pi)} en la ventana de Álgebra y hacer clic en el botón \button{Simplificar}.
\end{indicacion}


\item  Representar gráficamente la región del primer cuadrante limitada por la parábola $y^{2}=8x$, la recta $x=2$ y el eje $OX$, y hallar el volumen generado en la rotación alrededor del eje $OX$ de la región anterior.
\begin{indicacion}
Para dibujar la región:
\begin{enumerate}
\item Definir la función introduciendo la expresión  \command{f(x):=sqrt(8x)} en la ventana de Álgebra.
\item Abrir una nueva ventana gráfica con el menú \menu{Ventana > Nueva Ventana 2D} y seleccionar el menú \menu{Ventana -> Mosaico Vertical} para ver la ventana de Álgebra y la ventana gráfica a la vez.
\item Hacer clic en el botón \button{Representar Expresión} en la ventana gráfica.
\item Introducir la expresión \command{x>0 $\wedge$ x<2 $\wedge$ y>0 $\wedge$ y<f(x)} en la ventana de Álgebra y hacer clic en el botón \button{Representar Expresión} en la ventana gráfica.
\end{enumerate}
Para calcular el volumen del sólido de revolución introducir la expresión \command{INT(pi f(x)\^{}2, x, 0, 2)} en la ventana de Álgebra y hacer clic en el botón \button{Simplificar}.
\end{indicacion}

\end{enumerate}


\section{Ejercicios propuestos}
\begin{enumerate}[leftmargin=*]
\item Calcular las siguientes integrales:
\begin{enumerate}
\item $ \int{\dfrac{2x^{3}+2x^{2}+16}{x(x^{2}+4)^{2}}\;dx}$
\item $ \int{\dfrac{1}{x^{2}\sqrt{4+x^{2}}}\;dx}$
\end{enumerate}

\item Hallar el área encerrada la parábola $y=9-x^{2}$ y la recta $y=-x$.

\item Hallar el área encerrada por la curva $y=e^{-|x|}$ y su asíntota.

\item Hallar el volumen generado en la rotación alrededor del eje $OX$ de la región plana limitada por la parábola $y=2x^{2}$, las rectas
$x=0$, $x=5$ y el eje $OX$, representando previamente dicha región plana.

\item Hallar el volumen generado en la rotación alrededor del eje $OY$ del área limitada por la parábola $y^{2}=8x$ y la recta $x=2$.

\end{enumerate}
